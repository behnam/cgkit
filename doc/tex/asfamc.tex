\section{\module{asfamc} ---
          Acclaim skeleton and motion file (ASF/AMC) reader}

\declaremodule{extension}{cgkit.asfamc}
\modulesynopsis{Acclaim skeleton and motion file (ASF/AMC) reader}

%----------------------------------------------------------------------
\subsection{ASFReader class}

The \class{ASFReader} class reads Acclaim Skeleton Files (ASF) and
calls appropriate methods which have to be implemented in a derived class.

\begin{classdesc}{ASFReader}{filename}
  \var{filename} is the name of the ASF file that should be read.
\end{classdesc}

\begin{methoddesc}{read}{}
Read the entire file.
\end{methoddesc}

\begin{methoddesc}{onVersion}{version}
This method is called when the file format version is encountered.
\var{version} is a float containing the version number.
\end{methoddesc}

\begin{methoddesc}{onName}{name}
This method is called when the skeleton name is read.
\end{methoddesc}

\begin{methoddesc}{onUnits}{units}
This method is called when the units section was read. \var{units}
is a dictionary that contains all definitions that were present in the
units section of the input file. The key is the unit name (such as
\code{mass}, \code{length} and \code{angle}) and the value is the 
corresponding value. If possible the value was cast to float, otherwise
it's still a string.
\end{methoddesc}

\begin{methoddesc}{onDocumentation}{doc}
This method is called when the file documentation was read. \var{doc}
contains the documentation (which may contain several lines).
\end{methoddesc}

\begin{methoddesc}{onRoot}{data}
This method is called when the root section was read. This section contains
information about the root of the skeleton. \var{data} is a dictionary
that contains all the key-value pairs in the root section. The value
is always a tuple (even when it's only one single value). 
\end{methoddesc}

\begin{methoddesc}{onBonedata}{bones}
This method is called after the entire bone data was read. \var{bones}
is a list of bone definitions. Each definition is a data dictionary
containing the key-value pairs in the respective bone section. All values
are tuples (even when it's only one single value). An exception to this
is the \code{limits} attribute which is a list of (\var{min}, \var{max})
tuples that contain the minumum and maximum limits as floats (or as the
special strings \code{"-inf"} and \code{"inf"}).
\end{methoddesc}

\begin{methoddesc}{onHierarchy}{links}
This method is called after the hierarchy section was read. \var{links}
is a list of 2-tuples (\var{parent}, \var{children}) where \var{parent}
is the name of the parent bone and \var{children} is a list of children
bone names.
\end{methoddesc}

%----------------------------------------------------------------------
\subsection{AMCReader class}

The \class{AMCReader} class reads Acclaim Motion Capture Data (AMC) files
and calls \method{onFrame()} for every motion sample in the file.

\begin{classdesc}{AMCReader}{filename}
  \var{filename} is the name of the AMC file that should be read.
\end{classdesc}

\begin{methoddesc}{read}{}
Read the entire file.
\end{methoddesc}

\begin{methoddesc}{onFrame}{framenr, data}
This method is called for every frame. \var{framenr} is the frame number 
and \var{data} is a list of 2-tuples (\var{bone}, \var{values}) where
\var{bone} is a bone name and \var{values} the corresponding 
position/orientation for this frame. The number of values and the
meaning of the values is defined in the corresponding ASF file.
\end{methoddesc}

