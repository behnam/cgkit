\subsection{mat3 - 3x3 matrix}
\label{mat3}

A \class{mat3} represents a 3x3 matrix which can be used to store
linear transformations (if you want to store translations or
perspective transformations, you have to use a \class{mat4}). You can
construct a \class{mat3} in several ways:

\begin{verbatim}
# all components are set to zero
M = mat3()

[   0.0000,    0.0000,    0.0000]
[   0.0000,    0.0000,    0.0000]
[   0.0000,    0.0000,    0.0000]

# identity matrix
M = mat3(1.0)

[   1.0000,    0.0000,    0.0000]
[   0.0000,    1.0000,    0.0000]
[   0.0000,    0.0000,    1.0000]

# The elements on the diagonal are set to 2.5
M = mat3(2.5)

[   2.5000,    0.0000,    0.0000]
[   0.0000,    2.5000,    0.0000]
[   0.0000,    0.0000,    2.5000]

# All elements are explicitly set (values must be given in row-major order)
M = mat3(a,b,c,d,e,f,g,h,i)
M = mat3([a,b,c,d,e,f,g,h,i])

[ a, b, c]
[ d, e, f]
[ g, h, i]

# Create a copy of matrix N (which also has to be a mat3)
M = mat3(N)

# Specify the 3 columns of the matrix (as vec3's or sequences with 3 elements)
M = mat3(a,b,c)

[ a[0], b[0], c[0] ]
[ a[1], b[1], c[1] ]
[ a[2], b[2], c[2] ]

# All elements are explicitly set and are stored inside a string
M = mat3("1,2,3,4,5,6,7,8,9")

[   1.0000,    2.0000,    3.0000]
[   4.0000,    5.0000,    6.0000]
[   7.0000,    8.0000,    9.0000]
\end{verbatim}

%----------------------------------------
{\bf Mathematical operations}

The mathematical operators are supported with the following
combination of types:

\begin{verbatim}
mat3  =  mat3 + mat3
mat3  =  mat3 - mat3
mat3  =  mat3 * mat3
vec3  =  mat3 * vec3
vec3  =  vec3 * mat3
mat3  = float * mat3
mat3  =  mat3 * float
mat3  =  mat3 / float
mat3  =  mat3 % float     # each component
mat3  = -mat3
vec3  =  mat3[i]          # get or set column i (as vec3)
float =  mat3[i,j]        # get or set element in row i and column j
\end{verbatim}

Additionally, you can compare matrices with \code{==} and \code{!=}.

%----------------------------------------
{\bf Methods}

\begin{methoddesc}{getColumn}{index}
Return column index (0-based) as a \class{vec3}.
\end{methoddesc}

\begin{methoddesc}{setColumn}{index, value}
Set column index (0-based) to \var{value} which has to be a sequence
of 3 floats (this includes \class{vec3}).
\end{methoddesc}

\begin{methoddesc}{getRow}{index}
Return row index (0-based) as a \class{vec3}.
\end{methoddesc}

\begin{methoddesc}{setRow}{index, value}
Set row index (0-based) to \var{value} which has to be a sequence of
3 floats (this includes \class{vec3}).
\end{methoddesc}

\begin{methoddesc}{getDiag}{}
Return the diagonal as a \class{vec3}.
\end{methoddesc}

\begin{methoddesc}{setDiag}{value}
Set the diagonal to \var{value} which has to be a sequence of
3 floats (this includes \class{vec3}).
\end{methoddesc}

\begin{methoddesc}{toList}{rowmajor=False}
Returns a list containing the matrix elements. By default, the list is
in column-major order. If you set the optional argument \var{rowmajor} to
\code{True}, you'll get the list in row-major order.
\end{methoddesc}

\begin{methoddesc}{identity}{}
Returns the identity matrix. This is a static method.
\end{methoddesc}

\begin{methoddesc}{transpose}{}
Returns the transpose of the matrix.
\end{methoddesc}

\begin{methoddesc}{determinant}{}
Returns the determinant of the matrix.
\end{methoddesc}

\begin{methoddesc}{inverse}{}
Returns the inverse of the matrix.
\end{methoddesc}

\begin{methoddesc}{scaling}{s}
Returns a scaling transformation. The scaling vector \var{s} must be a
3-sequence (e.g. a \class{vec3}).

\[ \left( \begin{array}{ccc}
s.x & 0 & 0 \\
0 & s.y & 0 \\
0 & 0 & s.z \\
\end{array} \right) \]

This is a static method.
\end{methoddesc}

\begin{methoddesc}{rotation}{angle, axis}
Returns a rotation transformation. The angle must be given in radians,
\var{axis} has to be a 3-sequence (e.g. a \class{vec3}).\\
This is a static method.
\end{methoddesc}

\begin{methoddesc}{scale}{s}
Concatenates a scaling transformation and returns \var{self}. The scaling
vector s must be a 3-sequence (e.g. a \class{vec3}).
\end{methoddesc}

\begin{methoddesc}{rotate}{angle, axis}
Concatenates a rotation transformation and returns \var{self}. The angle
must be given in radians, axis has to be a 3-sequence (e.g. a \class{vec3}).
\end{methoddesc}

\begin{methoddesc}{ortho}{}
Returns a matrix with orthogonal base vectors.
\end{methoddesc}

\begin{methoddesc}{decompose}{}
Decomposes the matrix into a rotation and scaling part. The method
returns a tuple (rotation, scaling). The scaling part is given as a
\class{vec3}, the rotation is still a \class{mat3}.
\end{methoddesc}

\begin{methoddesc}{fromEulerXYZ}{x, y, z}
Returns a rotation matrix created from Euler angles. \var{x} is the angle
around the x axis, \var{y} the angle around the y axis and \var{z} the
angle around the z axis. All angles must be given in radians. The order
of the individual rotations is X-Y-Z (where each axis refers to the {\em
local} axis, i.e. the first rotation is about the x axis which rotates
the Y and Z axis, then the second rotation is about the rotated Y axis 
and so on).\\
This is a static method.
\end{methoddesc}

\begin{methoddesc}{fromEulerYZX}{x, y, z}
See above. The order is YZX.
This is a static method.
\end{methoddesc}

\begin{methoddesc}{fromEulerZXY}{x, y, z}
See above. The order is ZXY.
This is a static method.
\end{methoddesc}

\begin{methoddesc}{fromEulerXZY}{x, y, z}
See above. The order is XZY.
This is a static method.
\end{methoddesc}

\begin{methoddesc}{fromEulerYXZ}{x, y, z}
See above. The order is YXZ.
This is a static method.
\end{methoddesc}

\begin{methoddesc}{fromEulerZYX}{x, y, z}
See above. The order is ZYX.
This is a static method.
\end{methoddesc}

\begin{methoddesc}{toEulerXYZ}{}
Return the Euler angles of a rotation matrix. The order is XYZ.
\end{methoddesc}

\begin{methoddesc}{toEulerYZX}{}
Return the Euler angles of a rotation matrix. The order is YZX.
\end{methoddesc}

\begin{methoddesc}{toEulerZXY}{}
Return the Euler angles of a rotation matrix. The order is ZXY.
\end{methoddesc}

\begin{methoddesc}{toEulerXZY}{}
Return the Euler angles of a rotation matrix. The order is XZY.
\end{methoddesc}

\begin{methoddesc}{toEulerYXZ}{}
Return the Euler angles of a rotation matrix. The order is YXZ.
\end{methoddesc}

\begin{methoddesc}{toEulerZYX}{}
Return the Euler angles of a rotation matrix. The order is ZYX.
\end{methoddesc}

\begin{methoddesc}{fromToRotation}{_from, to}
Returns a rotation matrix that rotates one vector into another.
The generated rotation matrix will rotate the vector \var{_from} into the
vector \var{to}. \var{_from} and \var{to} must be unit vectors!

This method is based on the code from:

Tomas M\"oller and John Hughes\\
{\em Efficiently Building a Matrix to Rotate One Vector to Another}\\
Journal of Graphics Tools, 4(4):1-4, 1999\\
\url{http://www.acm.org/jgt/papers/MollerHughes99/}

This is a static method.
\end{methoddesc}
