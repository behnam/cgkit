\section{\module{glslangtokenize} ---
          Tokenizer for the OpenGL 2 shading language}

\declaremodule{extension}{cgkit.glslangtokenize}
\modulesynopsis{Tokenizer for the OpenGL 2 shading language}

This module provides a lexical scanner for the OpenGL 2 shading language.
You use it in the same way as the Python tokenize module is used.

\begin{funcdesc}{tokenize}{readline, tokeneater}
Reads an input stream and creates tokens. The first parameter,
\var{readline}, must be a callable object which provides the same interface
as the \method{readline()} method of built-in file objects. Each call to the
function should return one line of input as a string.

The second parameter, \var{tokeneater}, must also be a callable
object. It is called with six parameters: the token type, the token
string, a tuple (srow, scol) specifying the row and column where the
token begins in the source, a tuple (erow, ecol) giving the ending
position of the token, the line on which the token was found and the
filename of the current file.

The token type can be one of 

\begin{itemize}
\item \code{WHITESPACE}: This is a series of blanks and/or tabs. 
\item \code{NAME}: A valid identifier name or keyword. 
\item \code{NUMBER}: An integer or float. 
\item \code{STRING}: A string enclosed in '"'. 
\item \code{NEWLINE}: A newline character. 
\item \code{OPERATOR}: An operator such as '+', '-', '!', '==', '!=', etc. 
\item \code{CHARACTER}: A single character that doesn't fit anything else. 
\item \code{TYPE}: A language type (void, bool, float, int, vec2, vec3, vec4, etc.)
\item \code{QUALIFIER}: A type qualifier (const, attribute, uniform, varying, in, out, inout)
\end{itemize}

By default, the filename argument is an empty string. It will only be
the actual filename if you provide a preprocessed file stream as input
(so you should first run \code{cpp} on any shader). The tokenizer
actually expects preprocessed data as it does not handle comments.
\end{funcdesc}
