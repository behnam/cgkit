\section{\module{cgtypes} ---
         Vectors, matrices and quaternions}

\declaremodule{extension}{cgkit.cgtypes}
\modulesynopsis{Basic types useful for computer graphics.}

This module contains the fundamental types which make working with 3D
data much easier. The types are:

\begin{itemize}
\item vec3 -- a three dimensional vector type to store points, vectors, normals or even colors.
\item vec4 -- a four dimensional vector type to store homogenous points, for example.
\item mat3 -- a 3x3 matrix to store linear transformations.
\item mat4 -- a 4x4 matrix to store affine transformations.
\item quat -- a quaternion type as a specialized way to store rotations.
\end{itemize}

You import all of those types at once with

\begin{verbatim}
from cgkit.cgtypes import *
\end{verbatim}

or you can import them individually like this

\begin{verbatim}
from cgkit.cgtypes import vec3, mat4
\end{verbatim}

In general, you can use those types just as if they were built-in
types which means the mathematical operators can be used and have
their respective meaning. Each type has some additional methods which
are described in the respective documentation.

Here are some examples:

\begin{verbatim}
>>> from cgkit.cgtypes import *
>>> v=vec3(0.5,1.0,-2.5)
>>> print v
(0.5000, 1.0000, -2.5000)
>>> print v.length()
2.73861278753
>>> v=v.normalize()
>>> print v
(0.1826, 0.3651, -0.9129)
>>> print v.length()
1.0
\end{verbatim}

Now let's construct a rotation matrix that rotates points by 90
degrees around the z-axis:

\begin{verbatim}
>>> M=mat4(1).rotate(0.5*math.pi, vec3(0,0,1))
\end{verbatim}

and apply the rotation to the vector (1,0,0) (the x-axis):

\begin{verbatim}
>>> print M*vec3(1,0,0)
(0.0000, 1.0000, 0.0000)
\end{verbatim}

The module contains the following functions:

\begin{funcdesc}{getEpsilon}{}
Return the epsilon threshold which is used for doing comparisons.
\end{funcdesc}

\begin{funcdesc}{setEpsilon}{eps}
Sets a new epsilon threshold and returns the previously set value. Two
values are considered to be equal if their absolute difference is less
than or equal to epsilon.
\end{funcdesc}

\begin{funcdesc}{slerp}{t, q0, q1, shortest=True}
Performs a spherical linear interpolation between two quaternions \var{q0}
and \var{q1}. For \var{t}=0.0 the return value equals \var{q0}, for 
\var{t}=1.0 it equals \var{q1}. \var{q0} and \var{q1} must be unit 
quaternions. If \var{shortest} is \code{True} the interpolation will always
be along the shortest path, otherwise it depends on the orientation of
the input quaternions whether the shortest or longest path will be taken
(you can switch between the paths by negating either \var{q0} or \var{q1}).
\end{funcdesc}

\begin{funcdesc}{squad}{t, a, b, c, d}
Performs a spherical cubic interpolation between quaternion \var{a} and \var{d}
where quaternion \var{b} and \var{c} define the shape of the interpolation
curve. For \var{t}=0.0 the return value equals \var{a}, for \var{t}=1.0 it 
equals \var{d}. All quaternions must be unit quaternions.
\end{funcdesc}

%----------------------------------------------------------------------
\subsection{vec3 - 3d vector}
\label{vec3}

A \class{vec3} represents a 3D vector type that can be used to store
points, vector, normals or even colors. You can construct a
\class{vec3} by several ways:

\begin{verbatim}
# all components are set to zero
v = vec3()

-> (0.0000, 0.0000, 0.0000)

# set all components to one value
v = vec3(2.5)

-> (2.5000, 2.5000, 2.5000)

# set a 2d vector, the 3rd component will be zero
v = vec3(1.5, 0.8)

-> (1.5000, 0.8000, 0.0000)

# initialize all three components
v = vec3(1.5, 0.8, -0.5)

-> (1.5000, 0.8000, -0.5000)
\end{verbatim}

Additionally you can use all of the above, but store the values inside
a tuple, a list or a string:

\begin{verbatim}
v = vec3([1.5, 0.8, -0.5])
w = vec3("1.5, 0.8")
\end{verbatim}

Finally, you can initialize a vector with a copy of another vector:

\begin{verbatim}
v = vec3(w)
\end{verbatim}

A \class{vec3} can be used just like a list with 3 elements, so you
can read and write components using the index operator or by accessing
the components by name:

\begin{verbatim}
>>> v=vec3(1,2,3)
>>> print v[0]
1.0
>>> print v.y
2.0
\end{verbatim}

%----------------------------------------
{\bf Mathematical operations}

The mathematical operators are supported with the following
combination of types:

\begin{verbatim}
vec3  =  vec3 + vec3
vec3  =  vec3 - vec3
float =  vec3 * vec3      # dot product
vec3  = float * vec3
vec3  =  vec3 * float
vec3  =  vec3 / float
vec3  =  vec3 % float     # each component
vec3  =  vec3 % vec3      # component wise
vec3  = -vec3
float =  vec3[i]          # get or set element
\end{verbatim}

Additionally, you can compare vectors with \code{==}, \code{!=}, \code{<}, 
\code{<=}, \code{>}, \code{>=}. Each
comparison (except \code{<} and \code{>}) takes an epsilon environment
into account, this means two values are considered to be equal if
their absolute difference is less than or equal to a threshold value
epsilon. You can read and write this threshold value using the
functions \function{getEpsilon()} and \function{setEpsilon()}.

Taking the absolute value of a vector will return the length of the vector: 

\begin{verbatim}
float = abs(v)            # this is equivalent to v.length()
\end{verbatim}

%----------------------------------------
{\bf Methods}

\begin{methoddesc}{angle}{other}
Return angle (in radians) between \var{self} and \var{other}.
\end{methoddesc}

\begin{methoddesc}{cross}{other}
Return cross product of \var{self} and \var{other}.
\end{methoddesc}

\begin{methoddesc}{length}{}
Returns the length of the vector ($\sqrt{x^2+y^2+z^2}$). This is
equivalent to calling \code{abs(self)}.
\end{methoddesc}

\begin{methoddesc}{normalize}{}
Returns normalized vector. If the method is called on the null vector
(where each component is zero) a \exception{ZeroDivisionError} is raised.
\end{methoddesc}

\begin{methoddesc}{reflect}{N}
Returns the reflection vector. \var{N} is the surface normal which has to be
of unit length.
\end{methoddesc}

\begin{methoddesc}{refract}{N, eta}
Returns the transmitted vector. \var{N} is the surface normal which has to
be of unit length. \var{eta} is the relative index of refraction. If the
returned vector is zero then there is no transmitted light because of
total internal reflection.
\end{methoddesc}

\begin{methoddesc}{ortho}{}
Returns a vector that is orthogonal to \var{self} (where
\code{self*self.ortho()==0}).
\end{methoddesc}

\begin{methoddesc}{min}{}
Returns the minimum value of the components.
\end{methoddesc}

\begin{methoddesc}{max}{}
Returns the maximum value of the components.
\end{methoddesc}

\begin{methoddesc}{minIndex}{}
Return the index of the component with the minimum value.
\end{methoddesc}

\begin{methoddesc}{maxIndex}{}
Return the index of the component with the maximum value.
\end{methoddesc}

\begin{methoddesc}{minAbs}{}
Returns the minimum absolute value of the components.
\end{methoddesc}

\begin{methoddesc}{maxAbs}{}
Returns the maximum absolute value of the components.
\end{methoddesc}

\begin{methoddesc}{minAbsIndex}{}
Return the index of the component with the minimum absolute value.
\end{methoddesc}

\begin{methoddesc}{maxAbsIndex}{}
Return the index of the component with the maximum absolute value.
\end{methoddesc}


%----------------------------------------------------------------------
\subsection{vec4 - 4d vector}
\label{vec4}

A \class{vec4} represents a 4D vector type. You can construct a \class{vec4}
by several ways:

\begin{verbatim}
# all components are set to zero
v = vec4()

-> (0.0000, 0.0000, 0.0000, 0.0000)

# set all components to one value
v = vec4(2.5)

-> (2.5000, 2.5000, 2.5000, 2.5000)

# set a 2d vector, the ramaining components will be zero
v = vec4(1.5, 0.8)

-> (1.5000, 0.8000, 0.0000, 0.0000)

# set a 3d vector, the ramaining component will be zero
v = vec4(1.5, 0.8, -0.5)

-> (1.5000, 0.8000, -0.5000, 0.0000)

# set all components
v = vec4(1.5, 0.8, -0.5, 0.2)

-> (1.5000, 0.8000, -0.5000, 0.2000)
\end{verbatim}

Additionally you can use all of the above, but store the values inside
a tuple, a list or a string:

\begin{verbatim}
v = vec4([1.5, 0.8, -0.5])
w = vec4("1.5, 0.8")
\end{verbatim}

Finally, you can initialize a vector with a copy of another vector:

\begin{verbatim}
v = vec4(w)
\end{verbatim}

A \class{vec4} can be used just like a list with 4 elements, so you
can read and write components using the index operator or by accessing
the components by name:

\begin{verbatim}
>>> v=vec4(1,2,3,1)
>>> print v[0]
1.0
>>> print v.y
2.0
>>> print v.w
1.0
>>> print v.t   # this is the same as v.w
1.0
\end{verbatim}

The 4th component can be accessed either by the name \code{w} or
\code{t}. You might prefer the former name when using the vector as a
homogenous coordinate while the latter might be preferable when the
4th component shall represent a time value.

%----------------------------------------
{\bf Mathematical operations}

The mathematical operators are supported with the following
combination of types:

\begin{verbatim}
vec4  =  vec4 + vec4
vec4  =  vec4 - vec4
float =  vec4 * vec4      # dot product
vec4  = float * vec4
vec4  =  vec4 * float
vec4  =  vec4 / float
vec4  =  vec4 % float     # each component
vec4  =  vec4 % vec4      # component wise
vec4  = -vec4
float =  vec4[i]          # get or set element
\end{verbatim}

Additionally, you can compare vectors with \code{==}, \code{!=}, \code{<}, 
\code{<=}, \code{>}, \code{>=}. Each
comparison (except \code{<} and \code{>}) takes an epsilon environment
into account, this means two values are considered to be equal if
their absolute difference is less than or equal to a threshold value
epsilon. You can read and write this threshold value using the
functions \function{getEpsilon()} and \function{setEpsilon()}.

Taking the absolute value of a vector will return the length of the vector: 

\begin{verbatim}
float = abs(v)            # this is equivalent to v.length()
\end{verbatim}

%----------------------------------------
{\bf Methods}

\begin{methoddesc}{length}{}
Returns the length of the vector ($\sqrt{x^2+y^2+z^2}$). This is
equivalent to calling \code{abs(self)}.
\end{methoddesc}

\begin{methoddesc}{normalize}{}
Returns normalized vector. If the method is called on the null vector
(where each component is zero) a \exception{ZeroDivisionError} is raised.
\end{methoddesc}




%----------------------------------------------------------------------
\subsection{mat3 - 3x3 matrix}
\label{mat3}

A \class{mat3} represents a 3x3 matrix which can be used to store
linear transformations (if you want to store translations or
perspective transformations, you have to use a \class{mat4}). You can
construct a \class{mat3} in several ways:

\begin{verbatim}
# all components are set to zero
M = mat3()

[   0.0000,    0.0000,    0.0000]
[   0.0000,    0.0000,    0.0000]
[   0.0000,    0.0000,    0.0000]

# identity matrix
M = mat3(1.0)

[   1.0000,    0.0000,    0.0000]
[   0.0000,    1.0000,    0.0000]
[   0.0000,    0.0000,    1.0000]

# The elements on the diagonal are set to 2.5
M = mat3(2.5)

[   2.5000,    0.0000,    0.0000]
[   0.0000,    2.5000,    0.0000]
[   0.0000,    0.0000,    2.5000]

# All elements are explicitly set (values must be given in row-major order)
M = mat3(a,b,c,d,e,f,g,h,i)
M = mat3([a,b,c,d,e,f,g,h,i])

[ a, b, c]
[ d, e, f]
[ g, h, i]

# Create a copy of matrix N (which also has to be a mat3)
M = mat3(N)

# Specify the 3 columns of the matrix (as vec3's or sequences with 3 elements)
M = mat3(a,b,c)

[ a[0], b[0], c[0] ]
[ a[1], b[1], c[1] ]
[ a[2], b[2], c[2] ]

# All elements are explicitly set and are stored inside a string
M = mat3("1,2,3,4,5,6,7,8,9")

[   1.0000,    2.0000,    3.0000]
[   4.0000,    5.0000,    6.0000]
[   7.0000,    8.0000,    9.0000]
\end{verbatim}

%----------------------------------------
{\bf Mathematical operations}

The mathematical operators are supported with the following
combination of types:

\begin{verbatim}
mat3  =  mat3 + mat3
mat3  =  mat3 - mat3
mat3  =  mat3 * mat3
vec3  =  mat3 * vec3
vec3  =  vec3 * mat3
mat3  = float * mat3
mat3  =  mat3 * float
mat3  =  mat3 / float
mat3  =  mat3 % float     # each component
mat3  = -mat3
vec3  =  mat3[i]          # get or set column i (as vec3)
float =  mat3[i,j]        # get or set element in row i and column j
\end{verbatim}

Additionally, you can compare matrices with \code{==} and \code{!=}.

%----------------------------------------
{\bf Methods}

\begin{methoddesc}{getColumn}{index}
Return column index (0-based) as a \class{vec3}.
\end{methoddesc}

\begin{methoddesc}{setColumn}{index, value}
Set column index (0-based) to \var{value} which has to be a sequence
of 3 floats (this includes \class{vec3}).
\end{methoddesc}

\begin{methoddesc}{getRow}{index}
Return row index (0-based) as a \class{vec3}.
\end{methoddesc}

\begin{methoddesc}{setRow}{index, value}
Set row index (0-based) to \var{value} which has to be a sequence of
3 floats (this includes \class{vec3}).
\end{methoddesc}

\begin{methoddesc}{getDiag}{}
Return the diagonal as a \class{vec3}.
\end{methoddesc}

\begin{methoddesc}{setDiag}{value}
Set the diagonal to \var{value} which has to be a sequence of
3 floats (this includes \class{vec3}).
\end{methoddesc}

\begin{methoddesc}{toList}{rowmajor=False}
Returns a list containing the matrix elements. By default, the list is
in column-major order. If you set the optional argument \var{rowmajor} to
\code{True}, you'll get the list in row-major order.
\end{methoddesc}

\begin{methoddesc}{identity}{}
Returns the identity matrix. 
\end{methoddesc}

\begin{methoddesc}{transpose}{}
Returns the transpose of the matrix.
\end{methoddesc}

\begin{methoddesc}{determinant}{}
Returns the determinant of the matrix.
\end{methoddesc}

\begin{methoddesc}{inverse}{}
Returns the inverse of the matrix.
\end{methoddesc}

\begin{methoddesc}{scaling}{s}
Returns a scaling transformation. The scaling vector \var{s} must be a
3-sequence (e.g. a \class{vec3}).

\[ \left( \begin{array}{ccc}
s.x & 0 & 0 \\
0 & s.y & 0 \\
0 & 0 & s.z \\
\end{array} \right) \]
\end{methoddesc}

\begin{methoddesc}{rotation}{angle, axis}
Returns a rotation transformation. The angle must be given in radians,
\var{axis} has to be a 3-sequence (e.g. a \class{vec3}).
\end{methoddesc}

\begin{methoddesc}{scale}{s}
Concatenates a scaling transformation and returns \var{self}. The scaling
vector s must be a 3-sequence (e.g. a \class{vec3}).
\end{methoddesc}

\begin{methoddesc}{rotate}{angle, axis}
Concatenates a rotation transformation and returns \var{self}. The angle
must be given in radians, axis has to be a 3-sequence (e.g. a \class{vec3}).
\end{methoddesc}

\begin{methoddesc}{ortho}{}
Returns a matrix with orthogonal base vectors.
\end{methoddesc}

\begin{methoddesc}{decompose}{}
Decomposes the matrix into a rotation and scaling part. The method
returns a tuple (rotation, scaling). The scaling part is given as a
\class{vec3}, the rotation is still a \class{mat3}.
\end{methoddesc}

\begin{methoddesc}{fromEulerXYZ}{x, y, z}
Initializes a rotation matrix from Euler angles. \var{x} is the angle
around the x axis, \var{y} the angle around the y axis and \var{z} the
angle around the z axis. All angles must be given in radians. The order
of the individual rotations is X-Y-Z (where each axis refers to the {\em
local} axis, i.e. the first rotation is about the x axis which rotates
the Y and Z axis, then the second rotation is about the rotated Y axis 
and so on).
\end{methoddesc}

\begin{methoddesc}{fromEulerYZX}{x, y, z}
See above. The order is YZX.
\end{methoddesc}

\begin{methoddesc}{fromEulerZXY}{x, y, z}
See above. The order is ZXY.
\end{methoddesc}

\begin{methoddesc}{fromEulerXZY}{x, y, z}
See above. The order is XZY.
\end{methoddesc}

\begin{methoddesc}{fromEulerYXZ}{x, y, z}
See above. The order is YXZ.
\end{methoddesc}

\begin{methoddesc}{fromEulerZYX}{x, y, z}
See above. The order is ZYX.
\end{methoddesc}

\begin{methoddesc}{toEulerXYZ}{}
Return the Euler angles of a rotation matrix. The order is XYZ.
\end{methoddesc}

\begin{methoddesc}{toEulerYZX}{}
Return the Euler angles of a rotation matrix. The order is YZX.
\end{methoddesc}

\begin{methoddesc}{toEulerZXY}{}
Return the Euler angles of a rotation matrix. The order is ZXY.
\end{methoddesc}

\begin{methoddesc}{toEulerXZY}{}
Return the Euler angles of a rotation matrix. The order is XZY.
\end{methoddesc}

\begin{methoddesc}{toEulerYXZ}{}
Return the Euler angles of a rotation matrix. The order is YXZ.
\end{methoddesc}

\begin{methoddesc}{toEulerZYX}{}
Return the Euler angles of a rotation matrix. The order is ZYX.
\end{methoddesc}


%----------------------------------------------------------------------
\subsection{mat4 - 4x4 matrix}
\label{mat4}

A \class{mat4} represents a 4x4 matrix which can be used to store
transformations. You can construct a \class{mat4} in several ways:

\begin{verbatim}
# all components are set to zero
M = mat4()

[   0.0000,    0.0000,    0.0000,    0.0000]
[   0.0000,    0.0000,    0.0000,    0.0000]
[   0.0000,    0.0000,    0.0000,    0.0000]
[   0.0000,    0.0000,    0.0000,    0.0000]

# identity matrix
M = mat4(1.0)

[   1.0000,    0.0000,    0.0000,    0.0000]
[   0.0000,    1.0000,    0.0000,    0.0000]
[   0.0000,    0.0000,    1.0000,    0.0000]
[   0.0000,    0.0000,    0.0000,    1.0000]

# The elements on the diagonal are set to 2.5
M = mat4(2.5)

[   2.5000,    0.0000,    0.0000,    0.0000]
[   0.0000,    2.5000,    0.0000,    0.0000]
[   0.0000,    0.0000,    2.5000,    0.0000]
[   0.0000,    0.0000,    0.0000,    2.5000]

# All elements are explicitly set (values must be given in row-major order)
M = mat4(a,b,c,d,e,f,g,h,i,j,k,l,m,n,o,p)
M = mat4([a,b,c,d,e,f,g,h,i,j,k,l,m,n,o,p])

[ a, b, c, d ]
[ e, f, g, h ]
[ i, j, k, l ]
[ m, n, o, p ]

# Create a copy of matrix N (which also has to be a mat4)
M = mat4(N)

# Specify the 4 columns of the matrix (as vec4's or sequences with 4 elements)
M = mat4(a,b,c,d)

[ a[0], b[0], c[0], d[0] ]
[ a[1], b[1], c[1], d[1] ]
[ a[2], b[2], c[2], d[2] ]
[ a[3], b[3], c[3], d[3] ]

# All elements are explicitly set and are stored inside a string
M = mat4("1,2,3,4,5,6,7,8,9,10,11,12,13,14,15,16")

[   1.0000,    2.0000,    3.0000,    4.0000]
[   5.0000,    6.0000,    7.0000,    8.0000]
[   9.0000,   10.0000,   11.0000,   12.0000]
[  13.0000,   14.0000,   15.0000,   16.0000]
\end{verbatim}

%----------------------------------------
{\bf Mathematical operations}

The mathematical operators are supported with the following
combination of types:

\begin{verbatim}
mat4  =  mat4 + mat4
mat4  =  mat4 - mat4
mat4  =  mat4 * mat4
vec4  =  mat4 * vec4
vec4  =  vec4 * mat4
vec3  =  mat4 * vec3      # missing coordinate in vec3 is implicitly set to 1.0
vec3  =  vec3 * mat4      # missing coordinate in vec3 is implicitly set to 1.0
mat4  = float * mat4
mat4  =  mat4 * float
mat4  =  mat4 / float
mat4  =  mat4 % float     # each component
mat4  = -mat4
vec4  =  mat4[i]          # get or set column i (as vec4)
float =  mat4[i,j]        # get or set element in row i and column j
\end{verbatim}

Additionally, you can compare matrices with \code{==} and \code{!=}.

%----------------------------------------
{\bf Methods}

\begin{methoddesc}{getColumn}{index}
Return column index (0-based) as a \class{vec4}.
\end{methoddesc}

\begin{methoddesc}{setColumn}{index, value}
Set column index (0-based) to \var{value} which has to be a sequence
of 4 floats (this includes \class{vec4}).
\end{methoddesc}

\begin{methoddesc}{getRow}{index}
Return row index (0-based) as a \class{vec4}.
\end{methoddesc}

\begin{methoddesc}{setRow}{index, value}
Set row index (0-based) to \var{value} which has to be a sequence of
4 floats (this includes \class{vec4}).
\end{methoddesc}

\begin{methoddesc}{getDiag}{}
Return the diagonal as a \class{vec4}.
\end{methoddesc}

\begin{methoddesc}{setDiag}{value}
Set the diagonal to \var{value} which has to be a sequence of
4 floats (this includes \class{vec4}).
\end{methoddesc}

\begin{methoddesc}{toList}{rowmajor=False}
Returns a list containing the matrix elements. By default, the list is
in column-major order (which can be directly used in OpenGL or RenderMan). 
If you set the optional argument \var{rowmajor} to
\code{True}, you will get the list in row-major order.
\end{methoddesc}

\begin{methoddesc}{identity}{}
Returns the identity matrix. 
\end{methoddesc}

\begin{methoddesc}{transpose}{}
Returns the transpose of the matrix.
\end{methoddesc}

\begin{methoddesc}{determinant}{}
Returns the determinant of the matrix.
\end{methoddesc}

\begin{methoddesc}{inverse}{}
Returns the inverse of the matrix.
\end{methoddesc}

\begin{methoddesc}{translation}{t}
Returns a translation transformation. The translation vector t must be a
3-sequence (e.g. a vec3).

\[ \left( \begin{array}{cccc}
1 & 0 & 0 & t.x \\
0 & 1 & 0 & t.x \\
0 & 0 & 1 & t.x \\
0 & 0 & 0 & 1 
\end{array} \right) \]

\end{methoddesc}

\begin{methoddesc}{scaling}{s}
Returns a scaling transformation. The scaling vector \var{s} must be a
3-sequence (e.g. a \class{vec3}).

\[ \left( \begin{array}{cccc}
s.x & 0 & 0 & 0\\
0 & s.y & 0 & 0\\
0 & 0 & s.z & 0\\
0 & 0 & 0 & 1\\
\end{array} \right) \]
\end{methoddesc}

\begin{methoddesc}{rotation}{angle, axis}
Returns a rotation transformation. The angle must be given in radians,
\var{axis} has to be a 3-sequence (e.g. a \class{vec3}).
\end{methoddesc}

\begin{methoddesc}{translate}{t}
Concatenate a translation transformation and return \var{self}. The
translation vector \var{t} must be a 3-sequence (e.g. a \class{vec3}).
\end{methoddesc}

\begin{methoddesc}{scale}{s}
Concatenates a scaling transformation and returns \var{self}. The scaling
vector s must be a 3-sequence (e.g. a \class{vec3}).
\end{methoddesc}

\begin{methoddesc}{rotate}{angle, axis}
Concatenates a rotation transformation and returns \var{self}. The angle
must be given in radians, axis has to be a 3-sequence (e.g. a \class{vec3}).
\end{methoddesc}

\begin{methoddesc}{frustum}{left, right, bottom, top, near, far}
Returns a matrix that represents a perspective depth
transformation. This method is equivalent to the OpenGL command
\cfunction{glFrustum()}.

\note{If you want to use this transformation in RenderMan, keep in
mind that the RenderMan camera is looking in the positive z direction
whereas the OpenGL camera is looking in the negative direction.}
\end{methoddesc}

\begin{methoddesc}{perspective}{fovy, aspect, near, far}
Similarly to \method{frustum()} this method returns a perspective
transformation, the only difference is the meaning of the input
parameters. The method is equivalent to the OpenGL command
\cfunction{gluPerspective()}.
\end{methoddesc}

\begin{methoddesc}{orthographic}{left, right, bottom, top, near, far}
Returns a matrix that represents an orthographic transformation. This
method is equivalent to the OpenGL command \cfunction{glOrtho()}.
\end{methoddesc}

\begin{methoddesc}{lookAt}{pos, target, up=(0,0,1)}
Returns a transformation that moves the coordinate system to position
\var{pos} and rotates it so that the z-axis points onto point
\var{target}. The y-axis is pointing as closely as possible into the
same direction as \var{up}. All three parameters have to be a
3-sequence.

You can use this transformation to position objects in the scene that
have to be oriented towards a particular point. Or you can use it to
position the camera in the virtual world. In this case you have to
take the inverse of this transformation as viewport transformation (to
convert world space coordinates into camera space coordinates).
\end{methoddesc}

\begin{methoddesc}{ortho}{}
Returns a matrix with orthogonal base vectors. The x-, y- and z-axis
are made orthogonal. The fourth column and row remain untouched.
\end{methoddesc}

\begin{methoddesc}{decompose}{}
Decomposes the matrix into a translation, rotation and scaling. The
method returns a tuple (translation, rotation, scaling). The
translation and scaling parts are given as \class{vec3}, the rotation is
still given as a \class{mat4}.
\end{methoddesc}

\begin{methoddesc}{getMat3}{}
Returns a \class{mat3} which is a copy of self without the 4th column and row.
\end{methoddesc}

\begin{methoddesc}{setMat3}{m3}
Sets the first three columns and rows to the values in \var{m3}.
\end{methoddesc}




%----------------------------------------------------------------------
\subsection{quat - quaternions}
\label{quat}

A \class{quat} represents a quaternion type that can be used to store
rotations. A quaternion contains four values of which one can be seen
as the angle and the other three as the axis of rotation. The most
common way to initialize a quaternion is by specifying an angle (in
radians) and the axis of rotation:

\begin{verbatim}
# initialize the quaternion by specifying an angle and the axis of rotation
q = quat(0.5*pi, vec3(0,0,1))

# initialize by specifying a rotation matrix (as mat3 or mat4)
q = quat(R)

# all components are set to zero
q = quat()

(0.0000, 0.0000, 0.0000, 0.0000)

# set the w component
q = quat(2.5)

(0.5000, 0.0000, 0.0000, 0.0000)

# set all four components (w,x,y,z)
q = quat(1,0,0,0)
q = quat([1,0,0,0])
q = quat("1,0,0,0")

(1.0000, 0.0000, 0.0000, 0.0000)
\end{verbatim}

Finally, you can initialize a quaternion with a copy of another quaternion:

\begin{verbatim}
q = quat(r)
\end{verbatim}

%----------------------------------------
{\bf Mathematical operations}

The mathematical operators are supported with the following
combination of types:

\begin{verbatim}
quat  =  quat + quat
quat  =  quat - quat
quat  =  quat * quat
quat  = float * quat
quat  =  quat * float
quat  =  quat / float
quat  = -quat
quat  =  quat ** float = pow(quat, float)   (new in version 1.1)
quat  =  quat ** quat  = pow(quat, quat)    (new in version 1.1)
\end{verbatim}

Additionally, you can compare quaternions with \code{==} and \code{!=}. 
Taking the absolute value will return the magnitude of the quaternion:

\begin{verbatim}
float = abs(q)
\end{verbatim}

%----------------------------------------
{\bf Methods}

\begin{methoddesc}{conjugate}{}
Return the conjugate $(w, -x, -y, -z)$ of the quaternion.
\end{methoddesc}

\begin{methoddesc}{normalize}{}
Returns the normalized quaternion. If the method is called on the null
vector a \exception{ZeroDivisionError} is raised.
\end{methoddesc}

\begin{methoddesc}{inverse}{}
Return the inverse of the quaternion.
\end{methoddesc}

\begin{methoddesc}{toAngleAxis}{}
Returns a tuple containing the angle (in radians) and the axis of rotation.
The returned axis can also be zero if the rotation is actually the identity.
\end{methoddesc}

\begin{methoddesc}{fromAngleAxis}{angle, axis}
Initializes \var{self} from an angle (in radians) and an axis of
rotation and returns \var{self}. The initialized quaternion will be a
unit quaternion. Passing the null vector as axis has the same effect
as passing an angle of 0 (i.e. the quaternion will be set to (1,0,0,0)).
\end{methoddesc}

\begin{methoddesc}{toMat3}{}
Convert the quaternion into a rotation matrix and return the matrix as a 
\class{mat3}.
\end{methoddesc}

\begin{methoddesc}{toMat4}{}
Convert the quaternion into a rotation matrix and return the matrix as a 
\class{mat4}.
\end{methoddesc}

\begin{methoddesc}{fromMat}{matrix}
Initialize \var{self} from a rotation matrix, given either as a
\class{mat3} or a \class{mat4} and returns \var{self}. \var{matrix} must be
a rotation matrix (i.e. the determinant is 1), if you have a matrix
that is made up of other parts as well, call \method{matrix.decompose()} to get
the rotation part.
\end{methoddesc}

\begin{methoddesc}{dot}{b}
Returns the dot product of \var{self} with quaternion \var{b}.\\
New in version 1.1.
\end{methoddesc}

\begin{methoddesc}{log}{}
Returns the natural logarithm of \var{self}.\\
New in version 1.1.
\end{methoddesc}

\begin{methoddesc}{exp}{}
Returns the exponential of \var{self}. \\
New in version 1.1.
\end{methoddesc}

%----------------------------------------
{\bf Related functions}

\begin{funcdesc}{slerp}{t, q0, q1, shortest=True}
Performs a spherical linear interpolation between two quaternions \var{q0}
and \var{q1}. For \var{t}=0.0 the return value equals \var{q0}, for 
\var{t}=1.0 it equals \var{q1}. \var{q0} and \var{q1} must be unit 
quaternions. If \var{shortest} is \code{True} the interpolation will always
be along the shortest path, otherwise it depends on the orientation of
the input quaternions whether the shortest or longest path will be taken
(you can switch between the paths by negating either \var{q0} or \var{q1}).
\end{funcdesc}

\begin{funcdesc}{squad}{t, a, b, c, d}
Performs a spherical cubic interpolation between quaternion \var{a} and \var{d}
where quaternion \var{b} and \var{c} define the shape of the interpolation
curve. For \var{t}=0.0 the return value equals \var{a}, for \var{t}=1.0 it 
equals \var{d}. All quaternions must be unit quaternions.
\end{funcdesc}





