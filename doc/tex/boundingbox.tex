% BoundingBox

\section{\class{BoundingBox} ---
         An axis aligned bounding box}

\begin{classdesc}{BoundingBox}{[b1, b2]}

\var{b1} and \var{b2} are the initial bounds (as \class{vec3}) of 
the bounding box.
Everything that lies within the specified box (including the bounds)
is considered inside the bounding box. If you don't specify any bounds
the bounding box is initially empty.
\end{classdesc}


\begin{methoddesc}{clear}{}
Make the bounding box empty.
\end{methoddesc}

\begin{methoddesc}{isEmpty}{}
Return \code{True} if the bounding box is empty.
\end{methoddesc}

\begin{methoddesc}{getBounds}{}
Return the minimum and maximum bound. The bounds are returned as
\class{vec3} objects.
\end{methoddesc}

\begin{methoddesc}{setBounds}{b1, b2}
Set new bounds for the bounding box. The rectangle given
by \var{b1} and \var{b2} defines the new bounding box.
\end{methoddesc}

\begin{methoddesc}{addPoint}{p}
Enlarge the bounding box so that the point \var{p} is enclosed in the box.
\end{methoddesc}

\begin{methoddesc}{addBoundingBox}{bb}
Enlarge the bounding box so that the bounding box \var{bb} is enclosed in
the box.
\end{methoddesc}

\begin{methoddesc}{transform}{M}
Returns a transformed bounding box. The transformation is given by \var{M}
which must be a \class{mat4}. The result will still be axis aligned, so the
volume will not be preserved.
\end{methoddesc}
