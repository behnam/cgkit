\subsection{quat - quaternions}
\label{quat}

A \class{quat} represents a quaternion type that can be used to store
rotations. A quaternion contains four values of which one can be seen
as the angle and the other three as the axis of rotation. The most
common way to initialize a quaternion is by specifying an angle (in
radians) and the axis of rotation:

\begin{verbatim}
# initialize the quaternion by specifying an angle and the axis of rotation
q = quat(0.5*pi, vec3(0,0,1))

# initialize by specifying a rotation matrix (as mat3 or mat4)
q = quat(R)

# all components are set to zero
q = quat()

(0.0000, 0.0000, 0.0000, 0.0000)

# set the w component
q = quat(2.5)

(0.5000, 0.0000, 0.0000, 0.0000)

# set all four components (w,x,y,z)
q = quat(1,0,0,0)
q = quat([1,0,0,0])
q = quat("1,0,0,0")

(1.0000, 0.0000, 0.0000, 0.0000)
\end{verbatim}

Finally, you can initialize a quaternion with a copy of another quaternion:

\begin{verbatim}
q = quat(r)
\end{verbatim}

%----------------------------------------
{\bf Mathematical operations}

The mathematical operators are supported with the following
combination of types:

\begin{verbatim}
quat  =  quat + quat
quat  =  quat - quat
quat  =  quat * quat
quat  = float * quat
quat  =  quat * float
quat  =  quat / float
quat  = -quat
quat  =  quat ** float = pow(quat, float)   (new in version 1.1)
quat  =  quat ** quat  = pow(quat, quat)    (new in version 1.1)
\end{verbatim}

Additionally, you can compare quaternions with \code{==} and \code{!=}. 
Taking the absolute value will return the magnitude of the quaternion:

\begin{verbatim}
float = abs(q)
\end{verbatim}

%----------------------------------------
{\bf Methods}

\begin{methoddesc}{conjugate}{}
Return the conjugate $(w, -x, -y, -z)$ of the quaternion.
\end{methoddesc}

\begin{methoddesc}{normalize}{}
Returns the normalized quaternion. If the method is called on the null
vector a \exception{ZeroDivisionError} is raised.
\end{methoddesc}

\begin{methoddesc}{inverse}{}
Return the inverse of the quaternion.
\end{methoddesc}

\begin{methoddesc}{toAngleAxis}{}
Returns a tuple containing the angle (in radians) and the axis of rotation.
The returned axis can also be zero if the rotation is actually the identity.
\end{methoddesc}

\begin{methoddesc}{fromAngleAxis}{angle, axis}
Initializes \var{self} from an angle (in radians) and an axis of
rotation and returns \var{self}. The initialized quaternion will be a
unit quaternion. Passing the null vector as axis has the same effect
as passing an angle of 0 (i.e. the quaternion will be set to (1,0,0,0)).
\end{methoddesc}

\begin{methoddesc}{toMat3}{}
Convert the quaternion into a rotation matrix and return the matrix as a 
\class{mat3}.
\end{methoddesc}

\begin{methoddesc}{toMat4}{}
Convert the quaternion into a rotation matrix and return the matrix as a 
\class{mat4}.
\end{methoddesc}

\begin{methoddesc}{fromMat}{matrix}
Initialize \var{self} from a rotation matrix, given either as a
\class{mat3} or a \class{mat4} and returns \var{self}. \var{matrix} must be
a rotation matrix (i.e. the determinant is 1), if you have a matrix
that is made up of other parts as well, call \method{matrix.decompose()} to get
the rotation part.
\end{methoddesc}

\begin{methoddesc}{dot}{b}
Returns the dot product of \var{self} with quaternion \var{b}.\\
New in version 1.1.
\end{methoddesc}

\begin{methoddesc}{log}{}
Returns the natural logarithm of \var{self}.\\
New in version 1.1.
\end{methoddesc}

\begin{methoddesc}{exp}{}
Returns the exponential of \var{self}. \\
New in version 1.1.
\end{methoddesc}

%----------------------------------------
{\bf Related functions}

\begin{funcdesc}{slerp}{t, q0, q1}
Performs a spherical linear interpolation between two quaternions q0
and q1. For t=0.0 the return value equals q0, for t=1.0 it equals
q1. q0 and q1 must be unit quaternions. \\
New in version 1.1.
\end{funcdesc}

\begin{funcdesc}{squad}{t, a, b, c, d}
Performs a spherical cubic interpolation between quaternion a and d
where quaternion b and c define the shape of the interpolation
curve. For t=0.0 the return value equals a, for t=1.0 it equals d. All
quaternions must be unit quaternions. \\
New in version 1.1.
\end{funcdesc}



