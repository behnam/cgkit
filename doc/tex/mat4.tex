\subsection{mat4 - 4x4 matrix}
\label{mat4}

A \class{mat4} represents a 4x4 matrix which can be used to store
transformations. You can construct a \class{mat4} in several ways:

\begin{verbatim}
# all components are set to zero
M = mat4()

[   0.0000,    0.0000,    0.0000,    0.0000]
[   0.0000,    0.0000,    0.0000,    0.0000]
[   0.0000,    0.0000,    0.0000,    0.0000]
[   0.0000,    0.0000,    0.0000,    0.0000]

# identity matrix
M = mat4(1.0)

[   1.0000,    0.0000,    0.0000,    0.0000]
[   0.0000,    1.0000,    0.0000,    0.0000]
[   0.0000,    0.0000,    1.0000,    0.0000]
[   0.0000,    0.0000,    0.0000,    1.0000]

# The elements on the diagonal are set to 2.5
M = mat4(2.5)

[   2.5000,    0.0000,    0.0000,    0.0000]
[   0.0000,    2.5000,    0.0000,    0.0000]
[   0.0000,    0.0000,    2.5000,    0.0000]
[   0.0000,    0.0000,    0.0000,    2.5000]

# All elements are explicitly set (values must be given in row-major order)
M = mat4(a,b,c,d,e,f,g,h,i,j,k,l,m,n,o,p)
M = mat4([a,b,c,d,e,f,g,h,i,j,k,l,m,n,o,p])

[ a, b, c, d ]
[ e, f, g, h ]
[ i, j, k, l ]
[ m, n, o, p ]

# Create a copy of matrix N (which also has to be a mat4)
M = mat4(N)

# Specify the 4 columns of the matrix (as vec4's or sequences with 4 elements)
M = mat4(a,b,c,d)

[ a[0], b[0], c[0], d[0] ]
[ a[1], b[1], c[1], d[1] ]
[ a[2], b[2], c[2], d[2] ]
[ a[3], b[3], c[3], d[3] ]

# All elements are explicitly set and are stored inside a string
M = mat4("1,2,3,4,5,6,7,8,9,10,11,12,13,14,15,16")

[   1.0000,    2.0000,    3.0000,    4.0000]
[   5.0000,    6.0000,    7.0000,    8.0000]
[   9.0000,   10.0000,   11.0000,   12.0000]
[  13.0000,   14.0000,   15.0000,   16.0000]
\end{verbatim}

%----------------------------------------
{\bf Mathematical operations}

The mathematical operators are supported with the following
combination of types:

\begin{verbatim}
mat4  =  mat4 + mat4
mat4  =  mat4 - mat4
mat4  =  mat4 * mat4
vec4  =  mat4 * vec4
vec4  =  vec4 * mat4
vec3  =  mat4 * vec3      # missing coordinate in vec3 is implicitly set to 1.0
vec3  =  vec3 * mat4      # missing coordinate in vec3 is implicitly set to 1.0
mat4  = float * mat4
mat4  =  mat4 * float
mat4  =  mat4 / float
mat4  =  mat4 % float     # each component
mat4  = -mat4
vec4  =  mat4[i]          # get or set column i (as vec4)
float =  mat4[i,j]        # get or set element in row i and column j
\end{verbatim}

Additionally, you can compare matrices with \code{==} and \code{!=}.

%----------------------------------------
{\bf Methods}

\begin{methoddesc}{getColumn}{index}
Return column index (0-based) as a \class{vec4}.
\end{methoddesc}

\begin{methoddesc}{setColumn}{index, value}
Set column index (0-based) to \var{value} which has to be a sequence
of 4 floats (this includes \class{vec4}).
\end{methoddesc}

\begin{methoddesc}{getRow}{index}
Return row index (0-based) as a \class{vec4}.
\end{methoddesc}

\begin{methoddesc}{setRow}{index, value}
Set row index (0-based) to \var{value} which has to be a sequence of
4 floats (this includes \class{vec4}).
\end{methoddesc}

\begin{methoddesc}{getDiag}{}
Return the diagonal as a \class{vec4}.
\end{methoddesc}

\begin{methoddesc}{setDiag}{value}
Set the diagonal to \var{value} which has to be a sequence of
4 floats (this includes \class{vec4}).
\end{methoddesc}

\begin{methoddesc}{toList}{rowmajor=False}
Returns a list containing the matrix elements. By default, the list is
in column-major order (which can be directly used in OpenGL or RenderMan). 
If you set the optional argument \var{rowmajor} to
\code{True}, you will get the list in row-major order.
\end{methoddesc}

\begin{methoddesc}{identity}{}
Returns the identity matrix. 
\end{methoddesc}

\begin{methoddesc}{transpose}{}
Returns the transpose of the matrix.
\end{methoddesc}

\begin{methoddesc}{determinant}{}
Returns the determinant of the matrix.
\end{methoddesc}

\begin{methoddesc}{inverse}{}
Returns the inverse of the matrix.
\end{methoddesc}

\begin{methoddesc}{translation}{t}
Returns a translation transformation. The translation vector t must be a
3-sequence (e.g. a vec3).

\[ \left( \begin{array}{cccc}
1 & 0 & 0 & t.x \\
0 & 1 & 0 & t.x \\
0 & 0 & 1 & t.x \\
0 & 0 & 0 & 1 
\end{array} \right) \]

\end{methoddesc}

\begin{methoddesc}{scaling}{s}
Returns a scaling transformation. The scaling vector \var{s} must be a
3-sequence (e.g. a \class{vec3}).

\[ \left( \begin{array}{cccc}
s.x & 0 & 0 & 0\\
0 & s.y & 0 & 0\\
0 & 0 & s.z & 0\\
0 & 0 & 0 & 1\\
\end{array} \right) \]
\end{methoddesc}

\begin{methoddesc}{rotation}{angle, axis}
Returns a rotation transformation. The angle must be given in radians,
\var{axis} has to be a 3-sequence (e.g. a \class{vec3}).
\end{methoddesc}

\begin{methoddesc}{translate}{t}
Concatenate a translation transformation and return \var{self}. The
translation vector \var{t} must be a 3-sequence (e.g. a \class{vec3}).
\end{methoddesc}

\begin{methoddesc}{scale}{s}
Concatenates a scaling transformation and returns \var{self}. The scaling
vector s must be a 3-sequence (e.g. a \class{vec3}).
\end{methoddesc}

\begin{methoddesc}{rotate}{angle, axis}
Concatenates a rotation transformation and returns \var{self}. The angle
must be given in radians, axis has to be a 3-sequence (e.g. a \class{vec3}).
\end{methoddesc}

\begin{methoddesc}{frustum}{left, right, bottom, top, near, far}
Returns a matrix that represents a perspective depth
transformation. This method is equivalent to the OpenGL command
\cfunction{glFrustum()}.

\note{If you want to use this transformation in RenderMan, keep in
mind that the RenderMan camera is looking in the positive z direction
whereas the OpenGL camera is looking in the negative direction.}
\end{methoddesc}

\begin{methoddesc}{perspective}{fovy, aspect, near, far}
Similarly to \method{frustum()} this method returns a perspective
transformation, the only difference is the meaning of the input
parameters. The method is equivalent to the OpenGL command
\cfunction{gluPerspective()}.
\end{methoddesc}

\begin{methoddesc}{orthographic}{left, right, bottom, top, near, far}
Returns a matrix that represents an orthographic transformation. This
method is equivalent to the OpenGL command \cfunction{glOrtho()}.
\end{methoddesc}

\begin{methoddesc}{lookAt}{pos, target, up=(0,0,1)}
Returns a transformation that moves the coordinate system to position
\var{pos} and rotates it so that the z-axis points onto point
\var{target}. The y-axis is pointing as closely as possible into the
same direction as \var{up}. All three parameters have to be a
3-sequence.

You can use this transformation to position objects in the scene that
have to be oriented towards a particular point. Or you can use it to
position the camera in the virtual world. In this case you have to
take the inverse of this transformation as viewport transformation (to
convert world space coordinates into camera space coordinates).
\end{methoddesc}

\begin{methoddesc}{ortho}{}
Returns a matrix with orthogonal base vectors. The x-, y- and z-axis
are made orthogonal. The fourth column and row remain untouched.
\end{methoddesc}

\begin{methoddesc}{decompose}{}
Decomposes the matrix into a translation, rotation and scaling. The
method returns a tuple (translation, rotation, scaling). The
translation and scaling parts are given as \class{vec3}, the rotation is
still given as a \class{mat4}.
\end{methoddesc}

\begin{methoddesc}{getMat3}{}
Returns a \class{mat3} which is a copy of self without the 4th column and row.
\end{methoddesc}

\begin{methoddesc}{setMat3}{m3}
Sets the first three columns and rows to the values in \var{m3}.
\end{methoddesc}


