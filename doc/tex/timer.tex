% Timer component

\section{\class{Timer} ---
         Manages the virtual time}
\label{timer}

The \class{Timer} class manages the virtual time and drives the entire
animation/simulation. Everything that is animated either directly or
indirectly connects to the time slot of the global \class{Timer} instance.
This global timer is automatically created by the scene and can be
obtained by calling the \method{timer()} method of the \class{Scene} class.


\begin{classdesc}{Timer}{name = ''Timer'',\\ 
                         auto_insert = True}


\var{name} is the component name.
\end{classdesc}

% Attributes:

The \class{Timer} class has the following attributes:

\begin{datadesc}{time / time_slot}
Current virtual time in seconds (\code{DoubleSlot}).
This value is increased by \member{timestep} whenever the \method{step()} 
method is called.
\end{datadesc}

\begin{datadesc}{timestep}
The delta time step in seconds that represents one frame.
\end{datadesc}

\begin{datadesc}{duration}
The total duration in seconds of the animation/simulation.
\end{datadesc}

\begin{datadesc}{frame}
Current (fractional) frame number (as float).
\end{datadesc}

\begin{datadesc}{fps}
Current frame rate (as float).
\end{datadesc}

\begin{datadesc}{framecount}
Total number of frames of the animation/simulation (as float).
\end{datadesc}

\begin{datadesc}{clock}
This value contains the value in seconds of the real clock.
\end{datadesc}

% Methods:

The \class{Timer} class has the following methods:

\begin{methoddesc}{startClock}{}
Start the real clock.
\end{methoddesc}

\begin{methoddesc}{stopClock}{}
Stop the real clock.
\end{methoddesc}

\begin{methoddesc}{step}{}
Step to the next frame. This call increases the current virtual time by
\member{timestep} and emits a \code{STEP_FRAME} event.
\end{methoddesc}

