\section{\module{mayaascii} ---
        Reading Maya ASCII files}

\declaremodule{extension}{cgkit.mayaascii}
\modulesynopsis{Reading Maya ASCII files}

This module contains the \class{MAReader} class which can be used as a
base class for reading Maya ASCII files. The class reads the file
and invokes callback methods with the corresponding data in the
file. Derived classes have to implement those callback methods and
process the data as appropriate.

%----------------------------------------------------------------
\subsection{MAReader class}

The \class{MAReader} class reads Maya ASCII files and calls
appropriate methods which have to be implemented in a derived class.
The content of the file is actually a subset of the {\em Maya Embedded
Language} (MEL) which is the scripting language implemented inside
Maya.  The \class{MAReader} parses the file, breaks down the content
of the file in commands and their arguments and options (expressions
are not evaluated). Each MEL command will then trigger a callback
method that has to execute the command.  These callback methods have
to be implemented in a derived class.

There are 11 MEL commands that can appear in a Maya ASCII 
file\footnote{Actually, there could appear any MEL command, but at least
Maya will only export files containing the above commands.}:

\begin{itemize}
\item {\tt file}
\item {\tt requires}
\item {\tt fileInfo}
\item {\tt currentUnit}
\item {\tt createNode}
\item {\tt setAttr}
\item {\tt addAttr}
\item {\tt connectAttr}
\item {\tt disconnectAttr}
\item {\tt parent}
\item {\tt select}
\end{itemize}

Each command has a number of arguments and can also take options. The
callback methods receive the arguments as regular arguments to the
method and the options as an additional argument \code{opts} which is
a dictionary containing the options that were specified in the
file. The key is the long name of the option (without leading dash)
and the value is a list of strings containing the option values.  The
number of values and how they have to be interpreted depend on the
actual option.

\begin{classdesc}{MAReader}{}
  Creates an instance of the reader.
\end{classdesc}

\begin{methoddesc}{read}{f}
Read the content of a file. \var{f} must be a file like object that
can be used to read the content of the file.
\end{methoddesc}

\begin{methoddesc}{begin}{}
Callback method that is called before the file is read.
\end{methoddesc}

\begin{methoddesc}{end}{}
Callback method that is called after the file was read.
\end{methoddesc}

\begin{methoddesc}{onFile}{filename, opts}
Reference an external file.
\end{methoddesc}

\begin{methoddesc}{onRequires}{product, version}
Specify a requirement that is needed to load the file properly.
\var{product} is a string containing the required software component
and \var{version} is a string containing the required version of that
component.
\end{methoddesc}

\begin{methoddesc}{onFileInfo}{keyword, value, opts}
Specifies information about the file. \var{keyword} and \var{value}
are both strings.
\end{methoddesc}

\begin{methoddesc}{onCurrentUnit}{opts}
Specify the units (linear, angular, time) used in the file.
\end{methoddesc}

\begin{methoddesc}{onCreateNode}{nodetype, opts}
Create a new node. \var{nodetype} is a string specifying the type of node
that is to be created. The new node will automatically be selected (i.e.
subsequent setAttr commands refer to this node).
\end{methoddesc}

\begin{methoddesc}{onSetAttr}{attr, vals, opts}
Set a node attribute. \var{attr} is a string containing the attribute
to be set. \var{vals} is a list of values. The number of elements and
the type of each element depends on the attribute.
\end{methoddesc}

\begin{methoddesc}{onAddAttr}{opts}
Add a new attribute to the node.
\end{methoddesc}

\begin{methoddesc}{onConnectAttr}{srcattr, dstattr, opts}
Connect two attributes. \var{srcattr} is a string specifiying the
attribute that serves as a source and \var{dstattr} is the name of
the attribute that will receive the value. 
\end{methoddesc}

\begin{methoddesc}{onDisconnectAttr}{srcattr, dstattr, opts}
Break the attribute connection between two attributes.
\end{methoddesc}

\begin{methoddesc}{onParent}{objects, parent, opts}
Set the parent of one or more nodes. \var{objects} is a list of node names
and \var{parent} the name of the parent.
\end{methoddesc}

\begin{methoddesc}{onSelect}{objects, opts}
Select a node from a referenced file. \var{objects} is a list of strings
containing the node names.
\end{methoddesc}

%----------------------------------------------------------------
\subsection{Node class}

\begin{classdesc}{Node}{nodetype, opts}
  \var{nodetype} and \var{opts} are the arguments of the 
  \method{onCreateNode()} callback of the \class{MAReader} class.
\end{classdesc}

\begin{methoddesc}{getName}{}
Return the name of the node or \code{None} if no name was specified.
\end{methoddesc}

\begin{methoddesc}{getParentName}{}
Return the name of the parent node or \code{None} if no parent was specified.
\end{methoddesc}

\begin{methoddesc}{setAttr}{attr, vals, opts}
\var{attr}, \var{vals} and \var{opts} are the arguments of the 
\method{onSetAttr()} callback of the \class{MAReader} class. The Python
value of an attribute can be obtained by calling \method{getAttrValue()}.
\end{methoddesc}

\begin{methoddesc}{addAttr}{opts}
\var{opts} is the arguments of the \method{onAddAttr()} callback of 
the \class{MAReader} class.
\end{methoddesc}

\begin{methoddesc}{addInConnection}{localattr, node, attrname}
\end{methoddesc}

\begin{methoddesc}{addOutConnection}{localattr, node, nodename, attrname}
\end{methoddesc}

\begin{methoddesc}{getAttrValue}{lname, sname, type, n=1, default=None}
\end{methoddesc}

\begin{methoddesc}{getInNode}{lname, sname}
\end{methoddesc}

\begin{methoddesc}{getOutAttr}{lname, sname, dstnodetype}
\end{methoddesc}

%----------------------------------------------------------------
\subsection{Attribute class}

\begin{classdesc}{Attribute}{attr, vals, opts}

\var{attr}, \var{vals} and \var{opts} are the arguments of the 
\method{onSetAttr()} callback of the \class{MAReader} class.

\end{classdesc}

\begin{methoddesc}{getBaseName}{}
Return the base name of the attribute. This is the first part of the
attribute name (and may actually refer to another attribute).

\begin{verbatim}
  ".t"            -> "t"
  ".ed[0:11]"     -> "ed"
  ".uvst[0].uvsn" -> "uvst"
\end{verbatim}
\end{methoddesc}

\begin{methoddesc}{getFullName}{}
Return the full attribute specifier.
\end{methoddesc}

\begin{methoddesc}{getValue}{type=None, n=None}
Return the value of the attribute as appropriate Python value.
\var{type} is a string containing the required type of the value.
If \code{None} is passed, the method tries to retrieve the value from
the attribute itself. If it fails, an exception is thrown. The following
table lists the valid type strings and their corresponding Python type:

\begin{tableii}{l|l}{code}{type}{Python type}
\lineii{"bool"}{bool}
\lineii{"int"}{int}
\lineii{"float"}{float}
\lineii{"string"}{str}
\lineii{"short2"}{(int, int)}
\lineii{"short3"}{(int, int, int)}
\lineii{"long2"}{(int, int)}
\lineii{"long3"}{(int, int, int)}
\lineii{"int32Array"}{[int, ...]}
\lineii{"float2"}{(float, float)}
\lineii{"float3"}{(float, float, float)}
\lineii{"double2"}{(float, float)}
\lineii{"double3"}{(float, float, float)}
\lineii{"doubleArray"}{[float, ...]}
\lineii{"polyFaces"}{PolyFace (see \ref{polyface})}
\lineii{"nurbsSurface"}{NurbsSurface (see \ref{nurbssurface})}
\lineii{"nurbsCurve"}{NurbsCurve (see \ref{nurbscurve})}
\end{tableii}

The argument \var{n} specifies how many values are expected. An exception
is thrown if the number of values that were set by the \code{setAttr} call
doesn't match the required number. If \code{None} is passed, an arbitrary
number of values is allowed. The value of \var{n} also influences the
return type. If the value is 1 the method will return one of the types
in the above table, otherwise it will return a list of the above types.

\end{methoddesc}

%----------------------------------------------------------------
\subsection{PolyFace class}
\label{polyface}

\begin{classdesc*}{PolyFace}
This class stores the data of a polygonal face. \class{PolyFace} objects
are returned when the value of a \code{polyFaces} attribute is requested.
\end{classdesc*}

The class has the following data members:

\begin{memberdesc}{f}
This is a list of integers containing the edge indices of the edges 
making up the face. If an index is negative the edge has to be
reversed (the edge index then is -i-1).
\end{memberdesc}

\begin{memberdesc}{h}
This is a list of holes. Each hole is described by a list of integers 
containing the edge indices of the edges 
making up the hole in the face. If an index is negative the edge has to be
reversed (the edge index then is -i-1).
\end{memberdesc}

\begin{memberdesc}{mf}
This is a list of texture coordinate ids of the face. This data type
is obsolete as of Maya version 3.0. It is replaced by "mu".
\end{memberdesc}

\begin{memberdesc}{mh}
This is a list of texture coordinate ids of the hole. This data type
is obsolete as of Maya version 3.0. It is replaced by "mu".
\end{memberdesc}

\begin{memberdesc}{mu}
For each loop (i.e. outer loop or hole) this list contains a list of
2-tuples (\var{uvset}, \var{ids}) where \var{uvset} is the index of
the UV set and \var{ids} the indices of the texture coordinates.
\end{memberdesc}

\begin{memberdesc}{fc}
For each loop (outer loop or hole) this list contains a list of color 
index values.
\end{memberdesc}

%----------------------------------------------------------------
\subsection{NurbsSurface class}
\label{nurbssurface}

\begin{classdesc*}{NurbsSurface}
This class stores the data of a NURBS surface. \class{NurbsSurface} objects
are returned when the value of a \code{nurbsSurface} attribute is requested.
\end{classdesc*}

The class has the following data members:

\begin{memberdesc}{udegree}
Degree in u direction.
\end{memberdesc}

\begin{memberdesc}{vdegree}
Degree in v direction.
\end{memberdesc}

\begin{memberdesc}{uform}
Form attribute for the u direction. The attribute can have one of the 
following values:

\begin{tableii}{c|l}{code}{Value}{Meaning}
\lineii{0}{Open}
\lineii{1}{Closed}
\lineii{2}{Periodic}
\end{tableii}
\end{memberdesc}

\begin{memberdesc}{vform}
Form attribute for the v direction (see above).
\end{memberdesc}

\begin{memberdesc}{isrational}
This attribute is \code{True} if the surface contains a rational component.
In this case, the control vertices are given as 4-tuples, otherwise
as 3-tuples. 
\end{memberdesc}

\begin{memberdesc}{uknots}
This is a list of floats containing the knot values for the u direction.
\end{memberdesc}

\begin{memberdesc}{vknots}
This is a list of floats containing the knot values for the v direction.
\end{memberdesc}

\begin{memberdesc}{cvs}
A list of control vertices. Each vertex is given either as a 3-tuple or
4-tuple of floats.
\end{memberdesc}

%----------------------------------------------------------------
\subsection{NurbsCurve class}
\label{nurbscurve}

\begin{classdesc*}{NurbsCurve}
This class stores the data of a NURBS curve. \class{NurbsCurve} objects
are returned when the value of a \code{nurbsCurve} attribute is requested.
\end{classdesc*}

The class has the following data members:

\begin{memberdesc}{degree}
The degree of the curve.
\end{memberdesc}

\begin{memberdesc}{spans}
The number of spans.
\end{memberdesc}

\begin{memberdesc}{form}
Form attribute. The attribute can have one of the following values:

\begin{tableii}{c|l}{code}{Value}{Meaning}
\lineii{0}{Open}
\lineii{1}{Closed}
\lineii{2}{Periodic}
\end{tableii}
\end{memberdesc}

\begin{memberdesc}{isrational}
This attribute is \code{True} if the curve contains a rational component.
In this case, the control vertices have one additional value.
\end{memberdesc}

\begin{memberdesc}{dimension}
The dimension of the curve (2 or 3).
\end{memberdesc}

\begin{memberdesc}{knots}
This is a list of floats containing the knot values.
\end{memberdesc}

\begin{memberdesc}{cvs}
A list of control vertices. Each vertex is a tuple of 2, 3 or 4 floats
(the actual number depends on the \var{dimension} and \var{isrational}
settings).
\end{memberdesc}
