% Component

\section{\class{Component} ---
         Component base class}

A component is just something that has a name to identify it and that
stores slot objects. Usually, the slots are used for providing inputs
to the component which are used to generate outputs that can also be
accessed via slots. The component is then the 'black box' that computes
the output from the input.

\begin{classdesc}{Component}{name = "", \\
		             auto_insert = True}

\var{name} is the name of the component which can be used to identify it.

The component will be inserted into the scene automatically if 
\var{auto_insert} is set to \code{True}.
\end{classdesc}

\begin{memberdesc}{name}
The component's name. You can read and write this attribute.
\end{memberdesc}


% Methods
\begin{methoddesc}{addSlot}{name, slot}
Add a new slot to the component. \var{name} specifies the slot name and
\var{slot} is the slot object. A \exception{KeyError} exception is
thrown if there is already a slot with the specified name.
\end{methoddesc}

\begin{methoddesc}{hasSlot}{name}
Check if a slot with a particular name does exist.
\end{methoddesc}

\begin{methoddesc}{iterSlots}{}
Return an iterator that iterates over all slot names.
\end{methoddesc}

\begin{methoddesc}{numSlots}{}
Return the number of slots in this component.
\end{methoddesc}

\begin{methoddesc}{removeSlot}{name}
Remove the slot with the given name. A \exception{KeyError} exception
is thrown if there is no slot with the specified name.
\end{methoddesc}

\begin{methoddesc}{slot}{name}
Return the slot with the given name. A \exception{KeyError} exception
is thrown if there is no slot with the specified name.
\end{methoddesc}







