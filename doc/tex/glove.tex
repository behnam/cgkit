\section{\module{glove} ---
          Wrapper around the 5DT Data Glove SDK}

\declaremodule{extension}{cgkit.glove}
\modulesynopsis{Wrapper around the 5DT Data Glove SDK}

This module contains a wrapper around the Data Glove SDK by
\ulink{Fifth Dimension Technologies}{http://www.5dt.com/}. The underlying
SDK can either be version 1 or version 2. Note that some methods are
only implemented if the module was compiled with version 2 of the SDK.
All the functions from the SDK are wrapped as methods of the 
\class{DataGlove} class. Most methods require that a connection to
a glove device was previously established via the \method{open()} method.

% available()
\begin{funcdesc}{available}{}
Returns \code{True} if the Data Glove functionality is available. 

If this function returns \code{False}, an exception will be raised
whenever you try to instantiate a class from this module.
\end{funcdesc}

%------------------------------------------------------------------

\begin{classdesc}{DataGlove}{}
This class encapsulates the data glove handle and contains the
functions from the SDK as methods. Before you can use most of
the methods, you have to establish a connection to a data glove
by calling the \method{open()} method.
\end{classdesc}

% isConnected
\begin{methoddesc}{isConnected}{}
Returns \code{True} if a connection to a data glove was established (i.e.
if \method{open()} was successfully called).
\end{methoddesc}

% open
\begin{methoddesc}{open}{port}
Establish a connection to the data glove at the specified port. \var{port}
is a string containing the serial port to use. For example, under Windows
you might specify \code{"COM1"}, \code{"COM2"}, etc.
\end{methoddesc}

% close
\begin{methoddesc}{close}{}
Disconnect from the glove. The return value is a boolean indicating
if there was an error. Calling \method{close()} while there is no
connection established is not considered an error.
\end{methoddesc}

% getGloveHand
\begin{methoddesc}{getGloveHand}{}
Return the handedness of the glove. The return value is either
\code{FD_HAND_LEFT} or \code{FD_HAND_RIGHT}. If you want a string
representation of the handedness you can use the return value as
key for the dictionary \code{HANDEDNESS}.
\end{methoddesc}

% getGloveType
\begin{methoddesc}{getGloveType}{}
Return the type of the data glove. The return value is one of the following:

\begin{tableii}{l|l}{code}{Type}{Description}
\lineii{FD_GLOVENONE}{No data glove}
\lineii{FD_GLOVE5U}{Data Glove 5 Ultra serial}
\lineii{FD_GLOVE5UW}{Data Glove 5 Ultra serial, wireless}
\lineii{FD_GLOVE5U_USB}{Data Glove 5 Ultra USB}
\lineii{FD_GLOVE7}{Data Glove 5}
\lineii{FD_GLOVE7W}{Data Glove 5, wireless}
\lineii{FD_GLOVE16}{Data Glove 16}
\lineii{FD_GLOVE16W}{Data Glove 16, wireless}
\lineii{FD_GLOVE14U}{Data Glove 14 Ultra serial}
\lineii{FD_GLOVE14UW}{Data Glove 14 Ultra serial, wireless}
\lineii{FD_GLOVE14U_USB}{Data Glove 14 Ultra USB}
\end{tableii}

If you want the string representation of the glove type you can use the
return value as key for the dictionary \code{GLOVETYPE}.

Todo: The above are only the return values of the v2 SDK.
\end{methoddesc}

% getNumSensors
\begin{methoddesc}{getNumSensors}{}
Return the number of sensors.
\end{methoddesc}

% getNumGestures
\begin{methoddesc}{getNumGestures}{}
Return the number of available gestures.
\end{methoddesc}

% getSensorRawAll
\begin{methoddesc}{getSensorRawAll}{}
Return a list with the raw sensor values of all sensors.
\end{methoddesc}

% getSensorRaw
\begin{methoddesc}{getSensorRaw}{sensor}
Return the raw sensor value of the specified sensor. \var{sensor} is an
integer in the range between 0 and \method{getNumSensors()}-1.
\end{methoddesc}

% setSensorRawAll
\begin{methoddesc}{setSensorRawAll}{data}
\end{methoddesc}

\begin{methoddesc}{setSensorRaw}{sensor, raw}
\end{methoddesc}

% getSensorScaledAll
\begin{methoddesc}{getSensorScaledAll}{}
Return a list of the scaled sensor values of all sensors.
\end{methoddesc}

% getSensorScaled
\begin{methoddesc}{getSensorScaled}{sensor}
Return the scaled sensor value of the specified sensor. \var{sensor} is an
integer in the range between 0 and \method{getNumSensors()}-1.
\end{methoddesc}

% getCalibrationAll
\begin{methoddesc}{getCalibrationAll}{}
Return the current auto-calibration settings for all sensors. The return
value is a 2-tuple (\var{lower}, \var{upper}) where \var{lower} is a list
with the minimum values and \var{upper} a list with maximum values.
\end{methoddesc}

% getCalibration
\begin{methoddesc}{getCalibration}{sensor}
Return the auto-calibration for one particular sensor. The return value
is a 2-tuple (\var{lower}, \var{upper}) that contains the minimum and
maximum value.
\end{methoddesc}

% setCalibrationAll
\begin{methoddesc}{setCalibrationAll}{lower, upper}
Set the auto-calibration settings. \var{lower} is a list with the minimum
sensor values and \var{upper} is a list with the maximum values.
\end{methoddesc}

% setCalibration
\begin{methoddesc}{setCalibration}{sensor, lower, upper}
Set the auto-calibration values for one sensor. \var{lower} is the minimum
sensor value and \var{upper} is the maximum value.
\end{methoddesc}

% getSensorMaxAll
\begin{methoddesc}{getSensorMaxAll}{}
Return a list with the maximum scaled values of the sensors.
\end{methoddesc}

% getSensorMax
\begin{methoddesc}{getSensorMax}{sensor}
Return a maximum scaled sensor value for one particular sensor.
\end{methoddesc}

% setSensorMaxAll
\begin{methoddesc}{setSensorMaxAll}{max}
Set the maximum scaled sensor values. \var{max} is a list with the values.
\end{methoddesc}

% setSensorMax
\begin{methoddesc}{setSensorMax}{sensor, max}
Set the maximum scaled sensor value for one sensor.
\end{methoddesc}

% resetCalibration
\begin{methoddesc}{resetCalibration}{[sensor]}
Reset the auto-calibration values. In version 2 of the SDK you can pass
an optional sensor index which will only reset that particular sensor.
\end{methoddesc}

% getGesture
\begin{methoddesc}{getGesture}{}
Return the current gesture.
\end{methoddesc}

% getThresholdAll
\begin{methoddesc}{getThresholdAll}{}
Return the current gesture recognition threshold values. The return
value is a 2-tuple (\var{lower}, \var{upper}) that contains the lower
and upper threshold values.
\end{methoddesc}

% getThreshold
\begin{methoddesc}{getThreshold}{sensor}
Return the gesture recognition threshold values for one sensor. The return
value is a 2-tuple (\var{lower}, \var{upper}) that contains the lower
and upper threshold value.
\end{methoddesc}

% setThresholdAll
\begin{methoddesc}{setThresholdAll}{lower, upper}
Set the gesture recognition threshold values for all sensors. \var{lower}
is a list with the lower threshold values and \var{upper} is a list with
the upper threshold values.
\end{methoddesc}

% setThreshold
\begin{methoddesc}{setThreshold}{sensor, lower, upper}
Set the gesture recognition threshold values for one sensor. \var{lower}
is the lower threshold value and \var{upper} is the upper threshold value.
\end{methoddesc}

% getGloveInfo
\begin{methoddesc}{getGloveInfo}{}
Return a string with glove information.
\end{methoddesc}

% getDriverInfo
\begin{methoddesc}{getDriverInfo}{}
Return a string with driver information.
\end{methoddesc}

% scanUSB
\begin{methoddesc}{scanUSB}{}
Scans the USB for available gloves. The return value is a list with product
IDs of the gloves found. The product IDs can be one of \code{DG14U_R}, 
\code{DG14U_L}, \code{DG5U_R} and \code{DG5U_R}.\\
Availability: {\bf V2}
\end{methoddesc}

% setCallback
\begin{methoddesc}{setCallback}{callback}
Set a callback function that gets called when a new packet was received.
\var{callback} must be a callable object.\\
Availability: {\bf V2}
\end{methoddesc}

% getPacketRate
\begin{methoddesc}{getPacketRate}{}
Return the current packet rate in Hertz.\\
Availability: {\bf V2}
\end{methoddesc}

% newData
\begin{methoddesc}{newData}{}
Returns a boolean that indicates whether new data is available or not.\\
Availability: {\bf V2}
\end{methoddesc}

% getFWVersionMajor
\begin{methoddesc}{getFWVersionMajor}{}
Return the major version of the glove's firmware. This is only implemented
for the Data Glove 14 Ultra. Other gloves will always return 0.\\
Availability: {\bf V2}
\end{methoddesc}

% getFWVersionMinor
\begin{methoddesc}{getFWVersionMinor}{}
Return the minor version of the glove's firmware. This is only implemented
for the Data Glove 14 Ultra. Other gloves will always return 0.\\
Availability: {\bf V2}
\end{methoddesc}

% getAutoCalibrate
\begin{methoddesc}{getAutoCalibrate}{}
Returns \code{True} if auto calibration is activated.\\
Availability: {\bf V2}
\end{methoddesc}

% setAutoCalibrate
\begin{methoddesc}{setAutoCalibrate}{flag}
Enable or disable auto-calibration. If \var{flag} is \code{True} 
auto-calibration is enabled.\\
Availability: {\bf V2}
\end{methoddesc}

% saveCalibration
\begin{methoddesc}{saveCalibration}{filename}
Save the current calibration data to disk.\\
Availability: {\bf V2}
\end{methoddesc}

% loadCalibration
\begin{methoddesc}{loadCalibration}{filename}
Load the calibration settings from a file.\\
Availability: {\bf V2}
\end{methoddesc}



\begin{notice}[note]
The module uses the SDK by Fifth Dimension Technologies (5DT) which
can be found at \url{http://www.5dt.com/}. The following is
the copyright information of the SDK:

{\em Copyright (C) 2000-2004, 5DT <Fifth Dimension Technologies>}
\end{notice}
