% Joystick

\section{\class{Joystick} --- Represents a joystick\label{joystick}}

The \class{Joystick} class represents a joystick device. It can be used
to retrieve the current values of the joystick and to calibrate the
joystick axes. The minimum and maximum positions of the axes are
automatically determined while the joystick is in use. The middle
position has to be set manually by calling the \method{setAxisMiddle()}
method.

\begin{classdesc*}{Joystick}
The scene will contain one \class{Joystick} object for every joystick
that is found. 
\end{classdesc*}

\begin{memberdesc}{id}
The device id of the joystick.
\end{memberdesc}

\begin{memberdesc}{name}
The name of the joystick device.
\end{memberdesc}

\begin{memberdesc}{numaxes}
The number of axes this joystick has.
\end{memberdesc}

\begin{memberdesc}{numhats}
The number of hats this joystick has.
\end{memberdesc}

\begin{memberdesc}{numballs}
The number of balls this joystick has.
\end{memberdesc}

\begin{memberdesc}{numbuttons}
The number of buttons this joystick has.
\end{memberdesc}


\begin{methoddesc}{getAxis}{axis}
Returns a calibrated axis value as a float. \var{axis} is the number
of the axis (starting with 0).
\end{methoddesc}

\begin{methoddesc}{getHat}{hat}
Returns a hat value. \var{hat} is the number of the hat (starting with 0).
The return value is a 2-tuple of integers (x, y).
\end{methoddesc}

\begin{methoddesc}{getBall}{ball}
Returns a ball value as a float. \var{ball} is the number of the ball
(starting with 0).
\end{methoddesc}

\begin{methoddesc}{getButton}{button}
Return the state of button \var{button} (0-based number) as a boolean.
\end{methoddesc}

\begin{methoddesc}{setAxis}{axis, value}

\end{methoddesc}

\begin{methoddesc}{setHat}{hat, x, y}
\end{methoddesc}

\begin{methoddesc}{setBall}{ball, value}
\end{methoddesc}

\begin{methoddesc}{setButton}{button, value}
\end{methoddesc}

\begin{methoddesc}{setAxisMiddle}{axis=None, mid=None}
Sets the joystick axis position that will be mapped to the value 0.0.  This
means, if the joystick outputs the (uncalibrated) value \var{mid}, then
the calibrated value will be 0.0.
\var{axis} is the number of the axis you wish to set. If it is \code{None},
all axes are set at once. \var{mid} is the middle value. If it is \code{None},
the current value is used.
\end{methoddesc}


