\section{\module{mayabinary} ---
        Reading Maya Binary files}

\declaremodule{extension}{cgkit.mayabinary}
\modulesynopsis{Reading Maya Binary files}

This module contains the \class{MBReader} class which can be used as a
base class for reading Maya Binary (*.mb) files. The class parses the
structure of the file and invokes a callback method for every chunk found
in the file. The actual decoding of the chunk data has to be done in a
derived class.
%----------------------------------------------------------------
\subsection{MBReader class}

The \class{MBReader} class reads Maya Binary files and calls
appropriate methods which have to be implemented in a derived class.
A Maya Binary file is composed of chunks that contain the actual data.
There can be data chunks that contain the actual data and group chunks
that contain the data chunks.

\begin{classdesc}{MBReader}{}
  Creates an instance of the reader.
\end{classdesc}

\begin{memberdesc}{filename}
The file name (if it could be obtained). This may be used for generating
warning or error messages.
\end{memberdesc}

\begin{methoddesc}{read}{file}
Read the content of a file. \var{file} is either a file like object that
can be used to read the content of the file or the name of a file.
\end{methoddesc}

\begin{methoddesc}{abort}{}
Aborts reading the current file.
This method can be called in handler methods to abort reading the file.
\end{methoddesc}

\begin{methoddesc}{onBeginGroup}{chunk}
Callback that is called whenever a new group tag begins.
\var{chunk} is a GroupChunk object (see section \ref{groupchunkclass})
containing information about the group chunk.
\end{methoddesc}

\begin{methoddesc}{onEndGroup}{chunk}
Callback that is called whenever a group goes out of scope.
\var{chunk} is a GroupChunk object (see section \ref{groupchunkclass})
containing information about the group chunk (it is the same instance that was passed to \method{onBeginGroup()}).
\end{methoddesc}

\begin{methoddesc}{onDataChunk}{chunk}
Callback that is called for each data chunk.
\var{chunk} is a Chunk object (see section \ref{chunkclass}) that contains
information about the chunk and that can be used to read the actual chunk data.
\end{methoddesc}

%----------------------------------------------------------------
\subsection{Chunk class}
\label{chunkclass}

A \class{Chunk} object is passed to the callback methods of the
\class{MBReader} class. It contains information about the current
chunk and it can be used to read the actual chunk data.
A \class{Chunk} object has the following attributes and methods:

\begin{memberdesc}{tag}
This is a string containing four characters that represent the chunk
name.
\end{memberdesc}

\begin{memberdesc}{size}
The size in bytes of the data part of the chunk.
\end{memberdesc}

\begin{memberdesc}{pos}
The absolute position of the data part within the input file.
\end{memberdesc}

\begin{memberdesc}{depth}
The depth of the chunk (i.e. how deep it is nested). The root
has a depth of 0.
\end{memberdesc}

\begin{memberdesc}{parent}
The GroupChunk object of the parent chunk. In the case of the root group
chunk this attribute is \code{None}.
\end{memberdesc}


\begin{methoddesc}{chunkPath}{}
Return a string containing the full path to this chunk.
The result is a concatenation of all chunk names that lead to this chunk.
\end{methoddesc}

\begin{methoddesc}{read}{bytes=-1}
Read the specified number of bytes from the chunk.
If bytes is -1 the entire chunk data is read. The return value is a
string containing the binary data.
\end{methoddesc}

%----------------------------------------------------------------
\subsection{GroupChunk class}
\label{groupchunkclass}

The \class{GroupChunk} class is derived from the \class{Chunk} class (see
section \ref{chunkclass}), so it has the same attributes and methods. In addition
it defines one more attribute:

\begin{memberdesc}{type}
This is a string containing four characters that represent the group type.
\end{memberdesc}

