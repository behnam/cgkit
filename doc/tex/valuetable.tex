% ValueTable component

\section{\class{ValueTable} ---
         Animate a value from a table of values}

A \class{ValueTable} stores time/value pairs and has an output slot that holds
the appropriate value for the current time. The type of the value can
be specified in the constructor. The name of the output slot is always
\var{output_slot}.

\begin{classdesc}{ValueTable}{name = "ValueTable",\\ 
	                      type = "vec3", \\
	                      values = [], \\
	                      modulo = None, \\
	                      tscale = 1.0, \\
                              auto_insert = True}

\var{type} is the type of the values stored in the component. This is also
the type of the output slot.

\var{values} is a list of initial time/value pairs. Each item must
be a 2-tuple (time, value).

\var{modulo} is the duration of the loop. The animation sequence will be
replayed after \var{modulo} seconds. The sequence will not repeat
if \var{modulo} is \code{None}.

\var{tscale} is a scaling factor for the time. Values smaller than 1.0 will
slow down the animation.

\end{classdesc}

\begin{methoddesc}{add}{t, v}
Add a new time/value pair to the table. The value \var{v} must be of
the appropriate type.
\end{methoddesc}


The values can either be added using the \method{add()} method or using
the index operator. If you want to retrieve the value for a particular
time (without using the value slot) then you can either use the call operator
or the index operator:

\begin{verbatim}
>>> vt=ValueTable(type="double")
>>> vt.add(0.0, 1.0)
>>> vt[0.5] = 2.0
>>> list(vt)
[(0.0, 1.0), (0.5, 2.0)]
>>> vt(0.2)
1.0
>>> vt[1.5]
2.0
\end{verbatim}

You can iterate over a \class{ValueTable} to get the time/value pairs:

\begin{verbatim}
>>> vt=ValueTable(type="double")
>>> vt.add(0, 1)
>>> vt.add(1.5, -2)
>>> vt.add(1.0, 0.5)
>>> for t,v in vt: print t,v
...
0 1
1.0 0.5
1.5 -2
\end{verbatim}