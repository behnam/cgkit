% Joint

\section{\class{Joint} ---
         Joint class for creating a skeleton}

The \class{Joint} class can be used to create articulated
characters. Bones are implicitely created by linking two joints, i.e.
a bone is assumed between a joint and each of its children joints.
A \class{Joint} object corresponds to a ball joint that has three rotational
degrees of freedom. The actual joint is always located at the pivot point
of the \class{Joint} object and rotates about the pivot frame axes.

If you want your local coordinate frame to be oriented differently, it
is not enough to set the offset transform as this will also readjust the
local frame so that the effect is actually cancelled out and you will
still rotate about the same axes as before. To initialize the new frame
you have to call \method{freezePivot()} after setting the offset transform.
This will set the current offset transform as the new default pose where
all angles are 0 and will make the joint rotate about the new axes.
After freezing you may want to set the offset transform back to the identity
so that the local coordinate frame and the offset frame coincide again.
This won't affect the rotation of the joint but the location of its
children (as setting the offset transform on a \var{Joint} object also
modifies its local coordinate system). But it actually depends on the
situation whether you have to reset the offset transform or not.

Rotating a joint is not done by setting its \var{rot} attribute but
by setting its three individual angles (\var{anglex}, \var{angley}, 
\var{anglez}).

\begin{classdesc}{Joint}{name = "", \\
			 radius = 0.05, \\
	                 rotationorder = "xyz"
		         }

\var{name} is the name of the joint.

\var{radius} is the radius of the visual representation of the joint/bone.

\var{rotationorder} determines the order of rotation about the individual
axes.
\end{classdesc}

A \class{Joint} has the following slots:

\begin{tableiv}{l|l|c|l}{code}{Slot}{Type}{Access}{Description}
\lineiv{anglex_slot}{float}{rw}{Angle around x axis}
\lineiv{angley_slot}{float}{rw}{Angle around y axis}
\lineiv{anglez_slot}{float}{rw}{Angle around z axis}
\end{tableiv}

\begin{memberdesc}{anglex}
Rotation angle (in degrees) about the local x axis.
\end{memberdesc}

\begin{memberdesc}{angley}
Rotation angle (in degrees) about the local y axis.
\end{memberdesc}

\begin{memberdesc}{anglez}
Rotation angle (in degrees) about the local z axis.
\end{memberdesc}

% Methods
\begin{methoddesc}{freezePivot}{}
Make the current pivot coordinate system the default pose. After
calling this method, the current rotation of the pivot coordinate
system will define the default pose. This means, rotations are now
defined around the local pivot axes. 
\end{methoddesc}






