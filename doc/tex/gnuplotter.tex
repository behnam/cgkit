% GnuPlotter component

\section{\class{GnuPlotter} ---
         Plot values using Gnuplot}

The \class{GnuPlotter} class can be used to plot the graph of a
floating point slot. To do so, connect any \class{DoubleSlot} to
an input slot of the plotter. The input slots are called \code{input<n>_slot}
where \code{<n>} is the number of the slot.

\begin{classdesc}{GnuPlotter}{name = "GnuPlotter",\\ 
                              title = None, \\
                              xlabel = None, \\
                              ylabel = None, \\
                              xrange = None, \\
                              yrange = None, \\
                              inputs = 1, \\
                              plottitles = [], \\
                              starttime = 0.0, \\
                              endtime = 99999.0, \\
                              enabled = True, \\
                              auto_insert = True}

\var{title} is a string containing the title of the entire plot.

\var{xlabel}, \var{ylabel} are strings containing the labels for the X and
Y axis.

\var{xrange}, \var{yrange} is each a 2-tuple (\var{start}, \var{end}) 
containing the range of the X axis resp. Y axis.

\var{inputs} is the number of input slots that should be created (i.e.
the number of curves you want to plot).

\var{plottitles} is a list of strings containing the name of the respective
curve.

\var{starttime} and \var{endtime} defines the range in which values are
received and plotted. The times are given in seconds.

\var{enabled} is a flag that can be used to disable the plotter.
\end{classdesc}

A \class{GnuPlotter} object has the following slots:

\begin{tableiv}{l|l|c|l}{code}{Slot}{Type}{Access}{Description}
\lineiv{input1_slot}{float}{rw}{First curve}
\lineiv{input2_slot}{float}{rw}{Second curve}
\lineiv{...}{...}{...}{...}
\end{tableiv}

% --------------------
\begin{notice}[note]
To use the \class{GnuPlotter} component the
\ulink{Gnuplot.py}{http://gnuplot-py.sourceforge.net/} module has to be 
installed on your system (and of course, 
\ulink{gnuplot}{http://www.gnuplot.info/} itself as well).
\end{notice}

