\subsection{vec4 - 4d vector}
\label{vec4}

A \class{vec4} represents a 4D vector type. You can construct a \class{vec4}
by several ways:

\begin{verbatim}
# all components are set to zero
v = vec4()

-> (0.0000, 0.0000, 0.0000, 0.0000)

# set all components to one value
v = vec4(2.5)

-> (2.5000, 2.5000, 2.5000, 2.5000)

# set a 2d vector, the ramaining components will be zero
v = vec4(1.5, 0.8)

-> (1.5000, 0.8000, 0.0000, 0.0000)

# set a 3d vector, the ramaining component will be zero
v = vec4(1.5, 0.8, -0.5)

-> (1.5000, 0.8000, -0.5000, 0.0000)

# set all components
v = vec4(1.5, 0.8, -0.5, 0.2)

-> (1.5000, 0.8000, -0.5000, 0.2000)
\end{verbatim}

Additionally you can use all of the above, but store the values inside
a tuple, a list or a string:

\begin{verbatim}
v = vec4([1.5, 0.8, -0.5])
w = vec4("1.5, 0.8")
\end{verbatim}

Finally, you can initialize a vector with a copy of another vector:

\begin{verbatim}
v = vec4(w)
\end{verbatim}

A \class{vec4} can be used just like a list with 4 elements, so you
can read and write components using the index operator or by accessing
the components by name:

\begin{verbatim}
>>> v=vec4(1,2,3,1)
>>> print v[0]
1.0
>>> print v.y
2.0
>>> print v.w
1.0
>>> print v.t   # this is the same as v.w
1.0
\end{verbatim}

The 4th component can be accessed either by the name \code{w} or
\code{t}. You might prefer the former name when using the vector as a
homogenous coordinate while the latter might be preferable when the
4th component shall represent a time value.

%----------------------------------------
{\bf Mathematical operations}

The mathematical operators are supported with the following
combination of types:

\begin{verbatim}
vec4  =  vec4 + vec4
vec4  =  vec4 - vec4
float =  vec4 * vec4      # dot product
vec4  = float * vec4
vec4  =  vec4 * float
vec4  =  vec4 / float
vec4  =  vec4 % float     # each component
vec4  =  vec4 % vec4      # component wise
vec4  = -vec4
float =  vec4[i]          # get or set element
\end{verbatim}

Additionally, you can compare vectors with \code{==}, \code{!=}, \code{<}, 
\code{<=}, \code{>}, \code{>=}. Each
comparison (except \code{<} and \code{>}) takes an epsilon environment
into account, this means two values are considered to be equal if
their absolute difference is less than or equal to a threshold value
epsilon. You can read and write this threshold value using the
functions \function{getEpsilon()} and \function{setEpsilon()}.

Taking the absolute value of a vector will return the length of the vector: 

\begin{verbatim}
float = abs(v)            # this is equivalent to v.length()
\end{verbatim}

%----------------------------------------
{\bf Methods}

\begin{methoddesc}{length}{}
Returns the length of the vector ($\sqrt{x^2+y^2+z^2}$). This is
equivalent to calling \code{abs(self)}.
\end{methoddesc}

\begin{methoddesc}{normalize}{}
Returns normalized vector. If the method is called on the null vector
(where each component is zero) a \exception{ZeroDivisionError} is raised.
\end{methoddesc}

\begin{methoddesc}{min}{}
Returns the minimum value of the components.
\end{methoddesc}

\begin{methoddesc}{max}{}
Returns the maximum value of the components.
\end{methoddesc}

\begin{methoddesc}{minIndex}{}
Return the index of the component with the minimum value.
\end{methoddesc}

\begin{methoddesc}{maxIndex}{}
Return the index of the component with the maximum value.
\end{methoddesc}

\begin{methoddesc}{minAbs}{}
Returns the minimum absolute value of the components.
\end{methoddesc}

\begin{methoddesc}{maxAbs}{}
Returns the maximum absolute value of the components.
\end{methoddesc}

\begin{methoddesc}{minAbsIndex}{}
Return the index of the component with the minimum absolute value.
\end{methoddesc}

\begin{methoddesc}{maxAbsIndex}{}
Return the index of the component with the maximum absolute value.
\end{methoddesc}

