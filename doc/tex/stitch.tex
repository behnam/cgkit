\section{\module{stitch} ---
         Stitching together image tiles}

\declaremodule{extension}{cgkit.stitch}
\modulesynopsis{Stitching together image tiles}

This module can be used to stitch together images that have been
rendered in tiles (using RenderMan's RiCropWindow() functionality, 
for example). If you are using the render tool to create an image
you can use the global value \var{tile} to render the image in tiles
(see section \ref{ribexport}).
The module can also be used as a stand-alone command line utility.

\begin{funcdesc}{stitch}{filename, removetiles=False, infostream=None}
Stitches several image tiles together.   
\var{filename} is the base name of the image that determines the file names
of the tiles. \var{filename} is also the name of the output image.  If
\var{removetiles} is \code{True}, the individual image files will be deleted 
after the image has been stitched. If \var{infostream} is set to a file
like object it is used to output status information about the
stitching process.

The name of an image tile must contain the crop information that was
used to create the image. For example, the name of a tile for an image
"out.tif" could look like this: "out_0.0_0.5_0.75_1.0.tif".  The four
values are the x1,x2,y1,y2 values of the crop window. Those values
together with the resolution of the tile determine the resolution of
the entire image. The position of the tile within that image is given
by x1,y1.
\end{funcdesc}

