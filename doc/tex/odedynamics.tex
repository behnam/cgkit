% ODEDynamics component

\section{\class{ODEDynamics} ---
         Rigid body dynamics using the Open Dynamics Engine}

\begin{classdesc}{ODEDynamics}{name = "ODEDynamics",\\ 
			       gravity = 9.81, \\
                               substeps = 1, \\
                               enabled = True, \\
                               erp = None, \\
                               cfm = None, \\
                               defaultcontactproperties = None, \\
                               collision_events = False, \\
                               auto_add = False, \\
                               show_contacts = False, \\
                               contactmarkersize = 0.1, \\
                               contactnormalsize = 1.0, \\
                               auto_insert = True}

\var{gravity} is the acceleration due to gravity. The direction of the
acceleration is in negative "up" direction (as specified by the scene).

\var{substeps} is the number of simulation steps per frame. You can
increase this value to get a more accurate/stable simulation.

The simulation will only run if \var{enabled} is \code{True}, otherwise
it's halted.

\var{erp} and \var{cfm} are the global error reduction parameter and
constraint force mixing value to be useed (see the ODE manual).

\var{defaultcontactproperties} is a \class{ODEContactProperties} object
that specifies the default contact parameters. These parameters are
used for contacts between two objects (resp. materials) that have not
been set explicitly using \method{setContactProperties()}.

\var{collision_events} determines whether the component will generate
\code{ODE_COLLISION} events whenever a collision has occured. An event
handler takes a \class{ODECollisionEvent} (see section 
\ref{odecollisionevent}) object as argument.

If \var{auto_add} is \code{True} the component searches the scene for
rigid bodies and hinges and adds them automatically to the component.
This is done at the time the component is created, so any bodies or
hinges created afterwards will be ignored.

\var{show_contacts} determines whether the contact points and normals
are visualized or not (this is mainly for debugging purposes).
The size of the contact point markers and the length of the normals
can be specified via the \var{contactmarkersize} and \var{contactnormalsize}
arguments.
\end{classdesc}

\begin{methoddesc}{add}{objects, categorybits=None, collidebits=None}
Add world objects to the simulation. \var{objects} can be a single
object or a sequence of objects. An object may be specified by its
name or the object itself. \var{categorybits} and \var{collidebits}
are long values that control which objects can collide with which
other object. The specified category and collide bits are assigned to every
object in \var{objects}. Each bit in \var{categorybits} represents
one category the objects belong to. \var{collidebits} is another bit
field that specifies with which categories the objects may collide.
By default, every bit is set in both values.
\end{methoddesc}

\begin{methoddesc}{reset}{}
Reset the state of the simulated bodies. All bodies will be set to the
position and velocity they had when they were added to the simulation.
This method is also called when the RESET event is issued.
\end{methoddesc}

\begin{methoddesc}{setContactProperties}{(mat1, mat2), props}
Set the contact properties for a material pair. \var{mat1} and \var{mat2}
are two \class{Material} objects and props is a \class{ODEContactProperties}
object describing the contact properties.
\end{methoddesc}

\begin{methoddesc}{getContactProperties}{(mat1, mat2)}
Return the contact properties for a material pair. The order of the materials
is irrelevant. The return value is
a \class{ODEContactProperties} object. A default property object is
returned if the pair does not have any properties set.
\end{methoddesc}

\begin{methoddesc}{createBodyManipulator}{object}
Return an \class{ODEBodyManipulator} object that can be used to apply
external forces/torques to the world object \var{object}.
\end{methoddesc}


\begin{notice}[note]
To use the \class{ODEDynamics} component the
\ulink{PyODE}{http://pyode.sourceforge.net/} module has to be 
installed on your system which wraps the 
\ulink{Open Dynamics Engine}{http://www.ode.org/}.
\end{notice}

%------------------------------------------------------
\subsection{\class{ODEContactProperties} ---
         Contact properties during collision}

The \class{ODEContactProperties} class contains all the parameters
that are used when two objects collide. 

\begin{classdesc}{ODEContactProperties}{mode = 0,\\
			       mu = 0.3,\\
			       mu2 = None,\\
			       bounce = None,\\
			       bounce_vel = None,\\
			       soft_erp = None,\\
			       soft_cfm = None,\\
			       motion1 = None,\\
			       motion2 = None,\\
			       slip1 = None,\\
			       slip2 = None,\\
			       fdir1 = None}

See the ODE manual (chapter 
\ulink{7.3.7 {\em Contact}}{http://ode.org/ode-latest-userguide.html#sec_7_3_7})
for an explanation of these
parameters.

\note{You only have to specify the \var{mode} argument if you want to set
the ContactApprox* flags. The other flags are automatically set.}
\end{classdesc}

%------------------------------------------------------
\subsection{\class{ODEBodyManipulator} ---
         Apply external forces/torques to bodies}

The \class{ODEBodyManipulator} class can be used to apply external
forces and torques to a rigid body. 

\begin{classdesc*}{ODEBodyManipulator}
You get an instance of this class
by calling the
\method{createBodyManipulator()} method of the \class{ODEDynamics} component.
One particular body manipulator instance is always associated with one
particular rigid body. A manipulator object has the following attributes
and methods:
\end{classdesc*}

\begin{memberdesc}{body}
This attribute contains the rigid body (\class{WorldObject}) this
manipulator is associated with. You can only read this attribute. If
you want to control another body, use the
\method{createBodyManipulator()} method of the dynamics component.
\end{memberdesc}

\begin{memberdesc}{odebody}
This is the Body instance of the PyODE module. You can use this object
if you want to access special features of ODE that are not exposed otherwise.
But note that you won't get the expected results if you call methods like
\method{addForce()} directly on the ODE body and you're using more than
one sub step in your simulation. The force would only be applied during
the first sub step because it is reset after each step. Use this
manipulator class instead, that's what it's for.
\end{memberdesc}

% addForce
\begin{methoddesc}{addForce}{force, relforce=False, pos=None, relpos=False}
Add an external force to the current force vector. \var{force} is a vector
containing the force to apply. If \var{relforce} is \code{True} the force
is interpreted in local object space, otherwise it is assumed to be given
in global world space. By default, the force is applied at the center
of gravity. You can also pass a different position in the \var{pos} argument
which must describe a point in space. \var{relpos} determines if the
point is given in object space or world space (default).
\end{methoddesc}

% addTorque
\begin{methoddesc}{addTorque}{torque, reltorque=False}
Add an external torque to the current torque vector. \var{torque} is
a vector containing the torque to apply. \var{reltorque} determines if
the torque vector is given in object space or world space (default).
\end{methoddesc}

% setInitialPos
\begin{methoddesc}{setInitialPos}{pos}
Set the initial position of the body. \var{pos} must be a 3-sequence of 
floats containing the new position.
\end{methoddesc}

% setInitialRot
\begin{methoddesc}{setInitialRot}{rot}
Set the initial orientation of the body. \var{rot} must be a
\class{mat3} containing a rotation matrix.
\end{methoddesc}

% setInitialLinearVel
\begin{methoddesc}{setLinearVel}{vel}
Set the initial linear velocity of the body. \var{vel} must be a
3-sequence of floats containing the new velocity.
\end{methoddesc}

% setInitialAngularVel
\begin{methoddesc}{setAngularVel}{vel}
Set the initial angular velocity of the body. \var{vel} must be a
3-sequence of floats containing the new velocity.
\end{methoddesc}

% setPos
\begin{methoddesc}{setPos}{pos}
Set the position of the body. \var{pos} must be a 3-sequence of floats
containing the new position.
\end{methoddesc}

% setRot
\begin{methoddesc}{setRot}{rot}
Set the orientation of the body. \var{rot} must be a \class{mat3} containing a
rotation matrix.
\end{methoddesc}

% setLinearVel
\begin{methoddesc}{setLinearVel}{vel}
Set the linear velocity of the body. \var{vel} must be a 3-sequence of floats
containing the new velocity.
\end{methoddesc}

% setAngularVel
\begin{methoddesc}{setAngularVel}{vel}
Set the angular velocity of the body. \var{vel} must be a 3-sequence of floats
containing the new velocity.
\end{methoddesc}

%------------------------------------------------------
\subsection{\class{ODECollisionEvent} ---
         Collision event object}
\label{odecollisionevent}

An \class{ODECollisionEvent} object is passed as argument to the event
handler for \code{ODE_COLLISION} events.

\begin{classdesc}{ODECollisionEvent}{obj1, obj2, contacts, contactproperties}

\var{obj1} and \var{obj2} are the two world objects that have collided with 
each other.

\var{contacts} is a list of \class{ode.Contact} objects that each describes
a contact point.

\var{contactproperties} is a \class{ODEContactProperties} object that
describes the properties of the contact. It depends on the materials
of the \var{obj1} and \var{obj2}. The event handler may modify
this object to change the result of the collision. Note however, that
the changes will be permanent and also affect later collisions.
\end{classdesc}

% averageContactGeom
\begin{methoddesc}{averageContactGeom}{}
Return the average contact position, normal and penetration depth (in
this order). The position and normal are returned as \class{vec3}
objects, the penetration depth is a float.
\end{methoddesc}
