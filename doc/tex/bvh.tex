\section{\module{bvh} ---
        Reading Biovision Hierarchical (BVH) motion capture files}

\declaremodule{extension}{cgkit.bvh}
\modulesynopsis{Reading Biovision Hierarchical (BVH) motion capture files}

This module contains the \class{BVHReader} class which can be used as a base
class for reading Biovision Hierarchical (BVH) files. The class reads the
file and invokes callback methods with the corresponding data in the file.
Derived classes have to implement those callback methods and process
the data as appropriate.

\begin{classdesc}{BVHReader}{filename}
  \var{filename} is the name of the BVH file that should be read.
\end{classdesc}

\begin{methoddesc}{read}{}
Read the entire file.
\end{methoddesc}

\begin{methoddesc}{onHierarchy}{root}
This method is called after the joint hierarchy was read. The entire 
hierarchy is passed in the argument \var{root} which is a \class{Node}
object.
\end{methoddesc}

\begin{methoddesc}{onMotion}{frames, dt}
This method is called when the motion data begins. \var{frames} is the
number of motion samples that follow and \var{dt} is the time interval
that corresponds to one frame.
\end{methoddesc}

\begin{methoddesc}{onFrame}{values}
This method is called for each motion sample (frame) in the
file. \var{values} is a list of floats that contains the position and
angles of the entire skeleton. The order is the same than when
traversing the joint hierarchy in a depth-first manner.
\end{methoddesc}

% ------------------------------------
\subsection{Node objects}

The \method{onHierarchy()} method of the \class{BVHReader} class takes
the joint hierarchy of the skeleton as input. Each node in this hierarchy
is represented by a \class{Node} object that contains all information
stored in the BVH file.

\begin{classdesc*}{Node}
  A \class{Node} object represents one joint in the hierarchy.
\end{classdesc*}

\begin{memberdesc}{name}
This is the name of the joint (or the root).
\end{memberdesc}

\begin{memberdesc}{channels}
This is a list of channel names that are associated with this joint.
This list determines how many values are stored in the motion section
and how they are to be interpreted. Each channel name can be one
of \code{Xposition}, \code{Yposition}, \code{Zposition}, \code{Xrotation},
\code{Yrotation}, \code{Zrotation}.
\end{memberdesc}

\begin{memberdesc}{offset}
This is a 3-tuple of floats containing the offset position of this joint
relative to the parent joint.
\end{memberdesc}

\begin{memberdesc}{children}
This is a list of children joints (which are again described by
\class{Node} objects).
\end{memberdesc}

\begin{methoddesc}{isRoot}{}
Returns \code{True} if the node is the root node.
\end{methoddesc}

\begin{methoddesc}{isEndSite}{}
Returns \code{True} if the node is a leaf.
\end{methoddesc}
