% WorldObject

\section{\class{WorldObject} ---
         World object base class}

\begin{classdesc}{WorldObject}{name = "object", \\
                 transform = None,\\
                 pos = None, \\
	         rot = None,\\
                 scale = None,\\
                 pivot = None,\\
                 offsetTransform = None,\\
                 parent = None,\\
                 mass = None,\\
                 material = None,\\
                 visible = True,\\
                 auto_insert = True}

\var{name} is the name of the object which can be used to identify the
object.

\var{transform} is the initial transformation that should be applied to
the object. Alternatively, you can specify the individual components
\var{pos}, \var{rot} and \var{scale}.

\var{pivot} is the pivot point of the object. This is the 4th column of
the offset transformation. You can also specify the entire offset 
transformation using the \var{offsetTransform} argument.

\var{parent} is the parent world object and determines at which position
in the scene graph the new object is added. If \var{parent} is \code{None}
the object will become a child of the world root. The \var{parent} argument
is only used if \var{auto_insert} is \code{True}.

\var{mass} is the mass of this object (this does {\em not} include
the children objects).

\var{material} describes the appearance of the object. It can be either 
a single \class{Material} object or a sequence of \class{Material} objects.

\var{visible} is a flag that determines whether the object is visible
or not. This only affects the geometry of this WorldObject, it is not
inherited by children objects.

The object will be inserted into the scene automatically if 
\var{auto_insert} is set to \code{True}.
\end{classdesc}

A \class{WorldObject} always has the following slots:

\begin{tableiv}{l|l|c|l}{code}{Slot}{Type}{Access}{Description}
\lineiv{angularvel_slot}{vec3}{rw}{Angular velocity}
\lineiv{cog_slot}{vec3}{r}{Center of gravity}
\lineiv{inertiatensor_slot}{mat3}{r}{Inertia tensor}
\lineiv{linearvel_slot}{vec3}{rw}{Linear velocity}
\lineiv{mass_slot}{float}{rw}{Mass of the local geometry}
\lineiv{pos_slot}{vec3}{rw}{Position}
\lineiv{rot_slot}{mat3}{rw}{Orientation}
\lineiv{scale_slot}{vec3}{rw}{Scaling}
\lineiv{totalmass_slot}{float}{r}{Total mass (including the children)}
\lineiv{transform_slot}{mat4}{rw}{Object transformation}
\lineiv{visible_slot}{bool}{rw}{Visibility flag}
\lineiv{worldtransform_slot}{mat4}{rw}{World transformation}
\end{tableiv}

\begin{memberdesc}{geom}
This attribute holds the visible geometry which must be derived from 
\class{GeomObject}. The value can also be \code{None} if there is no
visible geometry. Geometry objects can be shared between different 
world objects. This value can be read and written.
\end{memberdesc}

\begin{memberdesc}{parent}
This attribute contains the parent world object or \code{None}. You can 
only read this attribute.
\end{memberdesc}

\begin{memberdesc}{transform}
This is the value of the mat4 slot \var{transform_slot} which contains
the object transformation T. You can read and write this attribute.
\end{memberdesc}

\begin{memberdesc}{worldtransform}
This is the value of the mat4 slot \var{worldtransform_slot} which
contains the world transformation (which is a concatenation of all local
transformations L). You can only read this attribute.

\note{Note that in contrast to the \var{transform} slot,
the \var{worldtransform} is not influenced by the offset transformation.}
\end{memberdesc}

\begin{memberdesc}{pos}
This is the value of the vec3 slot \var{pos_slot} which contains the 
position of the object. You can read and write this attribute.
\end{memberdesc}

\begin{memberdesc}{rot}
This is the value of the mat3 slot \var{rot_slot} which contains the 
orientation of the object. You can read and write this attribute.
\end{memberdesc}

\begin{memberdesc}{scale}
This is the value of the vec3 slot \var{scale_slot} which contains the 
scaling of the object. You can read and write this attribute.
\end{memberdesc}

\begin{memberdesc}{pivot}
This is the pivot point (vec3) of the object. You can read and write this
attribute. Reading or writing this attribute is equivalent to calling
\method{getOffsetTransform()} or \method{setOffsetTransform()} with
a matrix that only modifies the 4th column.
\end{memberdesc}

\begin{memberdesc}{cog}
This is the value of the vec3 slot \var{cog_slot} which contains the
physical center of gravity. This value is derived from the center of 
gravity provided by the geometry object and the cogs and masses of the
children objects. This means it represents the center of gravity of the 
entire hierarchy. The value is given with respect to the pivot coordinate
system P. You can only read this value.
\end{memberdesc}

\begin{memberdesc}{inertiatensor}
This is the value of the mat3 slot \var{inertiatensor_slot} which contains
the inertia tensor of the entire hierarchy (just like \var{cog}). You can
only read this value.
\end{memberdesc}

\begin{memberdesc}{mass}
This is the value of the double slot \var{mass_slot} which contains the
{\em local} mass of this object (not including the children). Or in other
words, this is the mass of the geometry directly set in this object.
You can read and write this value.
\end{memberdesc}

\begin{memberdesc}{totalmass}
This is the value of the double slot \var{totalmass_slot} which contains
the total mass of this object and its children. You can only read this value.
\end{memberdesc}

\begin{memberdesc}{angularvel}
This is the value of the vec3 slot \var{angular_slot} which contains the 
angular velocity of the object. The value is not computed but has to be
set by anyone who knows the angular velocity (such as a dynamics component).
You can read and write this attribute.
\end{memberdesc}

\begin{memberdesc}{linearvel}
This is the value of the vec3 slot \var{linearvel_slot} which contains the 
linear velocity of the object. The value is not computed but has to be
set by anyone who knows the linear velocity (such as a dynamics component).
You can read and write this attribute.
\end{memberdesc}


% Methods
\begin{methoddesc}{boundingBox}{}
Return the local axis aligned bounding box. The bounding box is
given with respect to the local transformation L (which is not
what you get from the transform slot of the world object).
\end{methoddesc}

\begin{methoddesc}{localTransform}{}
Returns the local transformation that has to be used for rendering.
The returned transformation L is calculated as follows: $L = T\cdot P^{-1}$,
where T is the current transform (taken from the transform slot)
and P is the offset transform.
\end{methoddesc}

\begin{methoddesc}{getOffsetTransform}{}
Return the current offset transformation as a \class{mat4}. This
transformation is given relative to the local object transformation.
\end{methoddesc}

\begin{methoddesc}{setOffsetTransform}{P}
Set the offset transformation. The transformation has to be given
relative to the local object transformation. After setting the offset
transformation, the transform slot will be updated so that 
\method{localTransform()} returns the same matrix as before, i.e. the
world position/orientation of the object does not change.
\end{methoddesc}

\begin{methoddesc}{getNumMaterials}{}
Return the current size of the material array.
\end{methoddesc}

\begin{methoddesc}{setNumMaterials}{num}
Set a new size for the material array.
\end{methoddesc}

\begin{methoddesc}{getMaterial}{idx=0}
Get a stored material. The method returns \code{None} if the given index
is out of range or there is no material stored at that position.
\end{methoddesc}

\begin{methoddesc}{setMaterial}{mat, idx=0}
Set a new material. An \exception{IndexError} exception is thrown if
the index is out of range.
\end{methoddesc}

\begin{methoddesc}{lenChilds}{}
Return the number of children objects.
\end{methoddesc}

\begin{methoddesc}{iterChilds}{}
Return an iterator that iterates over all children objects.
\end{methoddesc}

\begin{methoddesc}{hasChild}{name}
Check if a children with a particular name does exist.
\end{methoddesc}

\begin{methoddesc}{child}{name}
Return the children with a particluar name. A \exception{KeyError}
exception is thrown if there is no children with the specified name.
\end{methoddesc}

\begin{methoddesc}{addChild}{child}
Add a new children world object to this object. A
\exception{ValueError} exception is thrown if child was already added
to another object.  In this case you have to remove the object from
its previous parent yourself. You also have to make sure that the name
of \var{child} is unique among the children of this object, otherwise
a \exception{KeyError} exception is thrown.
\end{methoddesc}

\begin{methoddesc}{removeChild}{child}
Remove a children world object from this object. \var{child} can either
be the name of the children or the object itself. A \exception{KeyError}
exception is thrown if child is not a children of this object.
\end{methoddesc}

\begin{methoddesc}{makeChildNameUnique}{name}
Modify \var{name} so that it is unique among the children names. If
\var{name} is already the name of a children object, then it is modified
by adding/increasing a trailing number, otherwise it is returned
unchanged.
\end{methoddesc}




