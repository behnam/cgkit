% TriMeshGeom

\section{\class{TriMeshGeom} ---
         Triangle mesh geometry}

\begin{classdesc}{TriMeshGeom}{}
Creates an empty triangle mesh.
\end{classdesc}

A \class{TriMeshGeom} has the following slots:

\begin{tableiv}{l|l|c|l}{code}{Slot}{Type}{Access}{Description}
\lineiv{cog_slot}{vec3}{r}{The local center of gravity}
\lineiv{inertiatensor_slot}{mat3}{r}{The local inertia tensor}
\lineiv{verts_slot}{vec3 array}{rw}{The mesh vertices}
\lineiv{faces_slot}{int[3] array}{rw}{The mesh faces}
\end{tableiv}

% Attributes
\begin{memberdesc}{cog}
Center of gravity with respect to the local coordinate system of the 
triangle mesh.
\end{memberdesc}

\begin{memberdesc}{inertiatensor}
Inertia tensor with respect to the local coordinate system of the 
triangle mesh.
\end{memberdesc}

\begin{memberdesc}{verts}
This attribute contains the sequence of mesh vertices.
\end{memberdesc}

\begin{memberdesc}{faces}
This attribute contains the sequence of mesh faces. Each face contains
three vertex indices.
\end{memberdesc}

% Methods
\begin{methoddesc}{intersectRay}{origin, direction, earlyexit=False}
Intersect a ray with the mesh. The ray starts at \var{origin} and
travels along \var{direction} which must both be of type \class{vec3}.
\var{origin} and \var{direction} must be given with respect to the
local coordinate system L of the geometry. If \var{earlyexit} is
\code{True} the method returns after the first hit, otherwise all
triangles are tested and the result contains the nearest hit.
The return value is a tuple (\var{hit}, \var{t}, \var{faceindex}, 
\var{u}, \var{v}) where \var{hit} is a boolean that indicates if
the mesh was hit or not. \var{t} is the ray parameter, i.e. the
point of intersection is at \var{origin} + t*\var{direction}.
\var{faceindex} is the index of the face that was hit and \var{u}
\var{v} are the parameter coordinates of the intersection point.

This method tests the ray with all triangles, so it is not efficient if
you have a lot of rays to test. It is meant for only a few rays where
the preprocessing cost wouldn't be amortized.
 
The ray-triangle intersection code (non-culling case) is based on:
 
Tomas M\"oller and Ben Trumbore\\
{\em Fast, minimum storage ray-triangle intersection}\\
Journal of graphics tools, 2(1):21-28, 1997\\
\url{http://www.acm.org/jgt/papers/MollerTrumbore97/}
\end{methoddesc}





