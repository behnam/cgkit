% RIBExport

\subsection{RenderMan\textsuperscript{\textregistered} RIB export \label{ribexport}}

The module \module{cgkit.ribexport} contains the class \class{RIBExporter}
that exports the current scene as RIB and creates shaders from the materials
and light sources used in the scene.

To use the exporter you can simply use the \function{save()} command
and pass a file name with suffix \code{.rib} (see chapter \ref{commands}).
The plugin supports the following options that can be passed to
the \function{save()} command:

\begin{tableiii}{l|l|l}{code}{Option}{Default}{Description}
\lineiii{camera}{\code{None}}{Camera object to be used}
\lineiii{output}{\code{None}}{Output image file name or output specs}
\lineiii{output_framebuffer}{\code{True}}{Framebuffer output?}
\lineiii{bake}{\code{False}}{Activate texture baking mode}
\lineiii{bakemodel}{\code{None}}{Determines the model to bake}
\lineiii{bakestvar}{\code{"st"}}{Variable name of the bake texture coordinates}
\end{tableiii}

\var{camera} specifies the camera object to be used for rendering the scene.
If \code{None} is specified the first camera found in the scene is used.

\var{output} is the output image file name or a list of output specifiers.
Each specifier is a 4-tuple (\var{filename}, \var{type}, \var{mode},
\var{params}) containing the parameters for a \function{RiDisplay()} call.
The additional parameters must be given as a dictionary. \var{output} can
also be \code{None} in which case no \function{RiDisplay()} call is made.

\var{output_framebuffer} is a flag that specifies if a framebuffer
display should be opened in addition to writing the output image file
to disk. This flag is only used when \var{output} is a string.

If \var{bake} is \code{True} the exported scene will bake a texture map
instead of producing a final image. 

\var{bakemodel} is either the name of a WorldObject or a WorldObject itself.
The texture will be baked for the specified model. This parameter doesn't
have to be specified if there is only one mesh in the scene.

\var{bakestvar} is the name of the primitive variable that holds the 
texture coordinates that should be used for baking.

The plugin uses the following global options:

\begin{tableiii}{l|l|l}{code}{Option}{Default}{Description}
\lineiii{displaymode}{\code{"rgb"}}{Display mode}
\lineiii{output}{\code{None}}{Output image file name or output specs (see abvove)}
\lineiii{resolution}{\code{(640,480)}}{Output image resolution}
\lineiii{pixelsamples}{\code{(2,2)}}{Number of pixel samples}
\lineiii{shadingrate}{\code{1.0}}{Shading rate}
\lineiii{pixelfilter}{\code{None}}{Pixel filter setting}
\lineiii{tiles}{\code{None}}{Render the image in tiles}
\lineiii{rib}{\code{None}}{Additional global RIB requests}
\end{tableiii}

\var{displaymode} is the mode string for the \function{RiDisplay()} call
and determines what data is written to the output. Usually you might want
to switch between "rgb" (default) and "rgba".

\var{output} is the image output name or a list of output specifiers 
(see above).

\var{resolution} is either a 2-tuple (\var{width}, \var{height}) or a 3-tuple 
(\var{width}, \var{height}, \var{pixel aspect}) specifying the outpout
image resolution.

\var{pixelsamples} is a 2-tuple containing the pixel samples setting.

\var{shadingrate} contains the shading rate setting.

\var{pixelfilter} is a 2-tuple (\var{filter}, (\var{xwidth}, \var{ywidth}))
that sets the pixel filter to use. If \code{None} is passed the default
pixel filter of the renderer is used. Example: ("gaussian", (2,2))

\var{tiles} contains two sequences that defines the positions where
the image is split. The first sequence contains the x positions and
the second sequence the y positions where a split should occur. Each
value lies between 0 and 1, so for example tiles=((0.5,), (0.5,)) will
render the image in four equal tiles. The tiles can than be stitched
together using the \module{stitch} module.

\var{rib} is a string containing additional RIB requests that are written
in front of the frames.

It is possible to attach user RIB requests to an object simply by
adding a string attribute called \code{rib}. When present this string
will be written right before the geometry calls. For example, this
can be used to set attributes that are specific to a particular renderer:

\begin{verbatim}
s = Sphere(...)
s.rib = 'Attribute "visibility" "transmission" "opaque"'
\end{verbatim}

For an object to be exported as RIB it has to use a geometry that supports
the \class{IGeometry} protocol. Materials must support the \class{IMaterial}
protocol and light sources the \class{ILightSource} protocol. For details
on these protocols see the sub sections below. If there is an object that
does {\em not} support one of these protocols an adapter can be written
that implements the protocol on behalf of the original object.

The remainder of this section is meant to be read by developers who
want to extend the functionality of the exporter either by implementing 
adapter classes for existing classes or by implementing new geometries,
materials or light sources that natively support the respective protocol.

%--------------------------------------------
\subsubsection{The \class{IGeometry} protocol}

Every \class{GeomObject} that supports the \class{IGeometry} protocol
can be exported as RIB. If the geometry does not support the protocol
it will be ignored.

The \class{IGeometry} protocol only specifies the presence of one method:

\begin{methoddesc}[IGeometry]{render}{matid}
Creates Ri geometry requests for the geometry that has the material with 
id \var{matid} assigned to it. 
The geometry should be created in the local coordinate system
of the \class{GeomObject}. The primitive variables should also be exported.
The method can assume that there is already an enclosing 
\function{RiAttributeBegin()}/\function{RiAttributeEnd()} block around the 
call.
\end{methoddesc}

Here is an example of an adapter class that implements the \class{IGeometry}
protocol for the \class{SphereGeom} (which knows nothing about RenderMan):

\begin{verbatim}
import protocols
from ri import *

# Adapter class that implements the IGeometry protocol on behalf of the SphereGeom class
class SphereAdapter:

    protocols.advise(instancesProvide=[IGeometry], asAdapterForTypes=[SphereGeom])
    
    def __init__(self, spheregeom, proto):
        self.geom = spheregeom

    def render(self, matid):
        # A sphere can only have one single material
	if matid==0:
            r = self.geom.radius
            RiSphere(r, -r, r, 360)
\end{verbatim}


%--------------------------------------------
\subsubsection{The \class{IMaterial} protocol}

A material that supports the \class{IMaterial} protocol will be mapped
to a surface shader, displacement shader and interior shader. To support
the \class{IMaterial} protocol the following methods have to be implemented:

\begin{methoddesc}[IMaterial]{createPasses}{}
Returns a list of \class{RenderPass} objects necessary for this material
instance. These passes may be used to create environment maps, for example.
If no extra passes are required an empty list has to be returned.
\end{methoddesc}

\begin{methoddesc}[IMaterial]{preProcess}{exporter}
This method is called before the image is rendered and can be used to
create or copy image maps or do other initializations that have to be done
before the actual rendering starts. The exporter instance is provided
as argument to the method (for example, to find out where image maps
are stored).
\end{methoddesc}

\begin{methoddesc}[IMaterial]{color}{}
Return the color (as a 3-sequence of floats) for the \function{RiColor()}
call. If no color is required the method may return \code{None} in which
case no \code{Color} call is made.
\end{methoddesc}

\begin{methoddesc}[IMaterial]{opacity}{}
Return the opacity (as a 3-sequence of floats) for the \function{RiOpacity()}
call. If no opacity is required the method may return \code{None} in which
case no \code{Opacity} call is made.
\end{methoddesc}

\begin{methoddesc}[IMaterial]{surfaceShaderName}{}
Return the name of the surface shader for this material. The exporter
may still modify this name to make it unique among all generated
shaders.
\end{methoddesc}

\begin{methoddesc}[IMaterial]{surfaceShaderSource}{}
Return the RenderMan Shading Language source code for the surface shader.
Instead of the shader name the generated source code should contain
the variable \code{\$SHADERNAME} that will be substituted with the
actual name of the shader. The method may also return \code{None} if
no shader should be created. In this case, the name returned by
\method{surfaceShaderName()} is assumed to be the name of an existing shader.
\end{methoddesc}

\begin{methoddesc}[IMaterial]{surfaceShaderParams}{passes}
Returns a dictionary that contains the shader parameters that should
be used for the surface shader. The key is the name of the parameter
(including inline declarations) and the value is the actual parameter
value at the current time. The \var{passes} argument contains the list
of passes as generated by \method{createPasses()}. This list can be
used to obtain the actual name of an environment map, for example.
\end{methoddesc}

\begin{methoddesc}[IMaterial]{surfaceShaderTransform}{}
Return a mat4 containing the transformation that should be applied to
the shader.
\end{methoddesc}

\begin{methoddesc}[IMaterial]{displacementShaderName}{}
Return the name of the displacement shader for this material. The exporter
may still modify this name to make it unique among all generated
shaders. You can also return \code{None} if no displacement shader is required.
\end{methoddesc}

\begin{methoddesc}[IMaterial]{displacementShaderSource}{}
Return the RenderMan Shading Language source code for the displacement shader.
Instead of the shader name the generated source code should contain
the variable \code{\$SHADERNAME} that will be substituted with the
actual name of the shader. The method may also return \code{None} if
no shader should be created. In this case, the name returned by
\method{displacementShaderName()} is assumed to be the name of an existing 
shader.
\end{methoddesc}

\begin{methoddesc}[IMaterial]{displacementShaderParams}{passes}
Returns a dictionary that contains the shader parameters that should
be used for the displacement shader. The key is the name of the parameter
(including inline declarations) and the value is the actual parameter
value at the current time. The \var{passes} argument contains the list
of passes as generated by \method{createPasses()}.
\end{methoddesc}

\begin{methoddesc}[IMaterial]{displacementShaderTransform}{}
Return a mat4 containing the transformation that should be applied to
the shader.
\end{methoddesc}

\begin{methoddesc}[IMaterial]{displacementBound}{}
Returns a tuple (\var{coordinate system}, \var{distance}) that specifies the
maximum displacement. The distance is the maximum amount that a
surface point is displaced and is given in the specified coordinate
system.
\end{methoddesc}

\begin{methoddesc}[IMaterial]{interiorShaderName}{}
Return the name of the interior shader for this material. The exporter
may still modify this name to make it unique among all generated
shaders. You can also return \code{None} if no interior shader is required.
\end{methoddesc}

\begin{methoddesc}[IMaterial]{interiorShaderSource}{}
Return the RenderMan Shading Language source code for the interior shader.
Instead of the shader name the generated source code should contain
the variable \code{\$SHADERNAME} that will be substituted with the
actual name of the shader. The method may also return \code{None} if
no shader should be created. In this case, the name returned by
\method{displacementShaderName()} is assumed to be the name of an existing 
shader.
\end{methoddesc}

\begin{methoddesc}[IMaterial]{interiorShaderParams}{passes}
Returns a dictionary that contains the shader parameters that should
be used for the interior shader. The key is the name of the parameter
(including inline declarations) and the value is the actual parameter
value at the current time. The \var{passes} argument contains the list
of passes as generated by \method{createPasses()}.
\end{methoddesc}

\begin{methoddesc}[IMaterial]{interiorShaderTransform}{}
Return a mat4 containing the transformation that should be applied to
the shader.
\end{methoddesc}


%--------------------------------------------
\subsubsection{The \class{ILightSource} protocol}

Every world object that supports the \class{ILightSource} protocol
will be used as light source to illuminate the scene. In order to 
support the \class{ILightSource} protocol the following four methods
have to be implemented:

\begin{methoddesc}[ILightSource]{createPasses}{}
Returns a list of \class{RenderPass} objects necessary for this light
source instance. Usually the light sources will make use of the
\class{ShadowPass} pass. If no extra passes are required an empty
list has to be returned.
\end{methoddesc}

\begin{methoddesc}[ILightSource]{shaderName}{}
Return the name of the light source shader for this light source.
The exporter may still modify this name to make it unique among all
generated shaders.
\end{methoddesc}

\begin{methoddesc}[ILightSource]{shaderSource}{}
Return the RenderMan Shading Language source code for this light source.
Instead of the shader name the generated source code should contain
the variable \code{\$SHADERNAME} that will be substituted with the
actual name of the shader. The method may also return \code{None} if
no shader should be created. In this case, the name returned by
\method{shaderName()} is assumed to be the name of an existing shader.
\end{methoddesc}

\begin{methoddesc}[ILightSource]{shaderParams}{passes}
Returns a dictionary that contains the shader parameters that should
be used for this light source instance. The key is the name of the
parameter (including inline declarations) and the value is the actual
parameter value at the current time. The \var{passes} argument contains
the list of passes as generated by \method{createPasses()}. This list
can be used to obtain the actual name of a shadow map, for example.
\end{methoddesc}

%--------------------------------------------
\subsubsection{\class{RenderPass} --- Base class for all passes}

The creation of an image may take several render passes where shadow maps,
environment maps and eventually the final image are created. The number 
of passes is dependent on the number and types of light sources and the 
materials used in the scene. Every render pass is represented by a class
that is derived from the \class{RenderPass} class.

An actual pass has to implement the \method{doPass()} method where it 
has to output a frame block that generates one or more output files.
To do the scene export it can use the methods from the exporter.

\begin{classdesc}{RenderPass}{output, owner=None}
\var{output} is a list of tuples (\var{name}, \var{type}, \var{mode}, 
\var{params}) that each defines the output to create in this pass. 
Each tuple contains the parameters necessary for a \code{RiDisplay()} call.
\var{name} is the name of the output file, \var{type} is the output type
(such as "file" or "zfile"), \var{mode} specifies what information will
be stored in the file (\code{RI_RGB}, \code{RI_RGBA}, ...) and \var{params}
is a dictionary with extra parameters.

Every \class{RenderPass} instance has an attribute \var{exporter} that
will contain a reference to the exporter. So the pass object may use
the exporter methods to output the scene.
\end{classdesc}

\begin{methoddesc}{done}{}
Check if this pass is already done or not. This method is used when
shader parameters have to be determined. The output of a pass may
only be used once the pass is done and the output really exists (for
example, you cannot use a shadow map until it was created).
\end{methoddesc}

\begin{methoddesc}{doPass}{framenr}
This method has to be overwritten in derived classes where the actual
frame block is generated. \var{framenr} is the frame number to use
for the \function{RiFrameBegin()} call.
\end{methoddesc}

\begin{methoddesc}{realFilename}{filename}
Translate the logical file name \var{filename} into a real file name.
A \exception{ValueError} exception is generated if \var{filename} is
not a logical file name of this pass.
\end{methoddesc}

\begin{methoddesc}{getFilenameTable}{}
Return the filename translation table. The return value is a dictionary
that maps logical file names to real file names.
\end{methoddesc}

\begin{methoddesc}{setFilenameTable}{tab}
Set an updated filename table. This method is used to set the updated
file name table where no name clashes occur with other passes.
\end{methoddesc}

\begin{methoddesc}{initDisplays}{}
This method calls \function{RiDisplay()} for all specified outputs.
This is a helper method that can be used in \method{doPass()}.
\end{methoddesc}

%--------------------------------------------
\subsubsection{\class{ImagePass} --- Creates the final image pass}

\begin{classdesc}{ImagePass}{output, cam}
	
\end{classdesc}

%--------------------------------------------
\subsubsection{\class{ShadowPass} --- Creates a shadow map}

\begin{classdesc}{ShadowPass}{output, light, fov, resolution}
	
\end{classdesc}


%--------------------------------------------
\subsubsection{\class{RIBExporter} --- The exporter class}

Besides the usual methods that every exporter must implement this exporter
has the following helper methods that have to be used by \class{RenderPass}
objects:

\begin{methoddesc}[RIBExporter]{isExportable}{wobj}
\end{methoddesc}

\begin{methoddesc}[RIBExporter]{applyViewTransform}{V}
Applies the view transformation V (given as a \code{mat4}). This
method corresponds to a
\function{RiConcatTransform()} call. It outputs the view transformation 
\var{V} (which transforms from world to camera coordinates) as a RenderMan 
transformation. The handedness of the scene is taken into account.
\end{methoddesc}

\begin{methoddesc}[RIBExporter]{applyLightSource}{lgt}
Apply the light source \var{lgt} which must be a light source world
object. This method corresponds to a \function{RiLightSource()} call.
\end{methoddesc}

\begin{methoddesc}[RIBExporter]{applyTransformation}{T, linearvel=None, angularvel=None}
Apply the transformation \var{T} (given as a \code{mat4}). This method
corresponds to a \function{RiConcatTransform()} call. 
It outputs \var{T} as a RenderMan transformation. If a linear or angular 
velocity is given an motion block is written to enable motion blur.
\end{methoddesc}

\begin{methoddesc}[RIBExporter]{applyMaterial}{mat}
Apply the material \var{mat} which must be a \class{Material} object.
This method corresponds to the calls \function{RiColor()},
\function{RiOpacity()}, \function{RiSurface()}, \function{RiDisplacement()}
and \function{RiInterior()}.
\end{methoddesc}

\begin{methoddesc}[RIBExporter]{applyGeometry}{geom}
Apply the geometry \var{geom} which must be a \class{GeomObject}.
\end{methoddesc}

\begin{methoddesc}[RIBExporter]{writeShader}{name, source}
Write a shader source file and return the name of the shader. Usually, 
you don't have to call this method yourself as it is called by the above
applyXyz()-methods.
\end{methoddesc}

