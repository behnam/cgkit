\subsection{vec3 - 3d vector}
\label{vec3}

A \class{vec3} represents a 3D vector type that can be used to store
points, vector, normals or even colors. You can construct a
\class{vec3} by several ways:

\begin{verbatim}
# all components are set to zero
v = vec3()

-> (0.0000, 0.0000, 0.0000)

# set all components to one value
v = vec3(2.5)

-> (2.5000, 2.5000, 2.5000)

# set a 2d vector, the 3rd component will be zero
v = vec3(1.5, 0.8)

-> (1.5000, 0.8000, 0.0000)

# initialize all three components
v = vec3(1.5, 0.8, -0.5)

-> (1.5000, 0.8000, -0.5000)
\end{verbatim}

Additionally you can use all of the above, but store the values inside
a tuple, a list or a string:

\begin{verbatim}
v = vec3([1.5, 0.8, -0.5])
w = vec3("1.5, 0.8")
\end{verbatim}

Finally, you can initialize a vector with a copy of another vector:

\begin{verbatim}
v = vec3(w)
\end{verbatim}

A \class{vec3} can be used just like a list with 3 elements, so you
can read and write components using the index operator or by accessing
the components by name:

\begin{verbatim}
>>> v=vec3(1,2,3)
>>> print v[0]
1.0
>>> print v.y
2.0
\end{verbatim}

%----------------------------------------
{\bf Mathematical operations}

The mathematical operators are supported with the following
combination of types:

\begin{verbatim}
vec3  =  vec3 + vec3
vec3  =  vec3 - vec3
float =  vec3 * vec3      # dot product
vec3  = float * vec3
vec3  =  vec3 * float
vec3  =  vec3 / float
vec3  =  vec3 % float     # each component
vec3  =  vec3 % vec3      # component wise
vec3  = -vec3
float =  vec3[i]          # get or set element
\end{verbatim}

Additionally, you can compare vectors with \code{==}, \code{!=}, \code{<}, 
\code{<=}, \code{>}, \code{>=}. Each
comparison (except \code{<} and \code{>}) takes an epsilon environment
into account, this means two values are considered to be equal if
their absolute difference is less than or equal to a threshold value
epsilon. You can read and write this threshold value using the
functions \function{getEpsilon()} and \function{setEpsilon()}.

Taking the absolute value of a vector will return the length of the vector: 

\begin{verbatim}
float = abs(v)            # this is equivalent to v.length()
\end{verbatim}

%----------------------------------------
{\bf Methods}

\begin{methoddesc}{angle}{other}
Return angle (in radians) between \var{self} and \var{other}.
\end{methoddesc}

\begin{methoddesc}{cross}{other}
Return cross product of \var{self} and \var{other}.
\end{methoddesc}

\begin{methoddesc}{length}{}
Returns the length of the vector ($\sqrt{x^2+y^2+z^2}$). This is
equivalent to calling \code{abs(self)}.
\end{methoddesc}

\begin{methoddesc}{normalize}{}
Returns normalized vector. If the method is called on the null vector
(where each component is zero) a \exception{ZeroDivisionError} is raised.
\end{methoddesc}

\begin{methoddesc}{reflect}{N}
Returns the reflection vector. \var{N} is the surface normal which has to be
of unit length.
\end{methoddesc}

\begin{methoddesc}{refract}{N, eta}
Returns the transmitted vector. \var{N} is the surface normal which has to
be of unit length. \var{eta} is the relative index of refraction. If the
returned vector is zero then there is no transmitted light because of
total internal reflection.
\end{methoddesc}

\begin{methoddesc}{ortho}{}
Returns a vector that is orthogonal to \var{self} (where
\code{self*self.ortho()==0}).
\end{methoddesc}

