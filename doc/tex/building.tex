% Building and Installing

This chapter provides information about how to build the kit yourself
from the sources. You can skip this chapter if the kit is already
installed on your system or if you are installing binary versions.

%----------------------------------------------------------------------
\section{Building from the sources}

The following instructions will build and install the full version of cgkit.
You can also install the 'light' version by setting the variable 
\code{INSTALL_CGKIT_LIGHT} to \code{True} in your config file (just rename
the file \code{config_template.cfg} to \code{config.cfg} and uncomment the
corresponding line at the top of the file). Except for Python itself, 
installing the light version does not require any additional compiler, tool
or library, you simply have to call 

\begin{verbatim}
> python setup.py install    # that's all for the light version
\end{verbatim}

while being in the root directory of the cgkit sources. For installing
the light version you can skip the remainder of this section, but read
on if you actually want to install the full version.

The steps required to build the full version of the kit are the same
on every platform.  The build process assumes you have the following
tools/libraries installed on your system:

\begin{itemize}
\item \ulink{Boost}{http://www.boost.org/} (v1.31) - Make sure that Boost is
 installed including Boost.Python (e.g. Mac users who are using darwinports
 have to install the "+python" variant).
\item \ulink{SCons}{http://www.scons.org/} (v0.96.1)
\item \ulink{lib3ds}{http://lib3ds.sourceforge.net/} (v1.3.0) - {\bf Optional}. You need this if you want to import/export 3DS files.
\item \ulink{CyberX3D}{http://www.cybergarage.org/vrml/cx3d/cx3dcc/index.html} - {\bf Optional}. You need this if you want to import/export VRML or X3D files.
\item \ulink{OGRE}{http://www.ogre3d.org/} - {\bf Optional}. You need this if you want to use the OGRE-Viewer.
\item \ulink{STLport}{http://www.stlport.com/} - Only required on Windows when
 you want to use OGRE.
\item \ulink{3DxWare SDK}{http://www.3dconnexion.com/sdk.htm} - {\bf Optional} (currently Windows only). You need this if you want to use a SpaceMouse or SpaceBall.
\item \ulink{Wintab Programmer's Kit}{http://www.pointing.com/} - {\bf Optional} (Windows only). You need this if you want to use a tablet (or similar device).
\item \ulink{GloveSDK}{http://www.5dt.com/} - {\bf Optional}. You need this if you want to use a data glove. {\bf Note}: Download the version 2 SDK from the Ultra series, this can also be used with older gloves. You could also use version 1, but some functionality won't be available then.
\end{itemize}

If you have successfully installed the above tools and libraries you
can proceed with building the kit.The first step is to get the cgkit
sources. You can either download the source archive or check out the
latest version from CVS. The following build instructions apply to
both versions.

The package consists of three parts:

\begin{enumerate}
\item A pure C++ library that is independent of Python. This library is
 located in the \file{supportlib} directory.
\item Python wrappers around the above C++ library (using Boost.Python). 
 These wrappers are located in the \file{wrappers} directory.
\item The actual Python package using the wrappers. The package is located
 in the \file{cgkit} directory. The command line tools are in the main 
 directory.
\end{enumerate}

Part 2) and 3) is built and installed via the Python distutils (i.e.
the \file{setup.py} script). But before you can do so, you have to
build the C++ library manually using SCons.

{\bf Building the C++ support library:}

The C++ library is located in the directory \file{supportlib}. The library
uses SCons as its build system. If you have to customize the build process
you can create a file \file{cpp_config.cfg} where you can set some build
variables (e.g. adding search paths for include files). There is a template
file \file{cpp_config_template.cfg} which you can use to create the actual
config file. However, if you have installed every library in standard
directories it may well be that you don't need any config file at all.
Eventually you can build the library by calling \program{scons}:

\begin{verbatim}
> cd supportlib
> # ...create & modify cpp_config.cpp if necessary...
> scons
\end{verbatim}

If everything went fine you can see the result in the \file{lib}
subdirectory (\file{libcore.a} on Linux or \file{core.lib} on Windows).

{\bf Building the Python wrappers:}

The next step is to build and install the Python package. The package
uses the distutils, so you will find the familiar \file{setup.py}
script in the main directory. Customization of the build process can again
be done in a file called \file{config.cfg} which is executed by the setup
script. There is a template file \file{config_template.cfg} which you can
use to create your own config file. After setting up your
configuration you can install the package with the usual procedure:

\begin{verbatim}
> cd ..  # if you were still in the supportlib directory
> python setup.py install
\end{verbatim}

(see the manual \citetitle{Installing Python Modules} in your Python
documentation for more details about the distutils and how to proceed
if you have any special requirements)

Please also have a look at section \ref{externaldeps} for a list of
third-party packages you might have to install before you can use cgkit.
You can check cgkit and its dependencies with the script 
\file{utilities/checkenv.py} that is part of the source archive.
For a more thorough test you can run the unit tests in the \file{unittests}
directory.

A note to developers: You can build the package inplace by calling

\begin{verbatim}
> python setup.py build_ext --inplace
\end{verbatim}

In this case, the resulting binaries will be placed directly in the
\file{cgkit} directory which will then contain the entire package. Use
the \envvar{PYTHONPATH} variable and add the path to the main
directory if you want to use the inplace version from any other
directory.
