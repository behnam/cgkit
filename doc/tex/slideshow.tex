% SlideShow component

\section{\class{SlideShow} ---
         Displaying a series of images}

The \class{SlideShow} class can be used to display image files as
a slide show. The component sets up a scene and displays an image
sequence with user defined transitions.

\begin{classdesc}{SlideShow}{name = "SlideShow",\\ 
                             slides = [], \\
                             auto_insert = True}

\var{slides} is either a string specifying the image files (the string
may contain wildcards) or a list of \class{Slide} objects where each
object represents one or more images.
\end{classdesc}

Example:

\begin{verbatim}
# File: slides.py

SlideShow(
    slides = [
              Slide("image01.jpg", XFade(1.0) ),
              Slide("image02.jpg", XFade(1.0, 0.3) ),
              Slide("image03.jpg", XCube(2.0) ),
              Slide("presentation/slides*.png", XFade(0.5) )
             ]
)
\end{verbatim}

The slide show is started with the viewer tool like this:

\begin{verbatim}
viewer.py slides.py -f50 -F
\end{verbatim}

In this case, the frame rate is increased to 50 frames per second (to get
smoother transitions) and the display is set to full screen.

When the slide show is running you can jump to the next slide by pressing
a mouse button, the \kbd{Enter} key or the \kbd{PageDown} key.
In case you move the camera, you can reset it with the \kbd{q} key.

%----------------------------------------------------------------------
\subsection{Slide class}

The \class{Slide} class represents one or more image files and contains
one transition that is used for all files.

\begin{classdesc}{Slide}{filepattern, transition = XCube()}

\var{filepattern} specifies the image files to load and may include
wildcards to select more than one file.

\var{transition} is a transition class that determines the transition
that is applied after each image in this slide object.
\end{classdesc}

%----------------------------------------------------------------------
\subsection{XFade transition}

The \class{XFade} transition implements a smooth cross fade between two
images.

\begin{classdesc}{XFade}{duration=2.0, zmove=0.0}

\var{duration} is the length of the transition in seconds.

If \var{zmove} is greater than 0, the old image will be moved towards
the camera during the transition which makes it scale up when viewed
with a perspective camera.
\end{classdesc}

%----------------------------------------------------------------------
\subsection{XCube transition}

The \class{XCube} class implements a transition where the images seem
to be attached on two adjacent sides of a cube and the cube rotates
during the transition.

\begin{classdesc}{XCube}{duration=2.0}

\var{duration} is the length of the transition in seconds.
\end{classdesc}

