% PIDController component

\section{\class{PIDController} ---
         Proportional-Integral-Derivative controller}

A PID controller is a standard component in industrial control
applications which tries to keep a measured value at a given target
value (the {\em setpoint}). The measured value has to be plugged into
the input slot (\code{input_slot}) and the output of the PID controller
can be read from \code{output_slot}.

%\begin{displaymath}
%output(t) = K_p \cdot err(t) + K_i \cdot \int err(t')\, dt' + K_d \cdot \frac{d\,err}{dt}
%\end{displaymath}

For example, you can use a PID controller in conjunction with the
joints in the \class{ODEDynamics} component to keep a hinge or slider
at a particular position. In this case the angle or position is used
as input to the PID controller and the output controls the motor velocity.

\begin{classdesc}{PIDController}{name = "PIDController",\\ 
                              setpoint = 0.0, \\
                              Kp = 0.0, \\
                              Ki = 0.0, \\
                              Kd = 0.0, \\
                              maxout = 999999, \\
                              minout = -999999, \\
                              auto_insert = True}

\var{setpoint} is the target value that should be maintained.

\var{Kp} is the weight for the proportional part, \var{Ki} the weight
for the integral part and \var{Kd} the weight for the derivative part.

\var{maxout} and \var{minout} are used to clamp the output value.
\end{classdesc}

A \class{PIDController} has the following slots:

\begin{tableiv}{l|l|c|l}{code}{Slot}{Type}{Access}{Description}
\lineiv{input_slot}{float}{rw}{The "measured" value}
\lineiv{setpoint_slot}{float}{rw}{The target value}
\lineiv{output_slot}{float}{r}{Controller output}
\lineiv{maxout_slot}{float}{rw}{Maximum output value}
\lineiv{minout_slot}{float}{rw}{Minimum output value}
\lineiv{Kp_slot}{float}{rw}{Weight for the proportional term}
\lineiv{Ki_slot}{float}{rw}{Weight for the integral term}
\lineiv{Kd_slot}{float}{rw}{Weight for the derivative term}
\end{tableiv}



