% Scene class

\section{Scene object}

The \class{Scene} class is a container for anything that should also
be visible to other objects. The "visible" part of the scene is called
the {\em world} which is what gets rendered on screen (everything that
has a coordinate system is part of the world). However, the scene can
also contain objects that are not part of the world. For example, the
global timer or any other object that is used somehow but has no
direct visible representation.

You can retrieve the global scene object by calling the function 
\function{getScene()}. 

\begin{funcdesc}{getScene}{}
Returns the global scene object.
\end{funcdesc}

If a script is invoked via a tool such as the viewer or render tool,
the variable \code{scene} is automatically initialized with the global
scene object. To inspect what objects the scene currently contains,
you can just iterate over a scene:

\begin{verbatim}
>>> for item in scene: print item
...
<cgkit.timer.Timer object at 0x0907FEA0>
<cgkit.worldobject.WorldObject object at 0x0907FE70>
\end{verbatim}

This is what an empty scene looks like. The two scene items you see here
are the globel timer and the root object of the world. These objects are
automatically created by the scene.

%\begin{classdesc}{Scene}{}
%Scene object...
%\end{classdesc}

The scene object has the following attributes and methods:

\begin{memberdesc}{handedness}
Specifies if the scene uses a left-handed or right-handed coordinate
system. The value is a string that is either be \code{'l'} or \code{'r'}.
The default is right handed.
\end{memberdesc}

\begin{memberdesc}{up}
Specifies the direction that is considered to be the "up" direction. The 
value must be a \class{vec3}. The default is (0,0,1) (z axis).
\end{memberdesc}

\begin{memberdesc}{unit}
\end{memberdesc}

\begin{memberdesc}{unit_scale}
\end{memberdesc}

\begin{methoddesc}{worldRoot}{}
Return the root world object.
\end{methoddesc}

\begin{methoddesc}{walkWorld}{root=None}
Walk the world tree and yield each object. This method can be used to
iterator over the entire world tree or a subtree thereof. The argument 
\var{root} specifies the root of the tree which is to traverse (the 
root itself will not be returned).
\end{methoddesc}

\begin{methoddesc}{timer}{}
Return the global timer component (see section \ref{timer}).
\end{methoddesc}

\begin{methoddesc}{clear}{}
Clear the entire scene.
\end{methoddesc}

\begin{methoddesc}{insert}{item}
Insert an item into the scene.
\end{methoddesc}

\begin{methoddesc}{item}{name}
Return the item with the specified name.
\end{methoddesc}

\begin{methoddesc}{worldObject}{name}
Return the world object with the specified name. You can use the character
\character{|} as a path separator.
\end{methoddesc}

\begin{methoddesc}{boundingBox}{}
Return the bounding box of the entire scene.
\end{methoddesc}

\begin{methoddesc}{setJoystick}{joystick}
Set a joystick object.
\end{methoddesc}

\begin{methoddesc}{getJoystick}{id}
Get a joystick object (see section \ref{joystick}). A dummy joystick object is
returned if there is no joystick with the specified id.
\end{methoddesc}

\begin{methoddesc}{hasGlobal}{name}
Return \code{True} if a global option with name \var{name} exists.
\end{methoddesc}

\begin{methoddesc}{getGlobal}{name, default=None}
Get the global option with the given name. \var{default} is returned
if the option does not exist. The options can be set using the 
\class{Globals} class or the \method{setGlobal()} method.
\end{methoddesc}

\begin{methoddesc}{setGlobal}{name, value}
Set the global option \var{name} to \var{value}.
\end{methoddesc}
