\section{\module{events} ---
         Standard event names and classes}

\declaremodule{extension}{cgkit.events}
\modulesynopsis{Standard event names and classes}

This module defines the names of the standard events and the corresponding
event classes.

The actual event name is a capitalized version of the symbol where
all underscores are removed (for example, \code{STEP_FRAME = "StepFrame"}).

\begin{datadesc}{STEP_FRAME}
This event is created whenever the timer is stepped forward one frame
using \code{timer.step()}. The event takes no arguments.
\end{datadesc}

\begin{datadesc}{RESET}
This event is created to reset the simulation/animation. This
effectively sets the time back to 0. Any component that has an
internal state should connect to this method and reset its state
whenever the event is signalled.
\end{datadesc}

\begin{datadesc}{KEY_PRESS}
A key was pressed on the keyboard. The callback function receives 
a \class{KeyEvent} object as argument.
\end{datadesc}

\begin{datadesc}{KEY_RELEASE}
A key was released on the keyboard. The callback function receives 
a \class{KeyEvent} object as argument.
\end{datadesc}

\begin{datadesc}{LEFT_DOWN}
The left mouse button was pressed. The callback function receives 
a \class{MouseButtonEvent} object as argument.
\end{datadesc}

\begin{datadesc}{LEFT_UP}
The left mouse button was released. The callback function receives 
a \class{MouseButtonEvent} object as argument.
\end{datadesc}

\begin{datadesc}{MIDDLE_DOWN}
The middle mouse button was pressed. The callback function receives 
a \class{MouseButtonEvent} object as argument.
\end{datadesc}

\begin{datadesc}{MIDDLE_UP}
The middle mouse button was released. The callback function receives 
a \class{MouseButtonEvent} object as argument.
\end{datadesc}

\begin{datadesc}{RIGHT_DOWN}
The right mouse button was pressed. The callback function receives 
a \class{MouseButtonEvent} object as argument.
\end{datadesc}

\begin{datadesc}{RIGHT_UP}
The right mouse button was released. The callback function receives 
a \class{MouseButtonEvent} object as argument.
\end{datadesc}

\begin{datadesc}{MOUSE_BUTTON_DOWN}
A mouse button other than the left, middle or right button was pressed. The
callback function receives a \class{MouseButtonEvent} object as
argument.
\end{datadesc}

\begin{datadesc}{MOUSE_BUTTON_UP}
A mouse button other than the left, middle or right button was released. The
callback function receives a \class{MouseButtonEvent} object as
argument.
\end{datadesc}

\begin{datadesc}{MOUSE_MOVE}
The mouse was moved. The callback function receives a
\class{MouseMoveEvent} object as argument.
\end{datadesc}

\begin{datadesc}{MOUSE_WHEEL}
The mouse wheel was rotated. The callback function receives a
\class{MouseWheelEvent} object as argument.
\end{datadesc}

\begin{datadesc}{JOYSTICK_AXIS}
\end{datadesc}

\begin{datadesc}{JOYSTICK_BALL}
\end{datadesc}

\begin{datadesc}{JOYSTICK_HAT}
\end{datadesc}

\begin{datadesc}{JOYSTICK_BUTTON_DOWN}
\end{datadesc}

\begin{datadesc}{JOYSTICK_BUTTON_UP}
\end{datadesc}


\begin{datadesc}{SPACE_MOTION}
A SpaceMouse/SpaceBall was moved or rotated. The callback function receives a
\class{SpaceMotionEvent} object as argument.
\end{datadesc}

\begin{datadesc}{SPACE_BUTTON_DOWN}
A SpaceMouse/SpaceBall button was pressed. The callback function receives a
\class{SpaceButtonEvent} object as argument.
\end{datadesc}

\begin{datadesc}{SPACE_BUTTON_UP}
A SpaceMouse/SpaceBall button was released. The callback function receives a
\class{SpaceButtonEvent} object as argument.
\end{datadesc}

\begin{datadesc}{SPACE_BUTTON_ZERO}
The user stopped moving/rotating the SpaceMouse/SpaceBall. The event takes
no arguments.
\end{datadesc}


\begin{datadesc}{TABLET}
A tablet (or similar pointing device) generated an event.
\end{datadesc}


\begin{notice}[note]
It actually depends on the application used to process a scene if a
particular type of event occurs or not. For example, events like a key press
or mouse move do occur in the interactive viewer tool, but not in the
offline rendering tool.
\end{notice}

%-----------
\subsection{KeyEvent object}

The \class{KeyEvent} class is passed as argument to the \code{KEY_PRESS}
and \code{KEY_RELEASE} events. The class contains the attributes
\code{key}, \code{keycode} and \code{mods} (see the constructor for
a description).

\begin{classdesc}{KeyEvent}{key, keycode, mods=0}
\var{key} is a unicode string that contains the key that was 
pressed or released. The key is the translated key which means
that it may depend on other keys (such as Shift) or previously pressed
keys. For example, if you just press the 'a' key you receive a lower
case 'a', but if you press Shift and 'a' then you will get an upper
case 'A'.

\var{keycode} is the code of the untranslated key that was pressed
or released. The values for common keys are defined in the \module{keydefs}
module (i.e. \code{KEY_LEFT}, \code{KEY_UP}, etc.).

\var{mods} contains the modifier flags which is a combination of
\code{KEYMOD_SHIFT}, \code{KEYMOD_CONTROL}, \code{KEYMOD_ALT} and
\code{KEYMOD_META}. The respective flag is set if the Shift, Control,
Alt or Meta key was pressed while the event key was pressed or released.
\end{classdesc}

\begin{methoddesc}{shiftKey}{}
Return \code{True} if the key is a Shift key (no matter if it is the
left or right key).
\end{methoddesc}

\begin{methoddesc}{controlKey}{}
Return \code{True} if the key is a Control key (no matter if it is the
left or right key).
\end{methoddesc}

\begin{methoddesc}{altKey}{}
Return \code{True} if the key is an Alt key (no matter if it is the
left or right key).
\end{methoddesc}

%-----------
\subsection{MouseButtonEvent object}

The \class{MouseButtonEvent} class is passed as argument whenever
a mouse button was pressed or released.

\begin{classdesc}{MouseButtonEvent}{button, x, y, x0, y0}

\var{button} is the mouse button number (1 = left button, 2 = middle button,
3 = right button).

\var{x}/\var{y} is the pixel position of the mouse where 0/0 is the upper
left corner.

\var{x0}/\var{y0} is the normalized position of the mouse where each 
component lies in the range [0,1).

\end{classdesc}

%-----------
\subsection{MouseWheelEvent object}

The \class{MouseWheelEvent} class is passed as argument whenever
the mouse wheel is rotated.

\begin{classdesc}{MouseWheelEvent}{delta, x, y, x0, y0}

\var{delta} is the wheel delta (usually 120 if the wheel was rotated
forward and -120 if the wheel was rotated backward).

\var{x}/\var{y} is the pixel position of the mouse where 0/0 is the upper
left corner.

\var{x0}/\var{y0} is the normalized position of the mouse where each 
component lies in the range [0,1).

\end{classdesc}

%-----------
\subsection{MouseMoveEvent object}

The \class{MouseMoveEvent} class is passed as argument whenever the
mouse is moved.

\begin{classdesc}{MouseMoveEvent}{x, y, dx, dy, x0, y0, dx0, dy0, buttons}

\var{x}/\var{y} is the pixel position of the mouse where 0/0 is the upper
left corner.

\var{dx}/\var{dy} is the mouse delta, i.e. the distance travelled since
the last event.

\var{x0}/\var{y0} is the normalized position of the mouse where each 
component lies in the range [0,1).

\var{dx0}/\var{dy0} is the normalized mouse delta.

\var{buttons} contains the mouse buttons that are currently pressed.
Each bit corresponds to a mouse button (bit 0 = left button, bit 1 = middle
button, etc.).
\end{classdesc}

%-----------
\subsection{JoystickAxisEvent object}

\begin{classdesc}{JoystickAxisEvent}{joystick, axis, value}
\end{classdesc}


\subsection{JoystickHatEvent object}

\begin{classdesc}{JoystickHatEvent}{joystick, hat, x, y}
\end{classdesc}


\subsection{JoystickBallEvent object}

\begin{classdesc}{JoystickBallEvent}{joystick, ball, value}
\end{classdesc}


\subsection{JoystickButtonEvent object}

\begin{classdesc}{JoystickButtonEvent}{joystick, button}
\end{classdesc}

%-----------
\subsection{SpaceMotionEvent object}

\begin{classdesc}{SpaceMotionEvent}{translation, rotation, period}

\var{translation} is a \class{vec3} containing the curreent translation of
the space mouse. The coordinate system of the space mouse is left
handed and such that the X axis points to the right, the Y axis
upwards and the Z axis into the screen.

\var{rotation} is a \class{vec3} containing the current rotation of the
space mouse. The vector is the rotation axis and the magnitude represents
the angle.

\var{period} is the time in milliseconds since the last device event.
\end{classdesc}

\subsection{SpaceButtonEvent object}

\begin{classdesc}{SpaceButtonEvent}{button}
\end{classdesc}
