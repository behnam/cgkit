% PolyhedronGeom

\section{\class{PolyhedronGeom} ---
         Polyhedron geometry}

The \class{PolyhedronGeom} class stores a collection of general planar
concave polygons that may also contain holes. Each polygon is described
by a sequence of vertex loops. The first loop defines the polygon boundary
and each subsequent loop describes a hole. Each loop is a sequence of
vertex indices.

\begin{classdesc}{PolyhedronGeom}{}
Creates an empty polyhedron.
\end{classdesc}

A \class{PolyhedronGeom} has the following slots:

\begin{tableiv}{l|l|c|l}{code}{Slot}{Type}{Access}{Description}
\lineiv{verts_slot}{vec3 array}{rw}{The polygon vertices}
\end{tableiv}

% Attributes
\begin{memberdesc}{verts}
This attribute contains the sequence of polygon vertices.
\end{memberdesc}

% Methods
\begin{methoddesc}{hasPolysWithHoles}{}
Return \code{True} if there is at least one polygon with a hole.
\end{methoddesc}

\begin{methoddesc}{getNumPolys}{}
Return the number of polygons.
\end{methoddesc}

\begin{methoddesc}{getNumLoops}{poly}
Return the number of vertex loops in the polygon with index \var{poly}.
\end{methoddesc}

\begin{methoddesc}{getNumVerts}{poly, loop}
Return the number of vertex indices in one particular loop. \var{poly}
is the polygon index and \var{loop} the loop index.
\end{methoddesc}

\begin{methoddesc}{setNumPolys}{num}
Allocate space for \var{num} polygons.
\end{methoddesc}

\begin{methoddesc}{setNumLoops}{poly, num}
Allocate space for \var{num} loops in the polygon with index \var{poly}.
\end{methoddesc}

\begin{methoddesc}{getLoop}{poly, loop}
Return a loop from a polygon. \var{poly} is the polygon index and \var{loop}
the loop index. The return value is a sequence of vertex indices.
\end{methoddesc}

\begin{methoddesc}{setLoop}{poly, loop, vloop}
Set a new polygon loop. \var{poly} is the polygon index, \var{loop}
the loop index and \var{vloop} a sequence of vertex indices.
\end{methoddesc}

\begin{methoddesc}{getPoly}{poly}
Return a polygon. \var{poly} is the polygon index. The return value
is a sequence of vertex loops.
\end{methoddesc}

\begin{methoddesc}{setPoly}{poly, polydef}
Set a polygon. \var{poly} is the polygon index and \var{polydef} a sequence
of vertex loops.
\end{methoddesc}




