% Expression component

\section{\class{Expression} ---
         Using expressions to drive animations}

An \class{Expression} outputs a single value that is driven by a user
defined expression. The expression is specified by a string and can
use an arbitrary number of parameters. The parameters and their
default values have to be provided to the constructor via keyword
arguments. An exception is the special variable \code{t} which will always
hold the current time (unless you declare it explicitly).  For each
parameter a slot is created (\code{<name>_slot}), so it is also possible to
animate the parameters. The output value can be accessed via the
\code{output} and \code{output_slot} attributes.

\begin{classdesc}{Expression}{expr = "",\\ 
			      exprtype = None, \\
                              name = "Expression", \\
                              **parameters}

\var{expr} is the expression (in Python syntax) that is used to compute
the output values. It may use mathematical functions defined in the
\module{math} and \module{sl} modules.

\var{exprtype} is the return type of the expression (\code{float}, 
\code{vec3}, ...). If this parameter is not given, the component tries
to figure out the return type itself by evaluating the expression using
the default values and inspecting the return type. If the expression
returns a tuple the output type is automatically set to vec3, vec4, mat3
or mat4 (depending on the number of elements).

\var{name} is the component's name.

Any additional keyword argument is a parameter used in the expression.
Every parameter used in the expression has to be declared except the
variable \code{t} which will automatically receive the current time value.
If you declare \code{t} yourself, it will be just an ordinary variable.
\end{classdesc}


Example:

\begin{verbatim}
s = Sphere()
e = Expression("1.0 + amp*sin(freq*t)", amp=0.2, freq=2.0)
e.output_slot.connect(s.radius_slot)
\end{verbatim}

When the above expression is created, it will have the following slots:

\begin{itemize}
\item \code{output_slot}
\item \code{amp_slot}
\item \code{freq_slot}
\item \code{t_slot}
\end{itemize}