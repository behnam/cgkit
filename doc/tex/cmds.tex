% cmds

This chapter describes the builtin commands which are implemented in
the \module{cgkit.cmds} module and which are always available. New
commands can be added via the \function{register()} function (for
example, by plugins), so there might actually be more commands
available than are listed here.

% fitCone
\begin{funcdesc}{fitCone}{pos, obj}
Compute a cone that has its apex at \var{pos} and that includes
\var{obj}. The generated cone is the minimal cone that entirely
contains the bounding box of \var{obj} (which must be a
\class{WorldObject}). \var{pos} is the apex of the cone given in world
coordinates. 
The return value is a tuple (\var{n}, \var{w}) where \var{n} is the 
axis direction of the cone and \var{w} is the (full) angle in radians.
\end{funcdesc}

% convertToTriMesh
\begin{funcdesc}{convertToTriMesh}{objs}
Convert all specified objects into triangle meshes. \var{obj} may be an
individual world object or a sequence of objects. Each object may be
specified by name or by reference.
\end{funcdesc}

% setupObjectNames
\begin{funcdesc}{setupObjectNames}{}
Return a string that can be executed to 'import' all scene names.
After executing the returned string you can access all top level
objects via their name.

Example:

\begin{verbatim}
>>> listWorld()
Root
+---Bottom (Box/BoxGeom)
+---GLPointLight (GLPointLight/-)
+---GLPointLight1 (GLPointLight/-)
+---Middle (Sphere/SphereGeom)
+---TargetCamera (TargetCamera/-)
+---Top (Box/BoxGeom)
>>> exec setupObjectNames()
>>> Bottom
<cgkit.box.Box object at 0x094765D0>
>>> TargetCamera
<cgkit.targetcamera.TargetCamera object at 0x09470BD0>
>>> Top
<cgkit.box.Box object at 0x09476690>
\end{verbatim}
\end{funcdesc}

% group()
\begin{funcdesc}{group}{*children, **name}
Group several world objects together. All non keyword arguments
somehow refer to world objects that will all be grouped together. An
argument may be either a \class{WorldObject}, the name of a world
object or a sequence of world objects or names. The name of the new group 
may be given via the \code{name} keyword argument.
The return value is the newly created Group object.

Example:

\begin{verbatim}
>>> b1=Box()
>>> b2=Box()
>>> s1=Sphere()
>>> listWorld()
Root
+---Box (Box/BoxGeom)
+---Box1 (Box/BoxGeom)
+---Sphere (Sphere/SphereGeom)
>>> group("Box", "Box1", s1, name="MyGroup")
>>> listWorld()
Root
+---MyGroup (Group/-)
    +---Box (Box/BoxGeom)
    +---Box1 (Box/BoxGeom)
    +---Sphere (Sphere/SphereGeom)
\end{verbatim}
\end{funcdesc}

% ungroup()
\begin{funcdesc}{ungroup}{group}
Break up a group in its individual components. \var{group} is a group
object or the name of a group object. This function does not only work
with \class{Group} objects but actually with any object that has no direct
geometry assigned to it.

Example:

\begin{verbatim}
>>> listWorld()
Root
+---MyGroup (Group/-)
    +---Box (Box/BoxGeom)
    +---Box1 (Box/BoxGeom)
    +---Sphere (Sphere/SphereGeom)
>>> ungroup("MyGroup")
>>> listWorld()
Root
+---Box (Box/BoxGeom)
+---Box1 (Box/BoxGeom)
+---Sphere (Sphere/SphereGeom)
\end{verbatim}
\end{funcdesc}

% replaceMaterial()
\begin{funcdesc}{replaceMaterial}{name, newmat}
Iterate over all world objects and replace each material called \var{name}
with material object \var{newmat}. 
\end{funcdesc}

% link()
\begin{funcdesc}{link}{childs, parent=None, relative=False}
Link the world objects \var{childs} to \var{parent}. Previously existing
links are removed. If \var{parent} is \code{None} then the links are just
removed. The argument \var{childs} may be either a single world object or
a sequence of world objects. Instead of world objects you can also pass
the names of the world objects.

By default, the absolute position and orientation of the children is
maintained (i.e. the local transform is modified). If you set \var{relative}
to \code{True} the local transform is not modified which will change the
position/orientation of the children (unless the parent transform is the
identity).

Note: The function modifies the name of a child object if there would
be a clash with an existing object under the new parent.
\end{funcdesc}

% drawClear()
\begin{funcdesc}{drawClear}{}
Clear all drawing objects.
\end{funcdesc}

% drawMarker()
\begin{funcdesc}{drawMarker}{pos, col=(1,1,1), size=1}
Draw a marker (a point). \var{col} is the color of the marker and
\var{size} its radius.
\end{funcdesc}

% drawLine()
\begin{funcdesc}{drawLine}{pos1, pos2, col=(1,1,1), size=1}
Draw a line from \var{pos1} to \var{pos2}. \var{col} is the color of the 
line and \var{size} its width.
\end{funcdesc}

% drawText()
\begin{funcdesc}{drawText}{pos, txt, font=None, col=(1,1,1)}
Draw the text \var{txt} at position \var{pos} (3D position). \var{col}
is the color of the text. The text is drawn using GLUT
functionality. \var{font} is a GLUT font constant as defined in the
\module{OpenGL.GLUT} module. It can take one of the following values:

\begin{itemize}
\item \code{GLUT_BITMAP_8_BY_13}
\item \code{GLUT_BITMAP_9_BY_15} (default)
\item \code{GLUT_BITMAP_TIMES_ROMAN_10}
\item \code{GLUT_BITMAP_TIMES_ROMAN_24}
\item \code{GLUT_BITMAP_HELVETICA_10}
\item \code{GLUT_BITMAP_HELVETICA_12}
\item \code{GLUT_BITMAP_HELVETICA_18}
\end{itemize}

\end{funcdesc}

% listWorld()
\begin{funcdesc}{listWorld}{}
List the contents of the world as a tree. Example:

\begin{verbatim}
>>> load("demo4.py")
>>> listWorld()
Root
+---GLDistantLight (GLTargetDistantLight/-)
+---GLDistantLight1 (GLTargetDistantLight/-)
+---Sphere (Sphere/SphereGeom)
|   +---Box0 (Box/BoxGeom)
|   +---Box1 (Box/BoxGeom)
|   +---Box2 (Box/BoxGeom)
|   +---Box3 (Box/BoxGeom)
|   +---Box4 (Box/BoxGeom)
|   +---Box5 (Box/BoxGeom)
|   +---Box6 (Box/BoxGeom)
|   +---Box7 (Box/BoxGeom)
|   +---Box8 (Box/BoxGeom)
|   +---Box9 (Box/BoxGeom)
+---TargetCamera (TargetCamera/-)
\end{verbatim}
\end{funcdesc}

% load()
\begin{funcdesc}{load}{filename, **options}
Loads the given file without deleting the scene, so the contents of
the file is appended to the current scene. Any additional keyword argument
is considered to be an option and is passed to the importer.

To be able to load the file there must be an appropriate import class
(protocol: "Import") available in the plugin manager. The class is
determined by examining the file extension. If no importer is found a
\exception{NoImporter} exception is thrown. See chapter \ref{importplugins}
for the available standard importers.

Any exception generated in the importer is passed to the caller.
\end{funcdesc}

% save()
\begin{funcdesc}{save}{filename, **options}
Saves the current scene. Any additional keyword argument
is considered to be an option and is passed to the exporter.

To be able to save the scene there must be an appropriate export class
(protocol: "Export") available in the plugin manager. The class is
determined by examining the file extension. If no exporter is found a
\exception{NoExporter} exception is thrown. 

Any exception generated in the exporter is passed to the caller.
\end{funcdesc}

% reset()
\begin{funcdesc}{reset}{}
Reset an animation/simulation. This function sets the global time
back to 0 and signals the RESET event.
\end{funcdesc}

% worldObject()
\begin{funcdesc}{worldObject}{obj}
If \var{obj} is a string type this function searches the world object with
that name and returns it. If \var{obj} is not a string, it is returned
unchanged.
\end{funcdesc}

% worldObjects()
\begin{funcdesc}{worldObjects}{objs}
Similar to \function{worldObject()} but the argument may be a sequence of
objects or strings. The return value is always a list, even if only one
object was specified as input.
\end{funcdesc}

% register()
\begin{funcdesc}{register}{cmds,...}
Register functions in the cmds module. The function takes an arbitrary
number of arguments which must be callable objects. Each such function
is registered in the cmds module, so that other modules can call these
functions as if they were implemented directly in the cmds module.
\end{funcdesc}

% importDefaultPlugins
\begin{funcdesc}{importDefaultPlugins}{paths}
Import the default plugins. The plugin files/directories specified by
the \envvar{CGKIT_PLUGIN_PATH} environment variable (if it exists) are
imported.  The function already outputs error messages and returns a
list of plugin descriptors.
\end{funcdesc}

% splitPaths
\begin{funcdesc}{splitPaths}{paths}
Split a string containing paths into the individual paths. The paths
can either be separated by \character{:} or \character{;}. Windows
drive letters are maintained, even when \character{:} is used as separator.

\begin{verbatim}
>>> splitPaths("&:c:\\shaders:c:\\more_shaders;")
['&', 'c:\\shaders', 'c:\\more_shaders']    
\end{verbatim}
\end{funcdesc}