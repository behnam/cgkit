% SpotLight3DS

\section{\class{SpotLight3DS} ---
         3DS spot light}

The \class{SpotLight3DS} class stores all parameters of the spot lights
as they are stored in 3DS files. The main purpose of this class is to
ensure that data doesn't get lost during file conversions.

This light sources can also be used for RenderMan renderings. However,
not all of the parameters are currently taken into account in the
corresponding RenderMan shader (see below).

\begin{classdesc}{SpotLight3DS}{name = "SpotLight3DS",\\ 
                       enabled = True,\\
                       intensity = 1.0,\\
                       color = (1,1,1),\\
                       see_cone = False,\\
                       roll = 0.0,\\
                       outer_range = 0,\\
                       inner_range = 0,\\
                       attenuation = 0,\\
                       rectangular_spot = 0,\\
                       shadowed = False,\\
                       shadow_bias = 0,\\
                       shadow_filter = 4.0,\\
                       shadow_size = 256,\\
                       spot_aspect = 0,\\
                       use_projector = False,\\
                       projector = 0,\\
                       overshoot = False,\\
                       ray_shadows = False,\\
                       ray_bias = False,\\
                       hotspot = 43,\\
                       falloff = 45,\\
	               target = (0,0,0)
                  }

\var{enabled} is a boolean flag that can be used to turn the light source
on or off.

\var{intensity} is the overall intensity of the light source. The higher
the value, the brighter the light.

\var{color} defines the color of the light source. It must be a sequence
of 3 floats containing RGB values.

\var{see_cone} ...

\var{roll} ...

\var{inner_range} and \var{outer_range} specify an intensity range based
on distance. Any parts of the scene that are nearer to the light source
than \var{inner_range} are fully illuminated. Within the range between
\var{inner_range} and \var{outer_range} the brightness drops off to 0
and objects further than \var{outer_range} are not illuminated at all.
These range values are only taken into account when \var{attenuation}
is not 0.

\var{rectangular_spot} ...

If \var{shadowed} is set to \code{True} the light casts a shadow.

\var{shadow_bias} is a small value greater than 0 that is used to prevent
invalid self-shadowing (i.e. so that a surface element doesn't shadow itself).

\var{shadow_filter} specifies the size of the filter when doing shadow map
lookups. The higher the value the blurrier the shadow.

\var{shadow_size} defines the size of the shadow map. The shadow map will
always be square and have a width of \var{shadow_size} pixels.

\var{spot_aspect} ...

\var{use_projector} ...

\var{projector} ...

If \var{overshoot} is \code{True} the light virtually becomes a point light
source, i.e. it also illuminates the parts of the scene that lie outside
its cone but the shadow is still restricted to the cone.

\var{ray_shadows} ...

\var{ray_bias} ...

\var{hotspot} This is the angle (in degrees) of the "inner" cone which is fully illuminated.

\var{falloff} This is the angle (in degrees) of the "outer" cone. The light 
intensity between the inner and outer cone drops off to 0. The region outside
the outer cone is not illuminated (unless \var{overshoot} is activated).

\var{target} is the target point that the light source aims at.

\end{classdesc}

The following parameters are used in the corresponding RenderMan
shader, all other parameters are currently ignored:

\begin{itemize}
\item intensity
\item color
\item falloff
\item hotspot
\item attenuation
\item inner_range
\item outer_range
\item overshoot
\item shadow_bias
\item shadow_filter
\item shadow_size
\end{itemize}
