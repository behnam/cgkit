\section{\module{pluginmanager} ---
         Managing plugins}

\declaremodule{extension}{cgkit.pluginmanager}
\modulesynopsis{Managing plugins}

A {\em plugin} is a file that gets imported just like a regular Python
module. Each plugin can be uniquely identified either by its absolute
filename or its module name. The module name is only available once
the plugin is loaded. Any objects (usually classes) that are specially
marked will be made available by the plugin manager for later
retrieval.

A class gets imported if it has an attribute \code{_protocols} which is a
list of supported protocols (a protocol is identified by an arbitrary
hashable object). Each class can be uniquely identified by its module
name and class name.

Plugins are loaded by calling \function{importPlugin()} or 
\function{importPlugins()}. 
You can also register any class as plugin class any time by calling the
\function{register()} function. When a plugin is loaded or an object is
registered the plugin/object is represented by a descriptor class
(\class{PluginDescriptor} resp. \class{PluginObjectDescriptor}). 
These descriptor objects contain information about the plugin/object. The 
object descriptor also holds a reference to the actual plugin object.

You can iterate over all currently available protocols via the
\function{iterProtocols()} function. The \function{iterProtoObjects()} 
function can be used to iterate over all objects that support a
particular protocol and \function{iterObjects()} can be used to
iterate over all objects. The default iterator iterates over all
plugin descriptors.

Example plugin class:

\begin{verbatim}
# file: plugin.py
    
class PluginClass:

    _protocols = ["MyProtocol"]

    def __init__(self):
        ...
    ...
\end{verbatim}

And here's how to use it:

\begin{verbatim}
# Load the plugin
pdesc = importPlugin("plugin.py")
if pdesc.status!=STATUS_OK:
    # there was an error...

# Search a plugin class
objdesc = findObject("plugin.PluginClass")
# Create an instance of the plugin class...
PluginClass = objdesc.object
instance = PluginClass()
\end{verbatim}

The actual functionality is implemented in the \class{PluginManager}
class. However, the module creates a global instance of this class
which can be accessed by the module functions. These functions just
call the corresponding methods of the global plugin manager object.
If you want to load plugins that should be stored separately from
the plugins in your surrounding application then you can also create
your own instance of the \class{PluginManager} class.

% importPlugin
\begin{funcdesc}{importPlugin}{filename, overwrite=False}
Loads a plugin file. The given file is executed, i.e. it is imported
like a module. Any object (usually class or function) that has an
attribute \code{_protocols} is stored in the plugin manager. This
attribute is a list of protocol specifiers (which are just arbitrary
hashable objects).

The file must have a suffix that is recognized as a module by the
Python interpreter (one of the suffixes returned by
\function{imp.get_suffixes()}), otherwise an \exception{UnknownFileType}
exception is thrown.

It is possible to load a plugin several times if \var{overwrite} is set to
\code{True}. The new definitions override the previous ones. However, if
references or instances to old objects are kept somewhere they still
refer to the old definition. When writing a plugin you should always
bear in mind that the file could be executed several times and write
your initialization code accordingly.

The function returns a \class{PluginDescriptor} object which contains
information about the imported plugin.
\end{funcdesc}

% importPlugins
\begin{funcdesc}{importPlugins}{plugins, out=sys.stderr}
Import several plugins at once. \var{plugins} can be a single
file/directory name or a sequence of file/directory names. Directories
are recursively descended.
\var{out} is a stream which is used to output status messages.
\end{funcdesc}

\begin{funcdesc}{register}{obj, name=None, pdesc=None, overwrite=False}
\end{funcdesc}

\begin{funcdesc}{remove}{objdesc}
\end{funcdesc}

\begin{funcdesc}{removePlugin}{pdesc}
\end{funcdesc}

\begin{funcdesc}{removeAll}{}
\end{funcdesc}

\begin{funcdesc}{findPlugin}{filename}
\end{funcdesc}

\begin{funcdesc}{findObject}{name, modname=None}
\end{funcdesc}

\begin{funcdesc}{iterPlugins}{}
\end{funcdesc}

\begin{funcdesc}{iterProtocols}{}
\end{funcdesc}

\begin{funcdesc}{iterProtoObjects}{proto}
\end{funcdesc}

\begin{funcdesc}{iterObjects}{}
\end{funcdesc}

%---------------------------------------------------
\subsection{PluginManager class}

%---------------------------------------------------
\subsection{PluginDescriptor}

A \class{PluginDescriptor} object keeps the status information
of a plugin file. It has the following attributes:

\begin{datadesc}{filename}
The absolute file name of the plugin.
\end{datadesc}

\begin{datadesc}{filedesc}
A file descriptor tuple (\var{suffix}, \var{mode}, \var{type}) as
returned by the \function{get_suffixes()} function in the \module{imp}
module (\var{type} is one of \code{imp.PY_SOURCE}, \code{imp.PY_COMPILED} 
or \code{imp.C_EXTENSION}).
\end{datadesc}

\begin{datadesc}{objcount}
Number of imported objects (usually this should be number of classes + 
number of functions).
\end{datadesc}

\begin{datadesc}{classcount}
Number of imported classes.
\end{datadesc}

\begin{datadesc}{funccount}
Number of imported functions.
\end{datadesc}

\begin{datadesc}{status}
Status flags. If everything is ok the status is \code{STATUS_OK}, otherwise
its a combination of \code{STATUS_EXCEPTION} and \code{STATUS_DUPLICATE}.
\end{datadesc}

\begin{datadesc}{traceback}
A string containing the traceback message (if the \code{STATUS_EXCEPTION}
flag is set).
\end{datadesc}

\begin{datadesc}{objdescs}
A list of object descriptors of all imported objects.
\end{datadesc}

\begin{datadesc}{module}
The module object of the imported plugin.
\end{datadesc}

\begin{datadesc}{modulename}
The name of the plugin module.
\end{datadesc}

%---------------------------------------------------
\subsection{PluginObjectDescriptor}

A \class{PluginObjectDescriptor} object keeps the status information
of a plugin object. It has the following attributes and methods:

\begin{datadesc}{object}
The actual plugin object.
\end{datadesc}

\begin{datadesc}{name}
The name of the object.
\end{datadesc}

\begin{datadesc}{plugindesc}
The \class{PluginDescriptor} of the associated plugin.
\end{datadesc}

\begin{datadesc}{status}
Status flags. If everything is ok the status is \code{STATUS_OK}, otherwise
its \code{STATUS_DUPLICATE}.
\end{datadesc}

\begin{methoddesc}{moduleName}{}
Return the module name where the plugin object belongs to (or None).
\end{methoddesc}

\begin{methoddesc}{objectIdentifier}{}
Return the name under which the object is identified. This name is composed
of the module name and the object name, separated by a dot.
\end{methoddesc}
