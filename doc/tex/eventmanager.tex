\section{\module{eventmanager} ---
         Receiving and routing events}

\declaremodule{extension}{cgkit.eventmanager}
\modulesynopsis{Receiving and routing events}

This module defines the \class{EventManager} class which is used to
manage events. An event is something that occurs at a particular time
and that might trigger actions. Each type of event has to be
identified by a unique name and each occurrence of an event can be 
associated with arbitrary data (for example, if a key press occurs the
data contains the key code).

When dealing with events you have to consider two parts. One part is
producing the events and the other part is consuming them. And it is the 
\class{EventManager} that acts as a broker between the two. Whenever an
event occurs the producer tells the event manager about the event and
the event manager takes care of notifying anyone who showed interest in
this type of event. This way, the producer and consumer don't have to know
each other but can rely on the event manager who is always present (whereas
a particular producer or consumer might not be available all the time).

The event manager distinguishes between system wide events and normal events.
Normal events are those that have a relationship with the current scene
and that are removed if the scene is cleared. On the other hand, system wide
events are independent from the current scene and remain intact if the
scene is cleared.

There is always a global event manager instance that should be used for
managing all events:

\begin{funcdesc}{eventManager}{}
Returns the global event manager object.
\end{funcdesc}


\begin{classdesc}{EventManager}{}
Event manager class. Usually you don't have to create an instance of
this class but use the global event manager instead.
\end{classdesc}

\begin{methoddesc}{event}{name, *params, **keyargs}
Signal the event manager that an event has occurred. The type of event is
identified by the string \var{name} which must be a unique name among all
event types. The remaining arguments are the data associated with this
particular occurrence. This data is passed to the consumers.

The method returns \code{True} if any of the event handlers returned
\code{True} (i.e. the event was consumed). In this case, the notification
chain was interrupted. This means, any event handler that would have been
called after the one that returned \code{True} was not called anymore.
\end{methoddesc}

\begin{methoddesc}{connect}{name, receiver, priority=10, system=False}
Connect a function or method to an event type. Whenever an event of type
\var{name} occurs the event handler specified by \var{receiver} is called.
\var{receiver} can either be a callable such as a function or a method
or it can be an instance of a class that must implement an 
\code{on{\em name}()} method. \var{priority} determines the order in 
which the receivers are invoked. Receivers with lower values (=high priority)
are invoked first.

The argument \var{system} specifies if the connection is system wide or not.
\end{methoddesc}

\begin{methoddesc}{disconnect}{name, receiver=None, system=False}
Disconnect a function or method from the event type \var{name}. The argument
\var{receiver} has the same meaning as in the \method{connect()} method.
If \var{receiver} is \code{None} then all connections from the event
are removed. The argument \var{system} specifies if the connection is
system wide or not.
\end{methoddesc}

\begin{methoddesc}{disconnectAll}{system=False}
Remove all connections from all events. \var{system} specifies whether the
system wide connections or the scene wide connections shall be removed
\end{methoddesc}


