\section{\module{mayaascii} ---
        Reading Maya ASCII files}

\declaremodule{extension}{cgkit.mayaascii}
\modulesynopsis{Reading Maya ASCII files}

This module contains the \class{MAReader} class which can be used as a
base class for reading Maya ASCII (*.ma) files. The class reads the file
and invokes callback methods with the corresponding data in the
file. Derived classes have to implement those callback methods and
process the data as appropriate.

The module also contains a couple of helper classes that may be
used by a derived reader class to facilitate the processing of the
data.

\begin{funcdesc}{splitDAGPath}{path}
Split a Maya DAG path into its components. The path is given as a
string that may have the form <namespace>:<path> where <path> is a
sequence of strings separated by '|'.  The return value is a 2-tuple
(\var{namespace}, \var{names}) where \var{namespace} is \code{None} 
if the path did not contain a ':' character. \var{names} is a list 
of individual path names.
\end{funcdesc}

\begin{funcdesc}{stripQuotes}{s}
Remove surrounding quotes if there are any. The function returns the
string \var{s} without surrounding quotes (i.e. \code{"foo" -> foo}). 
If there are no quotes the string is returned unchanged.
\end{funcdesc}

%----------------------------------------------------------------
\subsection{MAReader class}

The \class{MAReader} class reads Maya ASCII files and calls
appropriate methods which have to be implemented in a derived class.
The content of the file is actually a subset of the {\em Maya Embedded
Language} (MEL) which is the scripting language implemented inside
Maya.  The \class{MAReader} parses the file, breaks down the content
of the file in commands and their arguments and options (expressions
are not evaluated). Each MEL command will then trigger a callback
method that has to execute the command.  These callback methods have
to be implemented in a derived class.

There are 12 MEL commands that can appear in a Maya ASCII 
file\footnote{Actually, there could appear any MEL command, but at least
Maya will only export files containing the above commands.}:

\begin{itemize}
\item {\tt file}
\item {\tt requires}
\item {\tt fileInfo}
\item {\tt currentUnit}
\item {\tt createNode}
\item {\tt setAttr}
\item {\tt addAttr}
\item {\tt connectAttr}
\item {\tt disconnectAttr}
\item {\tt parent}
\item {\tt select}
\item {\tt lockNode}
\end{itemize}

Each command has a number of arguments and can also take options. The
callback methods receive the arguments as regular arguments to the
method and the options as an additional argument \code{opts} which is
a dictionary containing the options that were specified in the
file. The key is the long name of the option (without leading dash)
and the value is a list of strings containing the option values.  The
number of values and how they have to be interpreted depend on the
actual option.

\begin{classdesc}{MAReader}{}
  Creates an instance of the reader.
\end{classdesc}

\begin{memberdesc}{filename}
The file name (if it could be obtained). This may be used for generating
warning or error messages.
\end{memberdesc}

\begin{memberdesc}{linenr}
The current line number. This may be used for generating warning or error
messages.
\end{memberdesc}

\begin{methoddesc}{abort}{}
Aborts reading the current file.
This method can be called in handler methods to abort reading the file.
\end{methoddesc}

\begin{methoddesc}{read}{f}
Read the content of a file. \var{f} is either a file like object that
can be used to read the content of the file or the name of a file.
\end{methoddesc}

\begin{methoddesc}{begin}{}
Callback method that is called before the file is read.
\end{methoddesc}

\begin{methoddesc}{end}{}
Callback method that is called after the file was read.
\end{methoddesc}

\begin{methoddesc}{onFile}{filename, opts}
Reference an external file.
\end{methoddesc}

\begin{methoddesc}{onRequires}{product, version}
Specify a requirement that is needed to load the file properly.
\var{product} is a string containing the required software component
and \var{version} is a string containing the required version of that
component.
\end{methoddesc}

\begin{methoddesc}{onFileInfo}{keyword, value, opts}
Specifies information about the file. \var{keyword} and \var{value}
are both strings.
\end{methoddesc}

\begin{methoddesc}{onCurrentUnit}{opts}
Specify the units (linear, angular, time) used in the file.
\end{methoddesc}

\begin{methoddesc}{onCreateNode}{nodetype, opts}
Create a new node. \var{nodetype} is a string specifying the type of node
that is to be created. The new node will automatically be selected (i.e.
subsequent setAttr commands refer to this node).
\end{methoddesc}

\begin{methoddesc}{onSetAttr}{attr, vals, opts}
Set a node attribute. \var{attr} is a string containing the attribute
to be set. \var{vals} is a list of values. The number of elements and
the type of each element depends on the attribute.
\end{methoddesc}

\begin{methoddesc}{onAddAttr}{opts}
Add a new attribute to the node.
\end{methoddesc}

\begin{methoddesc}{onConnectAttr}{srcattr, dstattr, opts}
Connect two attributes. \var{srcattr} is a string specifiying the
attribute that serves as a source and \var{dstattr} is the name of
the attribute that will receive the value. 
\end{methoddesc}

\begin{methoddesc}{onDisconnectAttr}{srcattr, dstattr, opts}
Break the attribute connection between two attributes.
\end{methoddesc}

\begin{methoddesc}{onParent}{objects, parent, opts}
Set the parent of one or more nodes. \var{objects} is a list of node names
and \var{parent} the name of the parent.
\end{methoddesc}

\begin{methoddesc}{onSelect}{objects, opts}
Select a node from a referenced file. \var{objects} is a list of strings
containing the node names.
\end{methoddesc}

\begin{methoddesc}{onLockNode}{objects, opts}
Lock/unlock a node. \var{objects} is a list of strings containing the node
names (the list may be empty).
\end{methoddesc}

%----------------------------------------------------------------
\subsection{DefaultMAReader class}

This class is derived from the \class{MAReader} class and implements 
the callback methods which build the graph in memory using the helper
classes (\class{Node}, etc.). You may derive from this class and only
implement the \method{end()} callback to process the graph. All nodes
of the graph can be found in the \member{nodelist} attribute.

\begin{classdesc}{DefaultMAReader}{}
  Creates an instance of the reader.
\end{classdesc}

\begin{memberdesc}{nodelist}
This list will contain all \class{Node} objects that have been created.
The order is the same as they have been encountered in the file.
\end{memberdesc}

\begin{methoddesc}{findNode}{path, create=False}
Return the \class{Node} object corresponding to a particular path.
\var{path} may also be \code{None} in which case \class{None} is returned.
If \var{create} is \code{True}, any missing node is automatically created.
\end{methoddesc}

%----------------------------------------------------------------
\subsection{Node class}

The \class{Node} class is a helper class which may be used in a
concrete implementation of the \class{MAReader} class to represent 
Maya nodes.

This class does not implement the actual functionality of a particular
Maya node, it just tracks attribute changes and connections which can
later be retrieved once the entire file was read. So this class can be
used for all Maya nodes in a file.

\begin{classdesc}{Node}{nodetype, opts, parent=None}
  \var{nodetype} and \var{opts} are the arguments of the 
  \method{onCreateNode()} callback of the \class{MAReader} class.
  \var{parent} is either None or another \class{Node} object that will 
  be the parent of the node.
\end{classdesc}

\begin{memberdesc}{nodetype}
This is a string containing the type of the node (this is the value
that was passed to the constructor).
\end{memberdesc}

\begin{memberdesc}{opts}
This is the option dictionary that was passed to the constructor
(i.e. that is used to create the node).
\end{memberdesc}

\begin{methoddesc}{getName}{}
Return the name of the node. If no node name was specified during the
creation of the object, the dummy name 'MayaNode' is returned.
\end{methoddesc}

\begin{methoddesc}{getParentName}{}
Return the name of the parent node or \code{None} if no parent was specified.
\end{methoddesc}

\begin{methoddesc}{getParent}{}
Return the parent \class{Node} object.
\end{methoddesc}

\begin{methoddesc}{setParent}{parent}
\var{parent} is either \code{None} to remove the node from its parent
or it is another \class{Node} object that will be the new parent.
\end{methoddesc}

\begin{methoddesc}{iterChildren}{}
Return an iterator that yields all children \class{Node} objects.
\end{methoddesc}

\begin{methoddesc}{setAttr}{attr, vals, opts}
\var{attr}, \var{vals} and \var{opts} are the arguments of the 
\method{onSetAttr()} callback of the \class{MAReader} class. The Python
value of an attribute can be obtained by calling \method{getAttrValue()}.
The final Python value can be retrieved with the \method{getAttrValue()}
 method.
\end{methoddesc}

\begin{methoddesc}{getAttrValue}{lname, sname, type, n=1, default=None}
Get the Python value of an attribute.
\var{lname} is the long name, \var{sname} the short name. \var{type}
is the required type and \var{n} the required number of elements 
(see the \method{Attribute.getValue()} method in section \ref{AttributeGetValue} for more information on the type). 
\var{type} and \var{n} may be None.
The return value is either a normal Python type (int, float, sequence)
or a \class{MultiAttrStorage} object in cases where the \code{setAttr}
command contained the index operator. When no attribute with the given long
or short name could be found the provided default value is returned.
\end{methoddesc}

\begin{methoddesc}{addAttr}{opts}
\var{opts} is the arguments of the \method{onAddAttr()} callback of 
the \class{MAReader} class.
\end{methoddesc}

\begin{methoddesc}{addInConnection}{localattr, nodename, attrname}
Specify an incoming connection.
\var{nodename} is the name of a node and \var{attrname} the full 
attribute name.
\end{methoddesc}

\begin{methoddesc}{addOutConnection}{localattr, node, nodename, attrname}
Specify an outgoing connection.
\var{node} is a \class{Node} object, \var{nodename} the name of the 
node and \var{attrname} the full attribute name.
\end{methoddesc}

\begin{methoddesc}{getInNode}{localattr_long, localattr_short}
Return the node and attribute that serves as input for the local
attribute with long name \var{localattr_long} and short name
\var{localattr_short}. The return value is a 2-tuple (\var{nodename},
\var{attrname}) that specifies the input connection for the 
local attribute. (\code{None}, \var{None}) is returned if there is 
no connection.
\end{methoddesc}

\begin{methoddesc}{getOutNodes}{localattr_long, localattr_short}
Return the nodes and attributes that the specified local attribute
connects to. \var{localattr_long} is the long name of the local
attribute and \var{localattr_short} its short name.
The return value is a list of 3-tuples (\var{node}, \var{nodename}, 
\var{attrname}) that specify the output connections for the
local attribute. The tuple contains the values that were previously
added with \method{addOutConnection()}.
\end{methoddesc}

\begin{methoddesc}{getOutAttr}{lname, sname, dstnodetype}
Check if a local attribute is connected to a particular type of node.
Returns a tuple (\var{node}, \var{attrname}) where \var{node} is the
\class{Node} object of the destination node and \var{attrname} the name of 
the destination attribute. If there is no connection with a node of
type \var{dstnodetype}, the method returns (\code{None}, \var{None}).

If the attribute is connected to more than one node with the given
type or to several attributes of the same node then only the first
connection encountered is returned.
\end{methoddesc}

%----------------------------------------------------------------
\subsection{Attribute class}

The \class{Attribute} class can be used to convert the value of an
attribute (as specified by the \code{setAttr} MEL command) into an
appropriate Python value.

An \class{Attribute} object is initialized with the arguments that
were passed to the \method{onSetAttr()} callback of the reader
class. The value can be retrieved using the \method{getValue()}
method. Whenever you have prior knowledge of the node you are 
currently processing you should pass the expected type of the
attribute to the \method{getValue()} method to prevent the
method from having to guess the type in case it is not specified
in the \code{setAttr} call.

\begin{classdesc}{Attribute}{attr, vals, opts}

\var{attr}, \var{vals} and \var{opts} are the arguments of the 
\method{onSetAttr()} callback of the \class{MAReader} class.

\end{classdesc}

\begin{methoddesc}{getBaseName}{}
Return the base name of the attribute. This is the first part of the
attribute name (and may actually refer to another attribute).

\begin{verbatim}
  ".t"            -> "t"
  ".ed[0:11]"     -> "ed"
  ".uvst[0].uvsn" -> "uvst"
\end{verbatim}
\end{methoddesc}

\begin{methoddesc}{getFullName}{}
Return the full attribute specifier.
\end{methoddesc}

\begin{methoddesc}{getValue}{type=None, n=None}
\label{AttributeGetValue}
Return the value of the attribute as an appropriate Python value.
\var{type} is a string containing the required type of the value.
If \code{None} is passed, the method tries to retrieve the value from
the attribute itself. If it fails, an exception is thrown. The following
table lists the valid type strings and their corresponding Python type:

\begin{tableii}{l|l}{code}{type}{Python type}
\lineii{"bool"}{bool}
\lineii{"int"}{int}
\lineii{"float"}{float}
\lineii{"string"}{str}
\lineii{"short2"}{(int, int)}
\lineii{"short3"}{(int, int, int)}
\lineii{"long2"}{(int, int)}
\lineii{"long3"}{(int, int, int)}
\lineii{"int32Array"}{[int, ...]}
\lineii{"float2"}{(float, float)}
\lineii{"float3"}{(float, float, float)}
\lineii{"double2"}{(float, float)}
\lineii{"double3"}{(float, float, float)}
\lineii{"doubleArray"}{[float, ...]}
\lineii{"polyFaces"}{PolyFace (see \ref{polyface})}
\lineii{"nurbsSurface"}{NurbsSurface (see \ref{nurbssurface})}
\lineii{"nurbsCurve"}{NurbsCurve (see \ref{nurbscurve})}
\end{tableii}

The argument \var{n} specifies how many values are expected. An exception
is thrown if the number of values that were set by the \code{setAttr} call
doesn't match the required number. If \code{None} is passed, an arbitrary
number of values is allowed. The value of \var{n} also influences the
return type. If the value is 1 the method will return one of the types
in the above table, otherwise it will return a list of the above types.

\end{methoddesc}

%----------------------------------------------------------------
\subsection{PolyFace class}
\label{polyface}

\begin{classdesc*}{PolyFace}
This class stores the data of a polygonal face. \class{PolyFace} objects
are returned when the value of a \code{polyFaces} attribute is requested.
\end{classdesc*}

The class has the following data members:

\begin{memberdesc}{f}
This is a list of integers containing the edge indices of the edges 
making up the face. If an index is negative the edge has to be
reversed (the edge index then is -i-1).
\end{memberdesc}

\begin{memberdesc}{h}
This is a list of holes. Each hole is described by a list of integers 
containing the edge indices of the edges 
making up the hole in the face. If an index is negative the edge has to be
reversed (the edge index then is -i-1).
\end{memberdesc}

\begin{memberdesc}{mf}
This is a list of texture coordinate ids of the face. This data type
is obsolete as of Maya version 3.0. It is replaced by "mu".
\end{memberdesc}

\begin{memberdesc}{mh}
This is a list of texture coordinate ids of the hole. This data type
is obsolete as of Maya version 3.0. It is replaced by "mu".
\end{memberdesc}

\begin{memberdesc}{mu}
For each loop (i.e. outer loop or hole) this list contains a list of
2-tuples (\var{uvset}, \var{ids}) where \var{uvset} is the index of
the UV set and \var{ids} the indices of the texture coordinates.
\end{memberdesc}

\begin{memberdesc}{fc}
For each loop (outer loop or hole) this list contains a list of color 
index values.
\end{memberdesc}

%----------------------------------------------------------------
\subsection{NurbsSurface class}
\label{nurbssurface}

\begin{classdesc*}{NurbsSurface}
This class stores the data of a NURBS surface. \class{NurbsSurface} objects
are returned when the value of a \code{nurbsSurface} attribute is requested.
\end{classdesc*}

The class has the following data members:

\begin{memberdesc}{udegree}
Degree in u direction.
\end{memberdesc}

\begin{memberdesc}{vdegree}
Degree in v direction.
\end{memberdesc}

\begin{memberdesc}{uform}
Form attribute for the u direction. The attribute can have one of the 
following values:

\begin{tableii}{c|l}{code}{Value}{Meaning}
\lineii{0}{Open}
\lineii{1}{Closed}
\lineii{2}{Periodic}
\end{tableii}
\end{memberdesc}

\begin{memberdesc}{vform}
Form attribute for the v direction (see above).
\end{memberdesc}

\begin{memberdesc}{isrational}
This attribute is \code{True} if the surface contains a rational component.
In this case, the control vertices are given as 4-tuples, otherwise
as 3-tuples. 
\end{memberdesc}

\begin{memberdesc}{uknots}
This is a list of floats containing the knot values for the u direction.
\end{memberdesc}

\begin{memberdesc}{vknots}
This is a list of floats containing the knot values for the v direction.
\end{memberdesc}

\begin{memberdesc}{cvs}
A list of control vertices. Each vertex is given either as a 3-tuple or
4-tuple of floats.
\end{memberdesc}

%----------------------------------------------------------------
\subsection{NurbsCurve class}
\label{nurbscurve}

\begin{classdesc*}{NurbsCurve}
This class stores the data of a NURBS curve. \class{NurbsCurve} objects
are returned when the value of a \code{nurbsCurve} attribute is requested.
\end{classdesc*}

The class has the following data members:

\begin{memberdesc}{degree}
The degree of the curve.
\end{memberdesc}

\begin{memberdesc}{spans}
The number of spans.
\end{memberdesc}

\begin{memberdesc}{form}
Form attribute. The attribute can have one of the following values:

\begin{tableii}{c|l}{code}{Value}{Meaning}
\lineii{0}{Open}
\lineii{1}{Closed}
\lineii{2}{Periodic}
\end{tableii}
\end{memberdesc}

\begin{memberdesc}{isrational}
This attribute is \code{True} if the curve contains a rational component.
In this case, the control vertices have one additional value.
\end{memberdesc}

\begin{memberdesc}{dimension}
The dimension of the curve (2 or 3).
\end{memberdesc}

\begin{memberdesc}{knots}
This is a list of floats containing the knot values.
\end{memberdesc}

\begin{memberdesc}{cvs}
A list of control vertices. Each vertex is a tuple of 2, 3 or 4 floats
(the actual number depends on the \var{dimension} and \var{isrational}
settings).
\end{memberdesc}
