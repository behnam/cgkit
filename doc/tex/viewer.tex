% Viewer tool

\section{The interactive viewer tool}

The tool \file{viewer.py} reads one or more files and displays the
scene using a simple OpenGL renderer. It creates events such as
keyboard, mouse or joystick events that can be used by the components
in the scene.

If the environment variable \envvar{VIEWER_DEFAULT_OPTIONS} exists, it
is read and parsed to set the default options. After that the options
in the command line are parsed.

{\bf Usage:}

\begin{verbatim}
viewer.py [options] inputfiles
\end{verbatim}

To exit the viewer press \kbd{Escape} or close the window. How to
navigate in the scene depends on the navigation mode (see option -N).

{\bf Options:}

\begin{description}
\item[\code{-f<int>} / \code{--fps=<int>}] 
Specifies the frame rate that should be tried to maintain while playing
back the scene (default: 30).

\item[\code{-h} / \code{--help}]
Print a help message.

\item[\code{-F} / \code{--full-screen}]
Open a full screen display.

\item[\code{-W<int>} / \code{--width=<int>}]
Specify the width in pixels of the window/screen (default: 640).

\item[\code{-H<int>} / \code{--height=<int>}]
Specify the height in pixels of the window/screen (default: 480). If
no height is specified it is computed based on the width and an aspect
ratio of 4/3.

\item[\code{-v} / \code{--verbose}]
Output info messages.

\item[\code{-p<str>} / \code{--plugin=<str>}]
If the argument specifies a file this file is loaded as a plugin, if it
is a directory, all files in this directory are loaded. You can specify
this option more than once or you can use a comma separated list of names.

You can also set files and paths via the environment variable
\envvar{CGKIT_PLUGIN_PATH}. The files/paths have to be separated either
by ':' or ';'.

\item[\code{-c<str>} / \code{--camera=<str>}]
Select the camera that should be used to view the scene. The argument
is the name of the camera. If no camera is specified the first camera
found in the scene is used.

\item[\code{-b<time>} / \code{--begin=<time>}]
Specify a time or frame where the animation/simulation should be started
(default: 0s). You can add the time unit such as 's' for seconds or 'f'
for frames. Frames are assumed if you don't specify a unit.

\item[\code{-e<time>} / \code{--end=<time>}]
Specify a time or frame where the animation/simulation should be stopped.
You can add the time unit such as 's' for seconds or 'f' for frames. 
Frames are assumed if you don't specify a unit.

\item[\code{-U<axis>} / \code{--up=<axis>}]
Specify the default value for the up axis. The argument may be either
'y' or 'z'.

\item[\code{-O<str>} / \code{--option=<str>}]
Add a global option that will be stored in the scene. The argument should
have the form "name=value". Using this option is equivalent to writing
"Globals( name=value )" in a scene file.

\item[\code{-t} / \code{--set-time}]
Set the starting time directly instead of stepping there from time 0.

\item[\code{-S<str>} / \code{--stereo=<str>}]
Activate stereo display. The argument specifies the method that should be
used to create a stereo image. It can be either \var{vsplit} or \var{glstereo}.

\item[\code{-D<float>} / \code{--eye-distance=<float>}]
Default eye distance for stereo display (default 0.07). This distance
value is used if the camera does not explicitly specify a distance value
(i.e. if it has no attribute \var{eye_distance}).

\item[\code{-B} / \code{--bounding-box}]
Display the bounding boxes.

\item[\code{-P<str>} / \code{--polygon-mode=<str>}]
Specify how polygons should be rendered. Possible values are
\var{fill} (default), \var{line} and \var{point}.

\item[\code{-s<str>} / \code{--save=<str>}]
When this option is specified each frame is saved under the given name
(+ frame number). The extension determines the image format.

\item[\code{-N<str>} / \code{--navigation-mode=<str>}]
Specify which navigation mode to activate. The viewer can emulate the
navigation of a few common 3D packages. Possible values are
\var{Maya} (default), \var{MAX} and \var{Softimage} (case-insensitive).

In Maya mode you navigate by pressing the \kbd{Alt} key in combination
with any of the three mouse buttons. In MAX mode you press the middle
mouse button either alone or in combination with \kbd{Alt} or \kbd{Control}
and \kbd{Alt}. In Softimage mode, only the Softimage Navigation Tool is
emulated (it's as if this tool is permanently active), i.e. you navigate
by using one of the three mouse buttons.

\item[\code{-X} / \code{--disable-spacedevice}]
Disables support for SpaceMouse/SpaceBall. This option can be used if there
are any problems with the driver or initialization takes too long.

\item[\code{-T} / \code{--disable-wintab}]
Disables tablet support. This option can be used if there are any
problems with the driver or initialization takes too long.

\end{description}

{\bf Events:}

The viewer tool generates the following user input events (see the
module \refmodule{cgkit.events} for more details about these events):

\begin{itemize}
\item \code{KEY_PRESS}
\item \code{KEY_RELEASE}
\item \code{LEFT_DOWN}
\item \code{MIDDLE_DOWN}
\item \code{RIGHT_DOWN}
\item \code{MOUSE_BUTTON_DOWN}
\item \code{LEFT_UP}
\item \code{MIDDLE_UP}
\item \code{RIGHT_UP}
\item \code{MOUSE_BUTTON_UP}
\item \code{MOUSE_WHEEL}
\item \code{MOUSE_MOVE}
\item \code{JOYSTICK_AXIS}
\item \code{JOYSTICK_BALL}
\item \code{JOYSTICK_HAT}
\item \code{JOYSTICK_BUTTON_DOWN}
\item \code{JOYSTICK_BUTTON_UP}
\item \code{SPACE_MOTION}
\item \code{SPACE_BUTTON_DOWN}
\item \code{SPACE_BUTTON_UP}
\item \code{SPACE_BUTTON_ZERO}
\item \code{TABLET}
\end{itemize}

{\bf Timing:}

The operations per frame are as follows (in this order):

\begin{enumerate}
\item Render and display the current scene at time $t$
\item Handle events
\item Step frame (i.e. increase the time by $\Delta t$ and signal the
  \code{STEP_FRAME} event)
\item Sync to the specified framerate
\end{enumerate}
