% BezierCurveGeom

\section{\class{BezierCurveGeom} ---
         Piecewise cubic Bezier curve}
\label{beziercurvegeom}

The \class{BezierCurveGeom} class represents a piecewise cubic curve
in 3D space that is composed of an arbitrary number of cubic Bezier
segments. The class stores a number of 3D points that are interpolated
by the curve. Each point has an in tangent and an out tangent associated
with it that define how the curve enters and leaves the point.

\begin{classdesc}{BezierPoint}{pos, intangent=vec3(0), outtangent=vec3(0)}

This class just stores a position and the in and out tangents and
is used to pass these parameters to the constructor of a 
\class{BezierCurveGeom}.

\var{pos} is a point position that the curve will interpolate.

\var{intangent} and \var{outtangent} define where the curve enters
and leaves the point.
\end{classdesc}


\begin{classdesc}{BezierCurveGeom}{pnts = None,\\
                                   closed = False,\\
                                   epsilon = 0.01,\\
                                   subdiv = 4,\\
                                   show_tangents = False}

\var{pnts} is a list of \class{BezierPoint} objects describing the
points to interpolate and the in and out tangents.

If \var{closed} is set to \code{True} the last point will be connected
to the first point.

\var{epsilon} is a threshold value that determines the accuracy of
length calculations of the curve.

\var{subdiv} is the number of subdivisions that are made to draw the
curve using OpenGL.

If \var{show_tangents} is set to \code{True} the OpenGL visualization
will also show the in and out tangents.
\end{classdesc}

A \class{BezierCurveGeom} has the following slots:

\begin{tableiv}{l|l|c|l}{code}{Slot}{Type}{Access}{Description}
\lineiv{pnts_slot}{[vec3]}{rw}{The curve points}
\lineiv{intangents_slot}{[vec3]}{rw}{The in tangents}
\lineiv{outtangents_slot}{[vec3]}{rw}{The out tangents}
\end{tableiv}

\begin{memberdesc}{closed}
This is a boolean indicating wheter the curve is closed or not. You
can read and write this attribute.
\end{memberdesc}

\begin{memberdesc}{numsegs}
The number of Bezier segments in the curve. You can only read this
attribute.
\end{memberdesc}

\begin{memberdesc}{paraminterval}
This is a tuple (\var{t_min}, \var{t_max}) containing the valid parameter
interval of the curve. You can only read this attribute.
\end{memberdesc}

% eval
\begin{methoddesc}{eval}{t}
Evaluate the curve at parameter \var{t} and return the curve point.
\end{methoddesc}

% evalFrame
\begin{methoddesc}{evalFrame}{t}
Evaluate the curve at parameter \var{t} and return the curve point,
the tangent and the second derivative.
\end{methoddesc}

% deriv
\begin{methoddesc}{deriv}{t}
Return the first derivative (the tangent) at parameter \var{t}.
\end{methoddesc}

% arcLen
\begin{methoddesc}{arcLen}{t}
Return the arc length of the curve up to the point specified by the parameter
\var{t}.
\end{methoddesc}

% length
\begin{methoddesc}{length}{}
Return the entire length of the curve. This is equivalent to 
\code{arcLen(t_max)}.
\end{methoddesc}






