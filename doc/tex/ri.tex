\section{\module{ri} ---
         Generic RenderMan binding to produce RIB}

\declaremodule{extension}{cgkit.ri}
\modulesynopsis{Generic RenderMan binding to produce RIB}

The RenderMan\textsuperscript{\textregistered} interface is an API
that is used to communicate a 3D scene description (which includes 3D
geometry, light sources, a camera description, etc) to a renderer
which will produce a 2D image of that scene. The API itself is
independent of a particular renderer and can be used for any renderer
that adheres to the RenderMan standard.
There are some excellent renderers freely available, such
as \ulink{3Delight}{http://www.3delight.com/} or the Open Source renderer 
\ulink{Aqsis}{http://aqsis.sourceforge.net/} or
\ulink{Pixie}{http://pixie.sourceforge.net/}. On the
commercial side, the most popular renderers are Pixar's
\ulink{Photorealistic RenderMan}{http://renderman.pixar.com/} (PRMan), 
\ulink{RenderDotC}{http://www.dotcsw.com/} and 
\ulink{AIR}{http://www.sitexgraphics.com/}. 
The RenderMan interface was
created by Pixar and the official specification can be downloaded from
their site.

This document is not an introduction to the RenderMan interface
itself, it just explains the usage of this particular Python
binding. The binding was written to be compliant to v3.2 of Pixar's
RenderMan Interface specification. However, it also supports some
newer features such as string handles for light sources or object
instances.

There is another RenderMan module called \refmodule[cgkit.cri]{cri} that
interfaces a renderer directly. Almost everything that is said in this section
applies to the \module{cri} module as well.

\subsection{Using the API}

It is safe to import the module using 

\begin{verbatim}
from cgkit.ri import * 
\end{verbatim}

All the functions that get imported start with the prefix \code{Ri},
all constants start with \code{RI_} or \code{RIE_}, so you probably
won't get into a naming conflict.

After importing the module this way you can use the functions just as
you're used to from the C API (well, almost).

\begin{verbatim}
from cgkit.ri import *

RiBegin(RI_NULL)
RiWorldBegin()
RiSurface("plastic")
RiSphere(1,-1,1,360)
RiWorldEnd()
RiEnd()
\end{verbatim}

The parameter to \function{RiBegin()} determines where the output is
directed to. You can pass one of the following:

\begin{itemize}
\item \code{RI_NULL} or an empty string: The RIB stream will be written to 
stdout. 
\item A name that contains the suffix \code{".rib"} (case insensitive): 
A file with the given name is created and the RIB stream is written into it. 
\item A name that contains the suffix \code{".rib.gz"} (case insensitive):
Same as before, but the stream is compressed. The result is the same as if you
would output a RIB file and then compress it using gzip.  
\item A name without the suffix \code{".rib"} or \code{".rib.gz"}: The name
is supposed to be an external program that reads RIB from stdin. 
The program is launched and the RIB stream is piped into it.
\end{itemize}

Note: When using the \module{cri} module you first have to load a library
and invoke the functions on the returned handle (see the section on the
\refmodule[cgkit.cri]{cri} module for more information about that). The
interpretation of the argument to \function{RiBegin()} is then dependent on the
renderer you are using.
%-----------
\subsection{Online documentation}

Every function has an associated doc string that includes a short
description of the function, some information about what parameters
the function expects and an example how the function is called.

Example (inside an interactive Python session):

\begin{verbatim}
>>> from ri import *
>>> help(RiPatch)
RiPatch(type, paramlist)

    type is one of RI_BILINEAR (4 vertices) or RI_BICUBIC (16 vertices).

    Number of array elements for primitive variables:
    -------------------------------------------------
    constant: 1              varying: 4
    uniform:  1              vertex:  4/16 (depends on type)

    Example: RiPatch(RI_BILINEAR, [0,0,0, 1,0,0, 0,1,0, 1,1,0])
\end{verbatim}

or from the shell (outside the Python shell):

\begin{verbatim}
> pydoc ri.RiCropWindow

Python Library Documentation: function RiCropWindow in ri

RiCropWindow(left, right, bottom, top)
    Specify a subwindow to render.

    The values each lie between 0 and 1.

    Example: RiCropWindow(0.0, 1.0 , 0.0, 1.0)  (renders the entire frame)
             RiCropWindow(0.5, 1.0 , 0.0, 0.5)  (renders the top right quarter)
\end{verbatim}

%-----------
\subsection{Differences between the C and Python API}

The Python RenderMan binding is rather close to the C API, however
there are some minor differences you should know about.

{\bf Types}

In this binding typing is not as strict as in the C API. For compatibility
reasons, the RenderMan types (RtBoolean, RtInt, RtFloat, etc.) do exist
but they are just aliases to the corresponding built-in Python types and
you never have to use them explicitly.
In the ctypes-based \module{cri} module, the types refer to the respective
ctypes types and you may want to use them occasionally to construct arrays.

Wherever the API expects vector types (RtPoint, RtMatrix, RtBound,
RtBasis) you can use any value that can be interpreted as a sequence
of the corresponding number of scalar values. These can be lists,
tuples or your own class that can be used as a sequence.

It is also possible to use nested sequences instead of flat ones. For
example, you can specify a matrix as a list of 16 values or as a list
of four 4-tuples. The following two calls are identical:

\begin{verbatim}
RiConcatTransform([2,0,0,0, 0,2,0,0, 0,0,2,0, 0,0,0,1]) 

RiConcatTransform([[2,0,0,0], [0,2,0,0], [0,0,2,0], [0,0,0,1]])
\end{verbatim}

{\bf Parameter lists}

When passing parameter lists you have to know the following points:

\begin{itemize}
\item In C parameter lists have to be terminated with \code{RI_NULL}. In Python
this is not necessary, the functions can determine the number of
arguments themselves. However, adding \code{RI_NULL} at the end of the list
will not generate an error. For example, if you are porting C code to
Python you don't have to change those calls. So the following two
calls are both valid:

\begin{verbatim}
RiSurface("plastic", "kd", 0.6, "ks", 0.4)
RiSurface("plastic", "kd", 0.6, "ks", 0.4, RI_NULL) 
\end{verbatim}

\item The tokens inside the parameter list have to be declared (either
inline or using \function{RiDeclare}), otherwise an error is
generated. Standard tokens (like \code{RI_P}, \code{RI_CS}, ...) are
already pre-declared.

\item Parameter lists can be specified in several ways. The first way is the
familiar one you already know from the C API, that is, the token and
the value are each an individual parameter:

\begin{verbatim}
RiSurface("plastic", "kd", 0.6, "ks", 0.4) 
\end{verbatim}

Alternatively, you can use keyword arguments: 

\begin{verbatim}
RiSurface("plastic", kd=0.6, ks=0.4) 
\end{verbatim}

But note that you can't use inline declarations using keyword
arguments. Instead you have to previously declare those variables
using \function{RiDeclare}. Also, you can't use keyword arguments if
the token is a reserved Python keyword (like the standard \code{"from"}
parameter).  The third way to specify the parameter list is to provide
a dictionary including the token/value pairs:

\begin{verbatim}
RiSurface("plastic", {"kd":0.6, "ks":0.4}) 
\end{verbatim}

This is useful if you generate the parameter list on the fly in your program.

\end{itemize}

{\bf Arrays}

In the C API functions that take arrays as arguments usually take the
length of the array as a parameter as well. This is not necessary in
the Python binding. You only have to provide the array, the length can
be determined by the function.

For example, in C you might write: 

\begin{verbatim}
RtPoint points[4] = {0,1,0, 0,1,1, 0,0,1, 0,0,0};
RiPolygon(4, RI_P, (RtPointer)points, RI_NULL); 
\end{verbatim}

The number of points has to be specified explicitly. In Python
however, this call could look like this:

\begin{verbatim}
points = [0,1,0, 0,1,1, 0,0,1, 0,0,0]
RiPolygon(RI_P, points) 
\end{verbatim}

The functions that are affected by this rule are: 

\begin{verbatim}
RiBlobby()
RiColorSamples()
RiCurves()
RiGeneralPolygon()
RiMotionBegin()
RiPoints()
RiPointsGeneralPolygons()
RiPointsPolygons()
RiPolygon()
RiSubdivisionMesh()
RiTransformPoints()
RiTrimCurve()
\end{verbatim}

When using the \module{cri} module it is particularly advantageous to pass
arrays as ctypes arrays or numpy arrays. In this case, no data conversion is
required which makes the function call considerably faster (particularly for
large amounts of data).

\begin{verbatim}
# Creating a ctypes array of floats
points = (12*RtFloat)(0,1,0, 0,1,1, 0,0,1, 0,0,0)

# Creating a numpy array of floats
points = numpy.array([0,1,0, 0,1,1, 0,0,1, 0,0,0], dtype=numpy.float32)
\end{verbatim}

{\bf User defined functions}

Some RenderMan functions may take user defined functions as input
which will be used during rendering. When using the \module{cri} module to
link to an actual RenderMan library you can use Python functions in
addition to the standard functions. However, in the case of the
generic (\module{ri}) module, you can only use the predefined
standard functions.
 
{\em Filter functions}

It is not possible to use your own filter functions in combination
with the \module{ri} module, you have to use
one of the predefined filters:

\begin{itemize}
\item RiGaussianFilter 
\item RiBoxFilter 
\item RiTriangleFilter 
\item RiSincFilter 
\item RiCatmullRomFilter 
\end{itemize}

{\em Procedurals}

It is not possible to use your own procedurals directly in the RIB
generating program, you can only use one of the predefined procedural
primitives:

\begin{itemize}
\item RiProcDelayedReadArchive 
\item RiProcRunProgram 
\item RiProcDynamicLoad
\end{itemize}

However, this is not really a restriction since you always can use
RiProcRunProgram to invoke your Python program that generates
geometry.

{\bf Extended transformation functions}

The transformation functions \function{RiTranslate()}, \function{RiRotate()}, 
\function{RiScale()} and \function{RiSkew()} have been extended in a way 
that is not part of the official
spec. Each of these functions takes one or two vectors as input which
usually are provided as 3 separate scalar values, like the axis of a
rotation for example:

\begin{verbatim}
RiRotate(45, 0,0,1) 
\end{verbatim}

Now in this implementation you can choose to provide such vectors as
sequences of 3 scalar values:

\begin{verbatim}
RiRotate(45, [0,0,1]) 

axis = vec3(0,0,1)
RiRotate(45, axis)
\end{verbatim}

{\bf Empty stubs}

In the \module{ri} module, the function RiTransformPoints() always
returns None and never transforms points (as the module just outputs
RIB and does not maintain transformations matrices).
In the \module{cri} module, on the other hand, the function is available
and can be used to transform points.

%-----------
\subsection{Implementation specific options}

There is currently one option that is specific to this RenderMan
binding and that won't produce any RIB call but will control what gets
written to the output stream:

\begin{funcdesc}{RiOption}{RI_RIBOUTPUT, RI_VERSION, 0}
If this option is set to 0 directly after \function{RiBegin()} is
called, then no \code{"version"} call will be generated in the RIB stream
(default is 1).\\
New in version 1.1.
\end{funcdesc}

%-----------
\subsection{Error handling}

Besides the three standard error handlers RiErrorIgnore, RiErrorPrint
(default) and RiErrorAbort the module provides an additional error
handler called RiErrorException. Whenever an error occurs
RiErrorException raises the exception \exception{RIException}.

If you install a new error handler with \function{RiErrorHandler()}
only the three standard error handlers will produce an output in the
RIB stream, if you install RiErrorException or your own handler then
the handler is installed but no RIB output is produced.

The module does some error checking, however there are still quite a
bit of possible error cases that are not reported. For example, the
module checks if parameters are declared, but it is not checked if you
provide the correct number of values. In general, the module also does
not check if a function call is valid in a given state (e.g. the
module won't generate an error if you call \function{RiFormat()}
inside a world block).


