\section{\module{wintab} ---
          Wrapper around the Wintab{\texttrademark} Developer Kit}

\declaremodule{extension}{cgkit.wintab}
\modulesynopsis{Wrapper around the Wintab Developer Kit}

The Wintab{\texttrademark} specification is an open industry standard
interface that provides access to pointing devices such as a pen
tablet, for example. The API was developed by
\ulink{LCS/Telegraphics}{http://www.pointing.com/}.

The \module{wintab} module is a wrapper around the
Wintab{\texttrademark} Developer Kit and can be used to add tablet
support to your Python application. Before you can get any data from
the tablet you have to create an instance of the \class{Context}
class, set the desired parameters and open the context. Once the
context is open you will either receive messages from the tablet
driver or you can poll the current tablet state. Either way, you will
receive the tablet data via \class{Packet} objects that contain all
the data that was generated by the tablet. See also the {\em Wintab
Interface Specification} at \url{http://www.pointing.com/} for more
detailed usage information.

The constants used in this module are either available in the 
\module{wintab} module or in the \module{wintab.constants} module.
The latter can be used if you want to import only the constants into
your namespace.

% available()
\begin{funcdesc}{available}{}
Returns \code{True} if the Wintab functionality is available. As the Wintab
Developer Kit is only available on Windows, this function will always
return \code{False} on other operating systems. 

On Windows, this function can still return \code{False} if either the
Wintab drivers are not installed or cgkit was compiled without Wintab
support.

If this function returns \code{False}, an exception will be raised
whenever you try to call another function or instantiate a class from
this module.
\end{funcdesc}

% info()
\begin{funcdesc}{info}{category}
This function returns a dictionary with global information about the 
interface. \var{category} specifies the category from which information
is being requested. It can be one of the values in the following table:

\begin{tableii}{l|l}{code}{Category}{Description}
\lineii{WTI_INTERFACE}{Global interface identification and capability information}
\lineii{WTI_STATUS}{Current interface resource usage statistics}
\lineii{WTI_DEFCONTEXT}{...}
\lineii{WTI_DEFSYSCTX}{...}
\lineii{WTI_DEVICES+n}{Capability and status information for a device}
\lineii{WTI_CURSORS+n}{Capability and status information for a cursor}
\lineii{WTI_EXTENSIONS+n}{Descriptive information and defaults for an extension}
\lineii{WTI_DDCTXS}{...}
\lineii{WTI_DSCTXS}{...}
\end{tableii}
\end{funcdesc}


\begin{notice}[note]
The module uses the Wintab{\texttrademark} Programmer's Kit which can be
found at \url{http://www.pointing.com/}.

{\em The Wintab Programmer's Kit is copyright 1991-1998 by LCS/Telegraphics.}
\end{notice}

%------------------------------------------------------------------
\subsection{Context class}

The \class{Context} class provides the interface to the tablet driver.

\begin{classdesc}{Context}{}
The class takes no parameters. All the context attributes will be
initialized with the default values as provided by the driver.
\end{classdesc}

Context attributes:

\begin{memberdesc}{name}
Context name.
\end{memberdesc}

\begin{memberdesc}{options}
Specifies options for the context and must be a combination of the following
flags:

\begin{tableii}{l|l}{code}{Option}{Description}
\lineii{CXO_SYSTEM}{The context is a system cursor context}
\lineii{CXO_PEN}{The context is a Pen Windows (and system cursor) context}
\lineii{CXO_MESSAGES}{The context sends WT_PACKET messages to its owner}
\lineii{CXO_MARGIN}{The input context will have a margin}
\lineii{CXO_MGNINSIDE}{The margin will be inside the specified context}
\lineii{CXO_CSRMESSAGES}{The context sends WT_CSRCHANGE messages to its owner}
\end{tableii}
\end{memberdesc}

\begin{memberdesc}{status}
Specifies current status conditions for the context. The status value
is a combination of the following bits:

\begin{tableii}{l|l}{code}{Status bit}{Description}
\lineii{CXS_DISABLED}{The context has been disabled}
\lineii{CXS_OBSCURED}{The context is at least partially obscured by another context}
\lineii{CXS_ONTOP}{The context is the topmost context}
\end{tableii}
\end{memberdesc}

\begin{memberdesc}{locks}
Specifies which attributes of the context the application wishes to be 
locked. The value can be a combination of the following bits:

\begin{tableii}{l|l}{code}{Lock}{Description}
\lineii{CXL_INSIZE}{The input size cannot be changed}
\lineii{CXL_INASPECT}{The input aspect ration cannot be changed}
\lineii{CXL_MARGIN}{The margin options cannot be changed}
\lineii{CXL_SENSITIVITY}{The sensitivity settings for x, y and z cannot be changed}
\lineii{CXL_SYSOUT}{The system pointing control variables cannot be changed}
\end{tableii}
\end{memberdesc}

\begin{memberdesc}{msgbase}
Base number for the message IDs.
\end{memberdesc}

\begin{memberdesc}{device}
Specifies the device whose input the context processes.
\end{memberdesc}

\begin{memberdesc}{pktrate}
Specifies the desired packet report rate in Hertz. Once the context is open,
this field will contain the actual report rate.
\end{memberdesc}

\begin{memberdesc}{pktdata}
Specifies which optional data items will be in packets returned from the
context.
\end{memberdesc}

\begin{memberdesc}{pktmode}
Specifies whether the packet data items will be returned in absolute
or relative mode. If the item's bit is set in this field, the item will
be returned in relative mode.
\end{memberdesc}

\begin{memberdesc}{movemask}
Specifies which packet data items can generate move events in the context.
\end{memberdesc}

\begin{memberdesc}{btndnmask}
Specifies the buttons for which button press events will be processed
in the context.
\end{memberdesc}

\begin{memberdesc}{btnupmask}
Specifies the buttons for which button release events will be processed
in the context.
\end{memberdesc}

\begin{memberdesc}{inorgx}
Specifies the origin of the context's input area in the tablet's
native coordinates along the x axis.
\end{memberdesc}

\begin{memberdesc}{inorgy}
Specifies the origin of the context's input area in the tablet's
native coordinates along the y axis.
\end{memberdesc}

\begin{memberdesc}{inorgz}
Specifies the origin of the context's input area in the tablet's
native coordinates along the z axis.
\end{memberdesc}

\begin{memberdesc}{inextx}
Specifies the extent of the context's input area in the tablet's native
coordinates along the x axis.
\end{memberdesc}

\begin{memberdesc}{inexty}
Specifies the extent of the context's input area in the tablet's native
coordinates along the y axis.
\end{memberdesc}

\begin{memberdesc}{inextz}
Specifies the extent of the context's input area in the tablet's native
coordinates along the z axis.
\end{memberdesc}

\begin{memberdesc}{outorgx}
Specifies the extent of the context's output area in context output
coordinates along the x axis.
\end{memberdesc}

\begin{memberdesc}{outorgy}
Specifies the extent of the context's output area in context output
coordinates along the y axis.
\end{memberdesc}

\begin{memberdesc}{outorgz}
Specifies the extent of the context's output area in context output
coordinates along the z axis.
\end{memberdesc}

\begin{memberdesc}{outextx}
Specifies the extent of the context's output area in context output
coordinates along the x axis.
\end{memberdesc}

\begin{memberdesc}{outexty}
Specifies the extent of the context's output area in context output
coordinates along the y axis.
\end{memberdesc}

\begin{memberdesc}{outextz}
Specifies the extent of the context's output area in context output
coordinates along the z axis.
\end{memberdesc}

\begin{memberdesc}{sensx}
Specifies the relative-mode sensitivity factor for the x axis.
\end{memberdesc}

\begin{memberdesc}{sensy}
Specifies the relative-mode sensitivity factor for the y axis.
\end{memberdesc}

\begin{memberdesc}{sensz}
Specifies the relative-mode sensitivity factor for the z axis.
\end{memberdesc}

\begin{memberdesc}{sysmode}
Specifies the system cursor tracking mode. \code{True} specifies absolute,
\code{False} means relative.
\end{memberdesc}

\begin{memberdesc}{sysorgx}
Specifies the origin in screen coordinates of the screen mapping area
for system cursor tracking.
\end{memberdesc}

\begin{memberdesc}{sysorgy}
Specifies the origin in screen coordinates of the screen mapping area
for system cursor tracking.
\end{memberdesc}

\begin{memberdesc}{sysextx}
Specifies the extent in screen coordinates of the screen mapping area
for system cursor tracking.
\end{memberdesc}

\begin{memberdesc}{sysexty}
Specifies the extent in screen coordinates of the screen mapping area
for system cursor tracking.
\end{memberdesc}

\begin{memberdesc}{syssensx}
Specifies the system-cursor relative-mode sensitivity factor for the
x axis.
\end{memberdesc}

\begin{memberdesc}{syssensy}
Specifies the system-cursor relative-mode sensitivity factor for the
y axis.
\end{memberdesc}

Other attributes:

\begin{memberdesc}{queuesize}
The number of packets the context's queue can hole. Setting this attribute
may result in an exception if the queue size could not be set. In such a
case, you must try again with a smaller value.
\end{memberdesc}

\begin{memberdesc}{id_packet}
The final id of the WT_PACKET message.
\end{memberdesc}

\begin{memberdesc}{id_csrchange}
The final id of the WT_CSRCHANGE message.
\end{memberdesc}

\begin{memberdesc}{id_ctxopen}
The final id of the WT_CTXOPEN message.
\end{memberdesc}

\begin{memberdesc}{id_ctxclose}
The final id of the WT_CTXCLOSE message.
\end{memberdesc}

\begin{memberdesc}{id_ctxupdate}
The final id of the WT_CTXUPDATE message.
\end{memberdesc}

\begin{memberdesc}{id_ctxoverlap}
The final id of the WT_CTXOVERLAP message.
\end{memberdesc}

\begin{memberdesc}{id_proximity}
The final id of the WT_PROXIMITY message.
\end{memberdesc}

\begin{memberdesc}{id_infochange}
The final id of the WT_INFOCHANGE message.
\end{memberdesc}

Methods:

\begin{methoddesc}{open}{hwnd, enable}
Opens the context. \var{hwnd} is the window handle (as integer) of the
window that owns the context and that receives messages from the context.
\var{enable} is a boolean that specifies whether the context will
immediately begin processing input data.

Set the context attributes to the desired values before calling this
method. Modifying context attributes after the context was opened is
possible with the \method{set()} method.
\end{methoddesc}

\begin{methoddesc}{restore}{hwnd, saveinfo, enable}
This method is equivalent to the \method{open()} method with the only
difference that it takes the context attributes via the binary
\var{saveinfo} string which was returned by the \method{save()} method
in a previous session.
\end{methoddesc}

\begin{methoddesc}{close}{}
Closes the context if it was open (or do nothing if it was already closed
or if it was not open at all).
\end{methoddesc}

\begin{methoddesc}{save}{}
Returns a binary \var{saveinfo} string containing the current context state.
This string can be used as argument to the \method{restore()} method.
\end{methoddesc}

\begin{methoddesc}{packet}{serial}
Returns the packet with the specified serial number.
\end{methoddesc}

\begin{methoddesc}{enable}{flag}
If \var{flag} is \code{True} the context is enabled, otherwise it is disabled.
Returns \code{True} if the operation was successful.
\end{methoddesc}

\begin{methoddesc}{overlap}{totop}
If \var{totop} is \code{True} the context will become the topmost context,
otherwise it will be send to the bottom.
Returns \code{True} if the operation was successful.
\end{methoddesc}

\begin{methoddesc}{config}{hwnd=0}
Prompts the user for changes to the context via a dialog box. \var{hwnd}
is the window handle of the window that will be parent of the dialog.
If it is 0, the context owning window will be used.
The return value is \code{True} if the context was changed.
\end{methoddesc}

\begin{methoddesc}{get}{}
Update the local context attributes.
\end{methoddesc}

\begin{methoddesc}{set}{}
Update the tablet context with the settings stored in the local context
attributes.
\end{methoddesc}

\begin{methoddesc}{packetsGet}{maxpkts}
Return the next \var{maxpkts} packets and remove them from the queue.
\end{methoddesc}

\begin{methoddesc}{packetsPeek}{maxpkts}
Return the next \var{maxpkts} packets without removing them from the queue.
\end{methoddesc}

\begin{methoddesc}{dataGet}{begin, end, maxpkts}
Return all packets with serial numbers between \var{begin} and \var{end}
inclusive and remove them from the queue. However, no more than \var{maxpkts}
packets are returned.
The return value is a 2-tuple (\var{numpackets}, \var{packet list}) where
\var{numpackets} is the total number of packets found in the queue between
\var{begin} and \var{end}.
\end{methoddesc}

\begin{methoddesc}{dataPeek}{begin, end, maxpkts}
This is the same as \method{dataGet()} but it doesn't remove the packets
from the queue.
\end{methoddesc}

\begin{methoddesc}{queuePacketsEx}{}
Returns the serial numbers of the oldest and newest packets currently
in the queue.
\end{methoddesc}

%------------------------------------------------------------------
\subsection{Packet class}

\begin{classdesc}{Packet}{}
\end{classdesc}

\begin{memberdesc}{pktdata}
This is a copy of the \code{pktdata} attribute of the corresponding
context, i.e. this attribute determines which of the following attributes
are actually valid.
\end{memberdesc}

\begin{memberdesc}{context}
Specifies the context that generated the event.
\end{memberdesc}

\begin{memberdesc}{status}
Contains a combination of the following status and error conditions:

\begin{tableii}{l|l}{code}{Status bit}{Description}
\lineii{TPS_PROXIMITY}{The cursor is out of the context}
\lineii{TPS_QUEUE_ERR}{The event queue for the context has overflowed}
\lineii{TPS_MARGIN}{The cursor is in the margin of the context}
\lineii{TPS_GRAB}{The cursor is out of the context, but the context has grabbed input}
\lineii{TPS_INVERT}{The cursor is in its inverted state}
\end{tableii}
\end{memberdesc}

\begin{memberdesc}{time}
In absolute mode, specifies the system time at which the event was posted.
In relative mode, specifies the elapsed time in milliseconds since the last
packet.
\end{memberdesc}

\begin{memberdesc}{changed}
Specifies which of the included packet data items have changed since the
previously posted event.
\end{memberdesc}

\begin{memberdesc}{serial}
Contains a serial number assigned to the packet by the context.
\end{memberdesc}

\begin{memberdesc}{cursor}
Specifies which cursor type generated the packet.
\end{memberdesc}

\begin{memberdesc}{buttons}
In absolute, contains the current button state. In relative mode, the low
word contains a button number and the high word contains one of the following
codes: \code{TBN_NONE} if there was no change in button state, \code{TBN_UP}
if the button was released or \code{TBN_DOWN} if the button was pressed.
\end{memberdesc}

\begin{memberdesc}{x}
In absolute mode, contains the scaled cursor location along the x axis.
In relative mode, contains the scaled change in cursor position.
\end{memberdesc}

\begin{memberdesc}{y}
In absolute mode, contains the scaled cursor location along the y axis.
In relative mode, contains the scaled change in cursor position.
\end{memberdesc}

\begin{memberdesc}{z}
In absolute mode, contains the scaled cursor location along the z axis.
In relative mode, contains the scaled change in cursor position.
\end{memberdesc}

\begin{memberdesc}{normalpressure}
In absolute mode, contains the adjusted state of the normal pressure.
In relative mode, contains the change in adjusted pressure state.
\end{memberdesc}

\begin{memberdesc}{tangentpressure}
In absolute mode, contains the adjusted state of the tangential pressure.
In relative mode, contains the change in adjusted pressure state.
\end{memberdesc}

\begin{memberdesc}{orient_azimuth}
Specifies the clockwise rotation of the cursor about the z axis through
a full circular range.
\end{memberdesc}

\begin{memberdesc}{orient_altitude}
Specifies the angle with the x-y plane through a signed, semicircular range.
\end{memberdesc}

\begin{memberdesc}{orient_twist}
Specifies the clockwise rotation of the cursor about its own major axis.
\end{memberdesc}

\begin{memberdesc}{rot_pitch}
Specifies the pitch of the cursor.
\end{memberdesc}

\begin{memberdesc}{rot_roll}
Specifies the roll of the cursor.
\end{memberdesc}

\begin{memberdesc}{rot_yaw}
Specifies the yaw of the cursor.
\end{memberdesc}


