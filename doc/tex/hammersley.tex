% Hammersley

\section{\module{hammersley} ---
          Generating Hammersley and Halton points}

\declaremodule{extension}{cgkit.hammersley}
\modulesynopsis{Generating Hammersley and Halton points}

This module contains functions to generate points that are uniformly
distributed and stochastic-looking on either a unit square or a unit
sphere. The Hammersley point set is more uniform but is
non-hierarchical, i.e. for different n arguments you get an entirely
new sequence. If you need hierarchical behavior you can use the Halton
point set.

This is a Python version of the implementation provided in:

Tien-Tsin Wong, Wai-Shing Luk, Pheng-Ann Heng\\
{\em Sampling with Hammersley and Halton points}\\
Journal of Graphics Tools, Vol. 2, No. 2, 1997, pp. 9-24\\
\url{http://www.acm.org/jgt/papers/WongLukHeng97/}\\
\url{http://www.cse.cuhk.edu.hk/~ttwong/papers/udpoint/udpoints.html}

% planeHammersley
\begin{funcdesc}{planeHammersley}{n}
Yields a sequence of \var{n} tuples (\var{x}, \var{y}) which represent a
point on the unit square. The sequence of points for a particular \var{n} is
always the same. When \var{n} changes an entirely new sequence will be
generated.
    
This function uses a base of 2.
\end{funcdesc}

% sphereHammersley
\begin{funcdesc}{sphereHammersley}{n}
This function yields \var{n} \class{vec3} objects representing points
on the unit sphere. The sequence of points for a particular \var{n} is
always the same. When \var{n} changes an entirely new sequence will be
generated.

This function uses a base of 2.    
\end{funcdesc}

% planeHalton
\begin{funcdesc}{planeHalton}{n, p2=3}
This function yields a sequence of two floats (\var{x}, \var{y}) which
represent a point on the unit square. The number of points to generate
is given by \var{n}. If \var{n} is set to \code{None}, an infinite
number of points is generated and the caller has to make sure the loop
stops by checking some other critera.  The sequence of generated
points is always the same, no matter what \var{n} is (i.e. the first n
elements generated by the sequence \code{planeHalton(n+1)} is identical to
the sequence \var{planeHalton(n)}).

This function uses 2 as its first prime base whereas the second
prime \var{p2} (which must be a prime number) can be provided by the user.
\end{funcdesc}

% sphereHalton
\begin{funcdesc}{sphereHalton}{n, p2=3}
This function yields a sequence of \class{vec3} objects representing
points on the unit sphere. The number of points to generate is given
by \var{n}. If \var{n} is set to \code{None}, an infinite number of
points is generated and the caller has to make sure the loop stops by
checking some other critera. The sequence of generated points is
always the same, no matter what \var{n} is (i.e. the first n elements
generated by the sequence \code{sphereHalton(n+1)} is identical to the
sequence \code{sphereHalton(n)}).

This function uses 2 as its first prime base whereas the second base
\var{p2} (which must be a prime number) can be provided by the user.
\end{funcdesc}

% ---Copyright---
\begin{notice}[note]
The original C versions of these functions are distributed under the
following license:
  
(c) Copyright 1997, Tien-Tsin Wong\\
ALL RIGHTS RESERVED\\
Permission to use, copy, modify, and distribute this software for
any purpose and without fee is hereby granted, provided that the above
copyright notice appear in all copies and that both the copyright notice
and this permission notice appear in supporting documentation,
 
THE MATERIAL EMBODIED ON THIS SOFTWARE IS PROVIDED TO YOU "AS-IS"
AND WITHOUT WARRANTY OF ANY KIND, EXPRESS, IMPLIED OR OTHERWISE,
INCLUDING WITHOUT LIMITATION, ANY WARRANTY OF MERCHANTABILITY OR
FITNESS FOR A PARTICULAR PURPOSE.  IN NO EVENT SHALL THE AUTHOR
BE LIABLE TO YOU OR ANYONE ELSE FOR ANY DIRECT,
SPECIAL, INCIDENTAL, INDIRECT OR CONSEQUENTIAL DAMAGES OF ANY
KIND, OR ANY DAMAGES WHATSOEVER, INCLUDING WITHOUT LIMITATION,
LOSS OF PROFIT, LOSS OF USE, SAVINGS OR REVENUE, OR THE CLAIMS OF
THIRD PARTIES, WHETHER OR NOT THE AUTHOR HAS BEEN
ADVISED OF THE POSSIBILITY OF SUCH LOSS, HOWEVER CAUSED AND ON
ANY THEORY OF LIABILITY, ARISING OUT OF OR IN CONNECTION WITH THE
POSSESSION, USE OR PERFORMANCE OF THIS SOFTWARE.
\end{notice}
